\chapter{有理型函数}%第6章

\begin{mysimplebox}{問1}
    円環$0<|c_1|<|z|<|c_2|$における$\frac{1}{(z-c_1)(z-c_2)}$のLaurent展開を求めよ。
\end{mysimplebox}
\paragraph{解答}
部分分数展開によって、
\begin{align*}
    \frac{1}{(z-c_1)(z-c_2)}=\frac{1}{c_1-c_2}\left(\frac{1}{z-c_1}-\frac{1}{z-c_2}\right)
\end{align*}
となる。

ここで、円環において、次が成り立つ。
\begin{align*}
    0<\left|\frac{c_1}{z}\right|<1,\quad0<\left|\frac{z}{c_2}\right|<1
\end{align*}

よって、
\begin{align*}
    \frac{1}{z-c_1}
    &=\frac{1}{z}\cdot\frac{1}{1-\frac{c_1}{z}}
    =\frac{1}{z}\left\{1+\frac{c_1}{z}+\left(\frac{c_1}{z}\right)^2+\cdots\right\}\\
    \frac{1}{z-c_2}
    &=-\frac{1}{c_2}\cdot\frac{1}{1-\frac{z}{c_2}}
    =-\frac{1}{c_2}\left\{1+\frac{z}{c_2}+\left(\frac{z}{c_2}\right)^2+\cdots\right\}
\end{align*}

ゆえに、
\begin{align*}
    \frac{1}{(z-c_1)(z-c_2)}
    =\frac{1}{c_1-c_2}\left(\cdots+\frac{c_1^2}{z^3}+\frac{c_1}{z^2}+\frac{1}{z}+\frac{1}{c_2}+\frac{z}{c_2^2}+\frac{z^2}{c_2^3}+\cdots\right)
\end{align*}
(解答終)

\begin{mysimplebox}{問2}
    $f(z)$が整函数で$|f(z)|\le|\sin z|$ならば$f(z)\equiv A\sin z$($A$は定数)であることを示せ。
\end{mysimplebox}
\paragraph{証明}

\begin{align*}
    g(z)=\frac{f(z)}{\sin z}
\end{align*}
とする。$g(z)$は複素数平面上の$\sin z$の零点($z=n\pi$、ただし$n$は整数)を除いた領域で定義された函数である。その領域を$D$とする。
すなわち、
\begin{align*}
    D:=\C\setminus\{n\pi\mid n\in\Z\}
\end{align*}
である。

$|f(z)|\le|\sin z|$から、$D$において$|g(z)|\le1$である。

$c$は$\sin z$の零点のうちの一つとする。たとえば$c=\pi$とすればよい。以下の議論はそれ以外の零点でも同様に成り立つ。

実数$r$は$0<r<\pi$を満たすとする。$0<|z-c|<r$において、$g(z)$は正則であり、次のLaurent展開
\begin{align*}
    g(z)=\sum_{k=0}^{\infty}a_k(z-c)^k+\sum_{k=1}^{\infty}\frac{a_{-k}}{(z-c)^k}
\end{align*}
が可能である。

よって、
\begin{align*}
    |a_{-k}|
    &=\left|\frac{1}{2\pi i}\int_{|z-c|=r}g(z)(z-c)^{k-1}dz\right|\\
    &\le\frac{1}{2\pi}\int_{|z-c|=r}\left|g(z)(z-c)^{k-1}\right||dz|\\
    &\le\frac{1}{2\pi}r^{k-1}\cdot2\pi r
    =r^k
\end{align*}
が成り立つ。

$r\longrightarrow0$の極限をとることで、$a_{-k}=0$($k\ge1$)であることがわかる。

よって、Laurent展開の式から
\begin{align*}
    g(z)=\sum_{k=0}^{\infty}a_k(z-c)^k
\end{align*}
である。ゆえに、$z=c$は除去可能特異点であり、
\begin{align*}
    \lim_{z\rightarrow c}g(z)=a_0
\end{align*}
が成り立つ($D$において$|g(z)|\le1$であるから、$|a_0|\le1$である)。

よって、$g(c)=a_0$と定義する。他の零点でも同様にすれば、$g(z)$は全平面において正則である($z=\infty$においても$|g(z)|\le1$である)。

したがって、Liouvilleの定理から$g(z)$は定数であることがわかり、証明が完了する。(証明終)

\paragraph{補足1}
この結果はRiemannの特異点除去定理と呼ばれる有名なものである。

\paragraph{補足2}
$|f(z)|\le|\sin z|$から$\sin z$の零点はすべて$f(z)$の零点でもある。そうであるなら、
\begin{align*}
    f(z)=A_1\sin z+A_2\sin 2z+A_3\sin 3z+\dots+A_k\sin kz
\end{align*}
となることもあり得るのでは?と思ってしまう。

しかし、$z=-it$とし、$t$は十分大きい正の実数、$k>1$とすると、
\begin{align*}
    \left|\sin kz\right|
    &=\left|\sin(-ikt)\right|
    =\frac{e^{kt}-e^{-kt}}{2}
    >\frac{e^{t}-e^{-t}}{2}
    =|\sin(-it)|
\end{align*}
が成り立つ。

よって、$|f(z)|\le|\sin z|$が成り立たなくなる点が存在することになってしまう。
ゆえに、2倍音以上の正弦函数は現れないのである。

\paragraph{補足3}
\begin{align*}
    \lim_{z\longrightarrow n\pi}\frac{z-n\pi}{\sin z}
    =\pm\lim_{z\longrightarrow n\pi}\frac{z-n\pi}{\sin(z-n\pi)}
    =\pm1
\end{align*}
であるから、$n\pi$は$\frac{1}{\sin z}$の1位の極である。

$|f(z)|\le|\sin z|$より、$f(z)$は$n\pi$に零点をもつから、
$\frac{f(z)}{\sin z}$は$n\pi$において除去可能特異点をもつ。


\begin{mysimplebox}{問3}
    $f(z)$が$z=\infty$において$m$位の零点または極を有するとき、$f'(z)$は$z=\infty$において零点または極を有するか。
\end{mysimplebox}
\paragraph{解答}
$f(z)$が$z=\infty$で$m$位の零点を有するとき
\begin{align*}
    f(z)&=\cdots+\frac{c_{-(m+1)}}{z^{m+1}}+\frac{c_{-m}}{z^{m}}\\
    f(z)&=\cdots-(m+1)\frac{c_{-(m+1)}}{z^{m+2}}-m\frac{c_{-m}}{z^{m+1}}
\end{align*}
よって、任意の$m\ge1$に対して、$f'(z)$も$z=\infty$で零点を有する。

$f(z)$が$z=\infty$で$m$位の極を有するとき
\begin{align*}
    f(z)&=\cdots+c_{m-1}z^{m-1}+c_mz^m\\
    f(z)&=\cdots+(m-1)c_{m-1}z^{m-2}+mc_mz^{m-1}
\end{align*}
よって、$m\ge2$ならば$f'(z)$も$z=\infty$で極を有する。
$m=1$ならば$f'(z)$は$z=\infty$に極をもたない。
(解答終)

\begin{mysimplebox}{問4}
    単連結な領域$D$から有限個の点$c_1,c_2,\dots,c_k$を取り除いた残りの領域$D'$で、$f(z)$が正則で$\lim_{z\longrightarrow c_\nu}(z-c_\nu)f(z)=0$($\nu=1,2,\dots,k$)であるとき、$C$が$D'$内の長さのある閉曲線ならば、
    \begin{align*}
        \int_Cf(z)dz=0
    \end{align*}
    が成り立つことを示せ。
\end{mysimplebox}
\paragraph{証明}
$\nu=1,2,\dots,k$に対して、$c_\nu$を中心とする$D$内の$k$個の円$C_\nu$が存在する。さらに、どの2つの円も交わらないようにできる。閉曲線$C$で囲まれた内部に点$c_1,c_2,\dots,c_k$があるとしてよい(点$c_1,c_2,\dots,c_k$のうち、閉曲線$C$の内部にない点がある場合、そういった点を無視しても以下の議論では差し支えない)。

$D'$で$f(z)$は正則であるから
\begin{align*}
    \int_Cf(z)dz
    =\sum_{\nu=1}^{k}\int_{C_\nu}f(z)dz
\end{align*}
である。

円$C_\nu$の半径を$r_\nu$とすると、$0<|z-c_\nu|<r_\nu$であるとき、
\begin{align*}
    f(z)
    =\sum_{n=0}^{\infty}a_{\nu,n}(z-c_\nu)^n
    +\sum_{n=1}^{\infty}\frac{a_{\nu,-n}}{(z-c_\nu)^n}
\end{align*}
とLaurent展開できる。このとき、
\begin{align*}
    \int_{C_\nu}f(z)dz=2\pi i a_{\nu,-1}
\end{align*}
であるから、
\begin{align*}
    \int_Cf(z)dz
    =2\pi i \sum_{\nu=1}^{k}a_{\nu,-1}
\end{align*}
が成り立つ。

ここで、$\lim_{z\longrightarrow c_\nu}(z-c_\nu)f(z)=0$であるから、任意の$\epsilon>0$に対して、ある$\delta>0$が存在し、$0<|z-c_\nu|<\delta$ならば、
\begin{align*}
    |(z-c_\nu)f(z)|<\epsilon
\end{align*}
が成り立つ。

よって、
\begin{align*}
    |a_{\nu,-2}|
    &=\frac{1}{2\pi}\left| \int_{C_\nu}(z-c_\nu)f(z)dz\right|
    \le \frac{1}{2\pi}\int_{C_\nu}|z-c_\nu|\left|f(z)\right||dz|\\
    &\le \frac{1}{2\pi}\epsilon\cdot2\pi \delta=\epsilon \delta
\end{align*}
であり、$\epsilon \delta$はいくらでも小さくできるから、$a_{\nu,-2}=0$である。

同様に、$k\ge2$に対して、
\begin{align*}
    |a_{\nu,-k}|
    &=\frac{1}{2\pi}\left| \int_{C_\nu}(z-c_\nu)^{k-1}f(z)dz\right|
    \le \frac{1}{2\pi}\int_{C_\nu}|z-c_\nu|^{k-1}\left|f(z)\right||dz|\\
    &\le \frac{1}{2\pi}\epsilon^{k-1}\cdot2\pi \delta=\epsilon^{k-1} \delta
\end{align*}
であり、$\epsilon^{k-1} \delta$はいくらでも小さくできるから、$a_{\nu,-k}=0$である。

よって、$0<|z-c_\nu|<\delta$において、
\begin{align*}
    f(z)
    =\sum_{n=0}^{\infty}a_{\nu,n}(z-c_\nu)^n
    +\frac{a_{\nu,-1}}{(z-c_\nu)}
\end{align*}
である。さらに、
\begin{align*}
    \lim_{z\longrightarrow c_\nu}(z-c_\nu)f(z)
    &=\lim_{z\longrightarrow c_\nu}\left\{\sum_{n=0}^{\infty}a_{\nu,n}(z-c_\nu)^{n+1}
    +a_{\nu,-1}\right\}\\
    &=a_{\nu,-1}=0
\end{align*}


ゆえに、
\begin{align*}
    \int_Cf(z)dz=0
\end{align*}
が成り立つ。(証明終)

\paragraph{別解}
任意の$l\in\N$に対して、$\lim_{z\longrightarrow c_\nu}(z-c_\nu)f(z)=0$であるから、書籍p.117のRiemannの定理より、$c_\nu$は$(z-c_\nu)f(z)$の除去可能な特異点である。よって、$c_\nu$のある近傍において、
\begin{align*}
    (z-c_\nu)f(z)
    =\sum_{k=0}^{\infty}b_{\nu,k}(z-c_\nu)^k
\end{align*}
と表される。

再び$\lim_{z\longrightarrow c_\nu}(z-c_\nu)f(z)=0$から、$b_{\nu,0}=0$である。ゆえに、
\begin{align*}
    f(z)
    =\sum_{k=0}^{\infty}b_{\nu,k+1}(z-c_\nu)^k
\end{align*}
である。以上から、$\int_Cf(z)dz=0$が言える。(証明終)

\begin{mysimplebox}{問5}
    $-1<\Re(z)\le0$ならば$\Gamma(z)=\frac{\Gamma(z+1)}{z}$とおき、次に、$-2<\Re(z)\le-1$ならば$\Gamma(z)=\frac{\Gamma(z+1)}{z}$とおくというふうに、この手順を繰り返していくと$\Gamma(z)$を平面$|z|<+\infty$全体で定義することができる。こうして定義された$\Gamma(z)$は有理型函数であるが、その極はどのような点か。
\end{mysimplebox}
\paragraph{解答}
$\Re(z)>0$において$\Gamma(z)$は正則で、$\Gamma(z+1)=z\Gamma(z)$が成り立つことはp.113の問13で示した。

$-1<\Re(z)\le0$において$\Gamma(z)=\frac{\Gamma(z+1)}{z}$とおく。
\begin{align*}
    \Gamma(1)
    &=\int_{0}^{\infty}e^{-x}dx=\left[-e^{-x}\right]_0^\infty=1
\end{align*}
であるから、$z=0$は1位の極である。

同様に、$-2<\Re(z)\le-1$において$\Gamma(z)=\frac{\Gamma(z+1)}{z}$とおく。
\begin{align*}
    \lim_{z\longrightarrow-1}(z+1)\Gamma(z)
    &=\lim_{z\longrightarrow-1}\frac{(z+1)\Gamma(z+1)}{z}
    =-\lim_{z\longrightarrow0}z\Gamma(z)=-1
\end{align*}
であるから、$z=-1$も1位の極である。

$-3<\Re(z)\le-2$において$\Gamma(z)=\frac{\Gamma(z+1)}{z}$とおく。
\begin{align*}
    \lim_{z\longrightarrow-2}(z+2)\Gamma(z)
    &=\lim_{z\longrightarrow-2}\frac{(z+2)\Gamma(z+1)}{z}
    =\lim_{z\longrightarrow-2}\frac{(z+2)\Gamma(z+2)}{z(z+1)}=\frac{1}{-2(-1)}=\frac{1}{2}
\end{align*}
であるから、$z=-2$も1位の極である。

\begin{align*}
    (z+k)\Gamma(z)
    &=\frac{(z+k)\Gamma(z+1)}{z}
    =\frac{(z+k)\Gamma(z+2)}{z(z+1)}
    =\cdots=\frac{(z+k)\Gamma(z+k)}{z(z+1)\cdots(z+k-1)}\\
    &\longrightarrow \frac{(-1)^k}{k!}\quad(z\longrightarrow-k)
\end{align*}
であるから、$z=-k$($k\in\Z_{\ge0}$)は1位の極である。(解答終)

\begin{mysimplebox}{問6}
    立体射影は等角な写像である。
    すなわち、球面$S$上の二つの円の交角はその立体射影である平面上の二つの円(直線)の交角に等しいことを示せ。
\end{mysimplebox}
\paragraph{証明(この証明は間違っている)}
立体射影を$\pi$と書くことにする。
\begin{align*}
    \pi\colon S &\to \C\\
    (\xi,\eta,\zeta) &\mapsto \left(\frac{\xi}{1-\zeta},\frac{\eta}{1-\zeta}\right)\quad((\xi,\eta,\zeta)\neq(0,0,1))\\
    (0,0,1)&\mapsto\infty
\end{align*}
$S$の二つの円がともにN$(0,0,1)$とS$(0,0,0)$を通るならば、$\pi$による像は$\C$上の直線であり、$\pi$によって等角に移されることは明らかである。

p.130の記述をヒントにする。

$\C$上の円または直線
\begin{align}
    a(x^2+y^2)+2bx+2cy+d=0 \label{eq:6en}
\end{align}
には、空間における平面
\begin{align}
    a\zeta+2b\xi+2c\eta+d(1-\zeta)=0 \label{eq:6S}
\end{align}
と$S$の交線である円が対応する。

$(\ref{eq:6S})$を変形すると
\begin{align*}
    2b\xi+2c\eta+(a-d)\zeta+d=0
\end{align*}
とできるから、平面$(\ref{eq:6S})$の法線ベクトルは
\begin{align*}
    \bm{n}=\begin{bmatrix}
        2b \\ 2c \\ a-d
    \end{bmatrix}
\end{align*}
である。

$k=1,2$に対し、
\begin{align*}
    \bm{n}_k=\begin{bmatrix}
        2b_k \\ 2c_k \\ a_k-d_k
    \end{bmatrix}
\end{align*}
とする。
\begin{align*}
    \cos\theta
    =\frac{\bm{n}_1\cdot\bm{n}_2}{|\bm{n}_1||\bm{n}_2|}
    =\frac{4b_1b_2+4c_1c_2+(a_1-d_1)(a_2-d_2)}{\prod_{k=1,2}\sqrt{4b_k^2+4c_k^2+(a_k-d_k)^2}}
\end{align*}
$\theta$は二つの法線$\bm{n}_1,\bm{n}_2$の角度である。この角を$S$上の二つの円の交角とする\footnote{法線のなす角を二つの円の交角とするのは誤りであり、この証明は正しくない。}。ただし、二つの円が交わると仮定している。

$a\neq0$とすると、$(\ref{eq:6en})$から
\begin{align*}
    \left(x+\frac{b}{a}\right)^2
    +\left(y+\frac{c}{a}\right)^2
    =\frac{b^2+c^2-ad}{a^2}
\end{align*}
このような$\C$上の二つの円が交わるとし、その交点におけるそれぞれの接線のなす角を求める?

(保留)←この証明は間違っている。次の証明を参照。
\newpage
\paragraph{証明}
球面上の交わる円弧の一部を$AB$、$CD$とする。その交点を$P$とする。
さらに、$P,A,B,C,D$の立体射影による像を$P',A',B',C',D'$とする。
%\begin{align*}
%    \pi(P)&=P'\\
%    \pi(A)&=A'\\
%    \pi(B)&=B'\\
%    \pi(C)&=C'\\
%    \pi(D)&=D'
%\end{align*}
%とおく。

円弧$AB$の点$P$における接線と$\zeta=1$平面との交点を$Q$とする。同様に、円弧$CD$の点$P$における接線と$\zeta=1$平面との交点を$R$とする。

線分$PQ$と$NQ$はどちらも球面の接線であるから、$|PQ|=|NQ|$である。同様に、$|PR|=|NR|$である。

よって、$\triangle PQR\equiv\triangle NQR$であるから、$\angle QPR=\angle QNR$である。

また、
$\overrightarrow{QQ'}
=\overrightarrow{RR'}
=\overrightarrow{NP'}$とする。
$Q',R'$は$\zeta=0$平面上の点である。

$P,N,Q,P',Q'$は同一平面上にある。
同様に、$P,N,R,P',R'$は同一平面上にある。

$\angle Q'P'R'=\angle QNR$であるから、
$\angle Q'P'R'=\angle QPR$である。

直線$P'Q'$は円弧$A'B'$の$P'$における接線である。
同様に、直線$P'R'$は円弧$C'D'$の$P'$における接線である。

以上で、立体射影が等角写像であることが示された。
(証明終)

\begin{figure}[h]
    \centering
    \includegraphics[width=10cm]{chap6_fig/prob6.png}
    \caption{立体射影は等角写像}
    \label{fig:prob6}
\end{figure}

\newpage
\begin{mysimplebox}{問7}
    (44.1)、(44.2)を示せ。
\end{mysimplebox}
\paragraph{証明}
球面$S$の半径を$r$とおく。$r=\frac{1}{2}$である。

$S$の中心の座標は$(0,0,r)$である。

P$(\xi,\eta,\zeta)$は$S$上の点であるから
\begin{align*}
    \xi^2+\eta^2+(\zeta-r)^2=r^2
\end{align*}
が成り立つ。

\begin{figure}[h]
    \centering
    \includegraphics[width=5cm]{chap6_fig/ONP.jpg}
    \caption{ONP平面}
    \label{fig:ONP}
\end{figure}

図$\ref{fig:ONP}$から、次の比の計算ができる。
\begin{align*}
    (1-\zeta):1&=\sqrt{\xi^2+\eta^2}:\sqrt{x^2+y^2}\\
    \sqrt{x^2+y^2}&=\frac{\sqrt{\xi^2+\eta^2}}{1-\zeta}
    =\frac{\sqrt{r^2-(\zeta-r)^2}}{1-\zeta}
    =\frac{\sqrt{\zeta(1-\zeta)}}{1-\zeta}\\
    x^2+y^2&=\frac{\zeta}{1-\zeta}
\end{align*}

\begin{figure}[h]
    \centering
    \includegraphics[width=5cm]{chap6_fig/xON.jpg}
    \caption{xON平面}
    \label{fig:xON}
\end{figure}

$x$ON平面への射影である図$\ref{fig:xON}$から、次の比の計算ができる。
\begin{align*}
    (1-\zeta):1&=\xi:x\\
    \xi&=x(1-\zeta)\\
    x&=\frac{\xi}{1-\zeta}
\end{align*}

同様にして、
\begin{align*}
    (1-\zeta):1&=\eta:y\\
    \eta&=y(1-\zeta)\\
    y&=\frac{\eta}{1-\zeta}
\end{align*}

上で得られた式から、$\zeta,\xi,\eta$について次のように書ける。
\begin{align*}
    (x^2+y^2)(1-\zeta)&=\zeta\\
    x^2+y^2-(x^2+y^2)\zeta&=\zeta\\
    x^2+y^2&=(1+x^2+y^2)\zeta\\
    \zeta&=\frac{x^2+y^2}{1+x^2+y^2}\\
    1-\zeta&=\frac{1}{1+x^2+y^2}\\
    \xi&=x(1-\zeta)=\frac{x}{1+x^2+y^2}\\
    \eta&=y(1-\zeta)=\frac{y}{1+x^2+y^2}
\end{align*}
(証明終)
\newpage
\begin{mysimplebox}{問8}
    平面上の2点$z,z'$に対応する球面$S$上の2点の距離を$[z,z']$で表すと、$[z,z']$は次の式で表される:
    \begin{align*}
        [z,z']&=\frac{|z-z'|}{\sqrt{(1+|z|^2)(1+|z'|^2)}}
        \quad(z\neq\infty,z'\neq\infty)\\
        [z,z']&=\frac{1}{\sqrt{1+|z|^2}}
        \quad(z\neq\infty,z'=\infty)
    \end{align*}
\end{mysimplebox}
\paragraph{証明}

\begin{figure}[h]
    \centering
    \includegraphics[width=5cm]{chap6_fig/Sdis.jpg}
    \caption{$[z,z']$}
    \label{fig:Sdis}
\end{figure}

図$\ref{fig:Sdis}$のように、$\theta=\angle zNz'$とする。
\begin{align*}
    zN&=\sqrt{1+|z|^2}\\
    z'N&=\sqrt{1+|z'|^2}\\
    1:(1-\zeta)=\sqrt{1+x^2+y^2}:NP\\
    NP&=(1-\zeta)\sqrt{1+x^2+y^2}=\frac{1}{\sqrt{1+x^2+y^2}}=\frac{1}{\sqrt{1+|z|^2}}\\
    NP'&=\frac{1}{\sqrt{1+|z'|^2}}
\end{align*}
$\triangle zNz'$において、余弦定理から、
\begin{align*}
    \cos\theta=\frac{|z-z'|^2-(1+|z|^2)-(1+|z'|^2)}{2\sqrt{1+|z|^2}\sqrt{1+|z'|^2}}
\end{align*}

$\triangle PNP'$において、余弦定理から、
\begin{align*}
    PP'^2
    &=NP^2+NP'^2+2NP\cdot NP'\cos\theta\\
    &=\frac{1}{1+|z|^2}+\frac{1}{1+|z'|^2}+\frac{|z-z'|^2-(1+|z|^2)-(1+|z'|^2)}{(1+|z|^2)(1+|z'|^2)}\\
    &=\frac{|z-z'|^2}{(1+|z|^2)(1+|z'|^2)}
\end{align*}

よって、
\begin{align*}
    [z,z']&=PP'=\frac{|z-z'|}{\sqrt{(1+|z|^2)(1+|z'|^2)}}
    \quad(z\neq\infty,z'\neq\infty)
\end{align*}
である。

$z'=\infty$ならば、
\begin{align*}
    [z,z']&=NP=\frac{1}{\sqrt{1+|z|^2}}
\end{align*}
である。
(証明終)

\paragraph{別証明}
44.1式を利用して示せる。入力未。

\newpage
\begin{mysimplebox}{問9}
    $\epsilon$がどの正数でも$m\ge N(\epsilon),n\ge N(\epsilon)$ならば$[z_m,z_n]<\epsilon$であるように自然数$N(\epsilon)$を定めうるとき、数列$\{z_n\}$は球面収束するという。$\{z_n\}$が球面収束するためには$\{z_n\}$が極限値$\alpha$($|\alpha|\le+\infty$)をもつことが必要十分であることを示せ。
\end{mysimplebox}
\paragraph{証明}
(必要性)

$\{|z_n|\mid n\in\N\}$が有界であるとき、$\sup_{n\in\N}|z_n|=M<\infty$とする。

$m,n\ge N(\epsilon)$ならば、
\begin{align*}
    [z_m,z_n]
    &=\frac{|z_m-z_n|}{\sqrt{(1+|z_m|^2)(1+|z_n|^2)}}
    <\epsilon
\end{align*}

よって、
\begin{align*}
    |z_m-z_n|<\epsilon\sqrt{(1+|z_m|^2)(1+|z_n|^2)}
    \le\epsilon\sqrt{(1+M^2)(1+M^2)}=\epsilon(1+M^2)
\end{align*}
となるが、$\epsilon$が任意の正数であるから、この最右辺はいくらでも小さくできる。よって、$\{z_n\}$は収束して極限値をもつ。

$\{|z_n|\mid n\in\N\}$が有界でないとき、$z_n\longrightarrow\infty$であることは明らか。

(十分性)

$z\longrightarrow\alpha$($|\alpha|<\infty$)のとき

\begin{align*}
    0\le[z_n,z_m]
    =\frac{|z_m-z_n|}{\sqrt{(1+|z_m|^2)(1+|z_n|^2)}}
    \le|z_m-z_n|\longrightarrow0\quad(m,n\longrightarrow\infty)
\end{align*}

$z\longrightarrow\infty$のときは

\begin{align*}
    [z_m,z_n]
    &=\frac{|z_m-z_n|}{\sqrt{(1+|z_m|^2)(1+|z_n|^2)}}
    \le\frac{|z_m|+|z_n|}{\sqrt{(1+|z_m|^2)(1+|z_n|^2)}}\\
    &\le\frac{\sqrt{1+|z_m|^2}+\sqrt{1+|z_n|^2}}{\sqrt{(1+|z_m|^2)(1+|z_n|^2)}}\\
    &=\frac{1}{\sqrt{1+|z_m|^2}}+\frac{1}{\sqrt{1+|z_n|^2}}
    =[z_m,\infty]+[z_n,\infty]\longrightarrow0
    \quad(m,n\longrightarrow\infty)
\end{align*}
(証明終)
\newpage
\paragraph{別証明}
球面距離と平面の距離が同値な距離であることを利用する。入力未。

\newpage
\begin{mysimplebox}{問10}
    点集合$E$で定義された函数$f_n(z)\ (n=1,2,\dots)$は$\infty$なる値をとり得るものとし、$E$の各点で数列$\{f_n(z)\}$が球面収束するときは、函数列$\{f_n(z)\}$は$E$で球面収束するという。

    また、$f(z)=\lim_{n\to\infty}f_n(z)$とおくと、$z$が$E$のどの点でも、また$\epsilon$がどの正数でも、$n\ge N(\epsilon)$に対しては$[f_n(z),f(z)]<\epsilon$であるように$z$に無関係な自然数$N(\epsilon)$を選び得るとき、函数列$\{f_n(z)\}$は$E$で``球面一様収束''するという。

    $f_n(z)\ (n=1,2,\dots)$が領域$D$で有理型で、$D$内のどの閉集合でも函数列$\{f_n(z)\}$が球面一様収束するとき($D$で``広義の球面一様収束''するとき)、極限函数$f(z)$は$D$で至る所$\infty$に等しいか、または$D$で有理型函数である。これを示せ。
\end{mysimplebox}
\paragraph{証明}
$c\in D$において$f(c)=\infty$とする。
$D$内で$f(z)\neq\infty$となる点$z$が存在するならば、$f(z)$は有理型であることを示す。

\begin{align*}
    [f(c),f_n(c)]=\frac{1}{\sqrt{1+\left|f_n(c)\right|^2}}
\end{align*}
であり、任意の正数$r$に対する閉集合$|z-c|\le r$において、任意の正数$\epsilon$に対して、ある自然数$N(\epsilon)$が存在して、$n\ge N(\epsilon)$に定しては$[f(z),f_n(z)]<\epsilon$である。

特に、$[f(c),f_n(c)]<\epsilon$であるから、