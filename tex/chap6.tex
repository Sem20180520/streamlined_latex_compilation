\chapter{有理型函数}%第6章

\begin{mysimplebox}{問1}
    円環$0<|c_1|<|z|<|c_2|$における$\frac{1}{(z-c_1)(z-c_2)}$のLaurent展開を求めよ。
\end{mysimplebox}
\paragraph{解答}
部分分数展開によって、
\begin{align*}
    \frac{1}{(z-c_1)(z-c_2)}=\frac{1}{c_1-c_2}\left(\frac{1}{z-c_1}-\frac{1}{z-c_2}\right)
\end{align*}
となる。

ここで、円環において、次が成り立つ。
\begin{align*}
    0<\left|\frac{c_1}{z}\right|<1,\quad0<\left|\frac{z}{c_2}\right|<1
\end{align*}

よって、
\begin{align*}
    \frac{1}{z-c_1}
    &=\frac{1}{z}\cdot\frac{1}{1-\frac{c_1}{z}}
    =\frac{1}{z}\left\{1+\frac{c_1}{z}+\left(\frac{c_1}{z}\right)^2+\cdots\right\}\\
    \frac{1}{z-c_2}
    &=-\frac{1}{c_2}\cdot\frac{1}{1-\frac{z}{c_2}}
    =-\frac{1}{c_2}\left\{1+\frac{z}{c_2}+\left(\frac{z}{c_2}\right)^2+\cdots\right\}
\end{align*}

ゆえに、
\begin{align*}
    \frac{1}{(z-c_1)(z-c_2)}
    =\frac{1}{c_1-c_2}\left(\cdots+\frac{c_1^2}{z^3}+\frac{c_1}{z^2}+\frac{1}{z}+\frac{1}{c_2}+\frac{z}{c_2^2}+\frac{z^2}{c_2^3}+\cdots\right)
\end{align*}
(解答終)

\begin{mysimplebox}{問2}
    $f(z)$が整函数で$|f(z)|\le|\sin z|$ならば$f(z)\equiv A\sin z$($A$は定数)であることを示せ。
\end{mysimplebox}
\paragraph{証明}

\begin{align*}
    g(z)=\frac{f(z)}{\sin z}
\end{align*}
とする。$g(z)$は複素数平面上の$\sin z$の零点($z=n\pi$、ただし$n$は整数)を除いた領域で定義された函数である。その領域を$D$とする。
すなわち、
\begin{align*}
    D:=\C\setminus\{n\pi\mid n\in\Z\}
\end{align*}
である。

$|f(z)|\le|\sin z|$から、$D$において$|g(z)|\le1$である。

$c$は$\sin z$の零点のうちの一つとする。たとえば$c=\pi$とすればよい。以下の議論はそれ以外の零点でも同様に成り立つ。

実数$r$は$0<r<\pi$を満たすとする。$0<|z-c|<r$において、$g(z)$は正則であり、次のLaurent展開
\begin{align*}
    g(z)=\sum_{k=0}^{\infty}a_k(z-c)^k+\sum_{k=1}^{\infty}\frac{a_{-k}}{(z-c)^k}
\end{align*}
が可能である。

よって、
\begin{align*}
    |a_{-k}|
    &=\left|\frac{1}{2\pi i}\int_{|z-c|=r}g(z)(z-c)^{k-1}dz\right|\\
    &\le\frac{1}{2\pi}\int_{|z-c|=r}\left|g(z)(z-c)^{k-1}\right||dz|\\
    &\le\frac{1}{2\pi}r^{k-1}\cdot2\pi r
    =r^k
\end{align*}
が成り立つ。

$r\longrightarrow0$の極限をとることで、$a_{-k}=0$($k\ge1$)であることがわかる。

よって、Laurent展開の式から
\begin{align*}
    g(z)=\sum_{k=0}^{\infty}a_k(z-c)^k
\end{align*}
である。ゆえに、$z=c$は除去可能特異点であり、
\begin{align*}
    \lim_{z\rightarrow c}g(z)=a_0
\end{align*}
が成り立つ($D$において$|g(z)|\le1$であるから、$|a_0|\le1$である)。

よって、$g(c)=a_0$と定義する。他の零点でも同様にすれば、$g(z)$は全平面において正則である($z=\infty$においても$|g(z)|\le1$である)。

したがって、Liouvilleの定理から$g(z)$は定数であることがわかり、証明が完了する。(証明終)

\paragraph{補足1}
この結果はRiemannの特異点除去定理と呼ばれる有名なものである。

\paragraph{補足2}
$|f(z)|\le|\sin z|$から$\sin z$の零点はすべて$f(z)$の零点でもある。そうであるなら、
\begin{align*}
    f(z)=A_1\sin z+A_2\sin 2z+A_3\sin 3z+\dots+A_k\sin kz
\end{align*}
となることもあり得るのでは?と思ってしまう。

しかし、$z=-it$とし、$t$は十分大きい正の実数、$k>1$とすると、
\begin{align*}
    \left|\sin kz\right|
    &=\left|\sin(-ikt)\right|
    =\frac{e^{kt}-e^{-kt}}{2}
    >\frac{e^{t}-e^{-t}}{2}
    =|\sin(-it)|
\end{align*}
が成り立つ。

よって、$|f(z)|\le|\sin z|$が成り立たなくなる点が存在することになってしまう。
ゆえに、2倍音以上の正弦函数は現れないのである。

\paragraph{補足3}
\begin{align*}
    \lim_{z\longrightarrow n\pi}\frac{z-n\pi}{\sin z}
    =\lim_{z\longrightarrow n\pi}\frac{z-n\pi}{\sin(z-n\pi)}
    =1
\end{align*}
であるから、$n\pi$は$\frac{1}{\sin z}$の1位の極である。

$|f(z)|\le|\sin z|$より、$f(z)$は$n\pi$に零点をもつから、
$\frac{f(z)}{\sin z}$は$n\pi$において除去可能特異点をもつ。


\begin{mysimplebox}{問3}
    $f(z)$が$z=\infty$において$m$位の零点または極を有するとき、$f'(z)$は$z=\infty$において零点または極を有するか。
\end{mysimplebox}
\paragraph{解答}
$f(z)$が$z=\infty$で$m$位の零点を有するとき
\begin{align*}
    f(z)&=\cdots+\frac{c_{-(m+1)}}{z^{m+1}}+\frac{c_{-m}}{z^{m}}\\
    f(z)&=\cdots-(m+1)\frac{c_{-(m+1)}}{z^{m+2}}-m\frac{c_{-m}}{z^{m+1}}
\end{align*}
よって、任意の$m\ge1$に対して、$f'(z)$も$z=\infty$で零点を有する。

$f(z)$が$z=\infty$で$m$位の極を有するとき
\begin{align*}
    f(z)&=\cdots+c_{m-1}z^{m-1}+c_mz^m\\
    f(z)&=\cdots+(m-1)c_{m-1}z^{m-2}+mc_mz^{m-1}
\end{align*}
よって、$m\ge2$ならば$f'(z)$も$z=\infty$で極を有する。
$m=1$ならば$f'(z)$は$z=\infty$に極をもたない。
(解答終)

\begin{mysimplebox}{問4}
    単連結な領域$D$から有限個の点$c_1,c_2,\dots,c_k$を取り除いた残りの領域$D'$で、$f(z)$が正則で$\lim_{z\longrightarrow c_\nu}(z-c_\nu)f(z)=0$($\nu=1,2,\dots,k$)であるとき、$C$が$D'$内の長さのある閉曲線ならば、
    \begin{align*}
        \int_Cf(z)dz=0
    \end{align*}
    が成り立つことを示せ。
\end{mysimplebox}
\paragraph{証明}
$\nu=1,2,\dots,k$に対して、$c_\nu$を中心とする$D$内の$k$個の円$C_\nu$が存在する。さらに、どの2つの円も交わらないようにできる。

$D'$で$f(z)$は正則であるから
\begin{align*}
    \int_Cf(z)dz
    =\sum_{\nu=1}^{k}\int_{C_\nu}f(z)dz
\end{align*}
である。

円$C_\nu$の半径を$r_\nu$とすると、$0<|z-c_\nu|<r_\nu$であるとき、
\begin{align*}
    f(z)
    =\sum_{n=0}^{\infty}a_{\nu,n}(z-c_\nu)^n
    +\sum_{n=1}^{\infty}\frac{a_{\nu,-n}}{(z-c_\nu)^{n-1}}
\end{align*}
とLaurent展開できる。このとき、
\begin{align*}
    \int_{C_\nu}f(z)dz=2\pi i a_{\nu,-1}
\end{align*}
であるから、
\begin{align*}
    \int_Cf(z)dz
    =2\pi i \sum_{\nu=1}^{k}a_{\nu,-1}
\end{align*}
が成り立つ。

ここで、$\lim_{z\longrightarrow c_\nu}(z-c_\nu)f(z)=0$であるから、任意の$\epsilon>0$に対して、ある$\delta>0$が存在し、$0<|z-c_\nu|<\delta$ならば、
\begin{align*}
    |(z-c_\nu)f(z)|<\epsilon
\end{align*}
が成り立つ。

よって、
\begin{align*}
    |a_{\nu,-1}|
    &=\frac{1}{2\pi}\left| \int_{C_\nu}(z-c_\nu)f(z)dz\right|\\
    &\le \frac{1}{2\pi}\int_{C_\nu}|z-c_\nu|\left|f(z)\right||dz|\\
    &\le \frac{1}{2\pi}\epsilon\cdot2\pi r=\epsilon r\\
    &\longrightarrow 0\quad(r\longrightarrow0)
\end{align*}
であるから、$a_{\nu,-1}=0$である。

ゆえに、
\begin{align*}
    \int_Cf(z)dz=0
\end{align*}
が成り立つ。(証明終)

\paragraph{補足}
任意の$l\in\N$に対して、$\lim_{z\longrightarrow c_\nu}(z-c_\nu)^lf(z)=0$であるから、上記のLaurent展開において、$c_{\nu,-l}=0$である。よって、$f(z)$の主要部は0であるから、$c_\nu$は除去可能な特異点である。

\begin{mysimplebox}{問5}
    $-1<\Re(z)\le0$ならば$\Gamma(z)=\frac{\Gamma(z+1)}{z}$とおき、次に、$-2<\Re(z)\le-1$ならば$\Gamma(z)=\frac{\Gamma(z+1)}{z}$とおくというふうに、この手順を繰り返していくと$\Gamma(z)$を平面$|z|<+\infty$全体で定義することができる。こうして定義された$\Gamma(z)$は有理型函数であるが、その極はどのような点か。
\end{mysimplebox}
\paragraph{解答}
$\Re(z)>0$において$\Gamma(z)$は正則で、$\Gamma(z+1)=z\Gamma(z)$が成り立つことはp.113の問13で示した。

$-1<\Re(z)\le0$において$\Gamma(z)=\frac{\Gamma(z+1)}{z}$とおく。
\begin{align*}
    \Gamma(1)
    &=\int_{0}^{\infty}e^{-x}dx=\left[-e^{-x}\right]_0^\infty=1
\end{align*}
であるから、$z=0$は1位の極である。

同様に、$-2<\Re(z)\le-1$において$\Gamma(z)=\frac{\Gamma(z+1)}{z}$とおく。
\begin{align*}
    \lim_{z\longrightarrow-1}(z+1)\Gamma(z)
    &=\lim_{z\longrightarrow-1}\frac{(z+1)\Gamma(z+1)}{z}
    =-\lim_{z\longrightarrow0}z\Gamma(z)=-1
\end{align*}
であるから、$z=-1$も1位の極である。

$-3<\Re(z)\le-2$において$\Gamma(z)=\frac{\Gamma(z+1)}{z}$とおく。
\begin{align*}
    \lim_{z\longrightarrow-2}(z+2)\Gamma(z)
    &=\lim_{z\longrightarrow-2}\frac{(z+2)\Gamma(z+1)}{z}
    =\lim_{z\longrightarrow-2}\frac{(z+2)\Gamma(z+2)}{z(z+1)}=\frac{1}{-2(-1)}=\frac{1}{2}
\end{align*}
であるから、$z=-2$も1位の極である。

\begin{align*}
    (z+k)\Gamma(z)
    &=\frac{(z+k)\Gamma(z+1)}{z}
    =\frac{(z+k)\Gamma(z+2)}{z(z+1)}
    =\cdots=\frac{(z+k)\Gamma(z+k)}{z(z+1)\cdots(z+k-1)}\\
    &\longrightarrow \frac{(-1)^k}{k!}\quad(z\longrightarrow-k)
\end{align*}
であるから、$z=-k$($k\in\Z_{\ge0}$)は1位の極である。(解答終)