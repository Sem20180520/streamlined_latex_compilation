\chapter{有理型函数}%第6章

\begin{mysimplebox}{問1}
    円環$0<|c_1|<|z|<|c_2|$における$\frac{1}{(z-c_1)(z-c_2)}$のLaurent展開を求めよ。
\end{mysimplebox}
\paragraph{解答}
部分分数展開によって、
\begin{align*}
    \frac{1}{(z-c_1)(z-c_2)}=\frac{1}{c_1-c_2}\left(\frac{1}{z-c_1}-\frac{1}{z-c_2}\right)
\end{align*}
となる。

ここで、円環において、次が成り立つ。
\begin{align*}
    0<\left|\frac{c_1}{z}\right|<1,\quad0<\left|\frac{z}{c_2}\right|<1
\end{align*}

よって、
\begin{align*}
    \frac{1}{z-c_1}
    &=\frac{1}{z}\cdot\frac{1}{1-\frac{c_1}{z}}
    =\frac{1}{z}\left\{1+\frac{c_1}{z}+\left(\frac{c_1}{z}\right)^2+\cdots\right\}\\
    \frac{1}{z-c_2}
    &=-\frac{1}{c_2}\cdot\frac{1}{1-\frac{z}{c_2}}
    =-\frac{1}{c_2}\left\{1+\frac{z}{c_2}+\left(\frac{z}{c_2}\right)^2+\cdots\right\}
\end{align*}

ゆえに、
\begin{align*}
    \frac{1}{(z-c_1)(z-c_2)}
    =\frac{1}{c_1-c_2}\left(\cdots+\frac{c_1^2}{z^3}+\frac{c_1}{z^2}+\frac{1}{z}+\frac{1}{c_2}+\frac{z}{c_2^2}+\frac{z^2}{c_2^3}+\cdots\right)
\end{align*}
(解答終)

\begin{mysimplebox}{問2}
    $f(z)$が整函数で$|f(z)|\le|\sin z|$ならば$f(z)\equiv A\sin z$($A$は定数)であることを示せ。
\end{mysimplebox}
\paragraph{証明}

\begin{align*}
    g(z)=\frac{f(z)}{\sin z}
\end{align*}
とする。$g(z)$は複素数平面上の$\sin z$の零点($z=n\pi$、ただし$n$は整数)を除いた領域で定義された函数である。その領域を$D$とする。
すなわち、
\begin{align*}
    D:=\C\setminus\{n\pi\mid n\in\Z\}
\end{align*}
である。

$|f(z)|\le|\sin z|$から、$D$において$|g(z)|\le1$である。

$c$は$\sin z$の零点のうちの一つとする。たとえば$c=\pi$とすればよい。以下の議論はそれ以外の零点でも同様に成り立つ。

実数$r$は$0<r<\pi$を満たすとする。$0<|z-c|<r$において、$g(z)$は正則であり、次のLaurent展開
\begin{align*}
    g(z)=\sum_{k=0}^{\infty}a_k(z-c)^k+\sum_{k=1}^{\infty}\frac{a_{-k}}{(z-c)^k}
\end{align*}
が可能である。

よって、
\begin{align*}
    |a_{-k}|
    &=\left|\frac{1}{2\pi i}\int_{|z-c|=r}g(z)(z-c)^{k-1}dz\right|\\
    &\le\frac{1}{2\pi}\int_{|z-c|=r}\left|g(z)(z-c)^{k-1}\right||dz|\\
    &\le\frac{1}{2\pi}r^{k-1}\cdot2\pi r
    =r^k
\end{align*}
が成り立つ。

$r\longrightarrow0$の極限をとることで、$a_{-k}=0$($k\ge1$)であることがわかる。

よって、Laurent展開の式から
\begin{align*}
    g(z)=\sum_{k=0}^{\infty}a_k(z-c)^k
\end{align*}
である。ゆえに、$z=c$は除去可能特異点であり、
\begin{align*}
    \lim_{z\rightarrow c}g(z)=a_0
\end{align*}
が成り立つ($D$において$|g(z)|\le1$であるから、$|a_0|\le1$である)。

よって、$g(c)=a_0$と定義する。他の零点でも同様にすれば、$g(z)$は全平面において正則である($z=\infty$においても$|g(z)|\le1$である)。

したがって、Liouvilleの定理から$g(z)$は定数であることがわかり、証明が完了する。(証明終)

\paragraph{補足1}
この結果はRiemannの特異点除去定理と呼ばれる有名なものである。

\paragraph{補足2}
$|f(z)|\le|\sin z|$から$\sin z$の零点はすべて$f(z)$の零点でもある。そうであるなら、
\begin{align*}
    f(z)=A_1\sin z+A_2\sin 2z+A_3\sin 3z+\dots+A_k\sin kz
\end{align*}
となることもあり得るのでは?と思ってしまう。

しかし、$z=-it$とし、$t$は十分大きい正の実数、$k>1$とすると、
\begin{align*}
    \left|\sin kz\right|
    &=\left|\sin(-ikt)\right|
    =\frac{e^{kt}-e^{-kt}}{2}
    >\frac{e^{t}-e^{-t}}{2}
    =|\sin(-it)|
\end{align*}
が成り立つ。

よって、$|f(z)|\le|\sin z|$が成り立たなくなる点が存在することになってしまう。
ゆえに、2倍音以上の正弦函数は現れないのである。

\paragraph{補足3}
\begin{align*}
    \lim_{z\longrightarrow n\pi}\frac{z-n\pi}{\sin z}
    =\pm\lim_{z\longrightarrow n\pi}\frac{z-n\pi}{\sin(z-n\pi)}
    =\pm1
\end{align*}
であるから、$n\pi$は$\frac{1}{\sin z}$の1位の極である。

$|f(z)|\le|\sin z|$より、$f(z)$は$n\pi$に零点をもつから、
$\frac{f(z)}{\sin z}$は$n\pi$において除去可能特異点をもつ。


\begin{mysimplebox}{問3}
    $f(z)$が$z=\infty$において$m$位の零点または極を有するとき、$f'(z)$は$z=\infty$において零点または極を有するか。
\end{mysimplebox}
\paragraph{解答}
$f(z)$が$z=\infty$で$m$位の零点を有するとき
\begin{align*}
    f(z)&=\cdots+\frac{c_{-(m+1)}}{z^{m+1}}+\frac{c_{-m}}{z^{m}}\\
    f(z)&=\cdots-(m+1)\frac{c_{-(m+1)}}{z^{m+2}}-m\frac{c_{-m}}{z^{m+1}}
\end{align*}
よって、任意の$m\ge1$に対して、$f'(z)$も$z=\infty$で零点を有する。

$f(z)$が$z=\infty$で$m$位の極を有するとき
\begin{align*}
    f(z)&=\cdots+c_{m-1}z^{m-1}+c_mz^m\\
    f(z)&=\cdots+(m-1)c_{m-1}z^{m-2}+mc_mz^{m-1}
\end{align*}
よって、$m\ge2$ならば$f'(z)$も$z=\infty$で極を有する。
$m=1$ならば$f'(z)$は$z=\infty$に極をもたない。
(解答終)

\begin{mysimplebox}{問4}
    単連結な領域$D$から有限個の点$c_1,c_2,\dots,c_k$を取り除いた残りの領域$D'$で、$f(z)$が正則で$\lim_{z\longrightarrow c_\nu}(z-c_\nu)f(z)=0$($\nu=1,2,\dots,k$)であるとき、$C$が$D'$内の長さのある閉曲線ならば、
    \begin{align*}
        \int_Cf(z)dz=0
    \end{align*}
    が成り立つことを示せ。
\end{mysimplebox}
\paragraph{証明}
$\nu=1,2,\dots,k$に対して、$c_\nu$を中心とする$D$内の$k$個の円$C_\nu$が存在する。さらに、どの2つの円も交わらないようにできる。閉曲線$C$で囲まれた内部に点$c_1,c_2,\dots,c_k$があるとしてよい(点$c_1,c_2,\dots,c_k$のうち、閉曲線$C$の内部にない点がある場合、そういった点を無視しても以下の議論では差し支えない)。

$D'$で$f(z)$は正則であるから
\begin{align*}
    \int_Cf(z)dz
    =\sum_{\nu=1}^{k}\int_{C_\nu}f(z)dz
\end{align*}
である。

円$C_\nu$の半径を$r_\nu$とすると、$0<|z-c_\nu|<r_\nu$であるとき、
\begin{align*}
    f(z)
    =\sum_{n=0}^{\infty}a_{\nu,n}(z-c_\nu)^n
    +\sum_{n=1}^{\infty}\frac{a_{\nu,-n}}{(z-c_\nu)^n}
\end{align*}
とLaurent展開できる。このとき、
\begin{align*}
    \int_{C_\nu}f(z)dz=2\pi i a_{\nu,-1}
\end{align*}
であるから、
\begin{align*}
    \int_Cf(z)dz
    =2\pi i \sum_{\nu=1}^{k}a_{\nu,-1}
\end{align*}
が成り立つ。

ここで、$\lim_{z\longrightarrow c_\nu}(z-c_\nu)f(z)=0$であるから、任意の$\epsilon>0$に対して、ある$\delta>0$が存在し、$0<|z-c_\nu|<\delta$ならば、
\begin{align*}
    |(z-c_\nu)f(z)|<\epsilon
\end{align*}
が成り立つ。

よって、
\begin{align*}
    |a_{\nu,-2}|
    &=\frac{1}{2\pi}\left| \int_{C_\nu}(z-c_\nu)f(z)dz\right|
    \le \frac{1}{2\pi}\int_{C_\nu}|z-c_\nu|\left|f(z)\right||dz|\\
    &\le \frac{1}{2\pi}\epsilon\cdot2\pi \delta=\epsilon \delta
\end{align*}
であり、$\epsilon \delta$はいくらでも小さくできるから、$a_{\nu,-2}=0$である。

同様に、$k\ge2$に対して、
\begin{align*}
    |a_{\nu,-k}|
    &=\frac{1}{2\pi}\left| \int_{C_\nu}(z-c_\nu)^{k-1}f(z)dz\right|
    \le \frac{1}{2\pi}\int_{C_\nu}|z-c_\nu|^{k-1}\left|f(z)\right||dz|\\
    &\le \frac{1}{2\pi}\epsilon^{k-1}\cdot2\pi \delta=\epsilon^{k-1} \delta
\end{align*}
であり、$\epsilon^{k-1} \delta$はいくらでも小さくできるから、$a_{\nu,-k}=0$である。

よって、$0<|z-c_\nu|<\delta$において、
\begin{align*}
    f(z)
    =\sum_{n=0}^{\infty}a_{\nu,n}(z-c_\nu)^n
    +\frac{a_{\nu,-1}}{(z-c_\nu)}
\end{align*}
である。さらに、
\begin{align*}
    \lim_{z\longrightarrow c_\nu}(z-c_\nu)f(z)
    &=\lim_{z\longrightarrow c_\nu}\left\{\sum_{n=0}^{\infty}a_{\nu,n}(z-c_\nu)^{n+1}
    +a_{\nu,-1}\right\}\\
    &=a_{\nu,-1}=0
\end{align*}


ゆえに、
\begin{align*}
    \int_Cf(z)dz=0
\end{align*}
が成り立つ。(証明終)

\paragraph{別解}
任意の$l\in\N$に対して、$\lim_{z\longrightarrow c_\nu}(z-c_\nu)f(z)=0$であるから、書籍p.117のRiemannの定理より、$c_\nu$は$(z-c_\nu)f(z)$の除去可能な特異点である。よって、$c_\nu$のある近傍において、
\begin{align*}
    (z-c_\nu)f(z)
    =\sum_{k=0}^{\infty}b_{\nu,k}(z-c_\nu)^k
\end{align*}
と表される。

再び$\lim_{z\longrightarrow c_\nu}(z-c_\nu)f(z)=0$から、$b_{\nu,0}=0$である。ゆえに、
\begin{align*}
    f(z)
    =\sum_{k=0}^{\infty}b_{\nu,k+1}(z-c_\nu)^k
\end{align*}
である。以上から、$\int_Cf(z)dz=0$が言える。(証明終)

\begin{mysimplebox}{問5}
    $-1<\Re(z)\le0$ならば$\Gamma(z)=\frac{\Gamma(z+1)}{z}$とおき、次に、$-2<\Re(z)\le-1$ならば$\Gamma(z)=\frac{\Gamma(z+1)}{z}$とおくというふうに、この手順を繰り返していくと$\Gamma(z)$を平面$|z|<+\infty$全体で定義することができる。こうして定義された$\Gamma(z)$は有理型函数であるが、その極はどのような点か。
\end{mysimplebox}
\paragraph{解答}
$\Re(z)>0$において$\Gamma(z)$は正則で、$\Gamma(z+1)=z\Gamma(z)$が成り立つことはp.113の問13で示した。

$-1<\Re(z)\le0$において$\Gamma(z)=\frac{\Gamma(z+1)}{z}$とおく。
\begin{align*}
    \Gamma(1)
    &=\int_{0}^{\infty}e^{-x}dx=\left[-e^{-x}\right]_0^\infty=1
\end{align*}
であるから、$z=0$は1位の極である。

同様に、$-2<\Re(z)\le-1$において$\Gamma(z)=\frac{\Gamma(z+1)}{z}$とおく。
\begin{align*}
    \lim_{z\longrightarrow-1}(z+1)\Gamma(z)
    &=\lim_{z\longrightarrow-1}\frac{(z+1)\Gamma(z+1)}{z}
    =-\lim_{z\longrightarrow0}z\Gamma(z)=-1
\end{align*}
であるから、$z=-1$も1位の極である。

$-3<\Re(z)\le-2$において$\Gamma(z)=\frac{\Gamma(z+1)}{z}$とおく。
\begin{align*}
    \lim_{z\longrightarrow-2}(z+2)\Gamma(z)
    &=\lim_{z\longrightarrow-2}\frac{(z+2)\Gamma(z+1)}{z}
    =\lim_{z\longrightarrow-2}\frac{(z+2)\Gamma(z+2)}{z(z+1)}=\frac{1}{-2(-1)}=\frac{1}{2}
\end{align*}
であるから、$z=-2$も1位の極である。

\begin{align*}
    (z+k)\Gamma(z)
    &=\frac{(z+k)\Gamma(z+1)}{z}
    =\frac{(z+k)\Gamma(z+2)}{z(z+1)}
    =\cdots=\frac{(z+k)\Gamma(z+k)}{z(z+1)\cdots(z+k-1)}\\
    &\longrightarrow \frac{(-1)^k}{k!}\quad(z\longrightarrow-k)
\end{align*}
であるから、$z=-k$($k\in\Z_{\ge0}$)は1位の極である。(解答終)

\begin{mysimplebox}{問6}
    立体射影は等角な写像である。
    すなわち、球面$S$上の二つの円の交角はその立体射影である平面上の二つの円(直線)の交角に等しいことを示せ。
\end{mysimplebox}
\paragraph{証明(この証明は間違っている)}
立体射影を$\pi$と書くことにする。
\begin{align*}
    \pi\colon S &\to \C\\
    (\xi,\eta,\zeta) &\mapsto \left(\frac{\xi}{1-\zeta},\frac{\eta}{1-\zeta}\right)\quad((\xi,\eta,\zeta)\neq(0,0,1))\\
    (0,0,1)&\mapsto\infty
\end{align*}
$S$の二つの円がともにN$(0,0,1)$とS$(0,0,0)$を通るならば、$\pi$による像は$\C$上の直線であり、$\pi$によって等角に移されることは明らかである。

p.130の記述をヒントにする。

$\C$上の円または直線
\begin{align}
    a(x^2+y^2)+2bx+2cy+d=0 \label{eq:6en}
\end{align}
には、空間における平面
\begin{align}
    a\zeta+2b\xi+2c\eta+d(1-\zeta)=0 \label{eq:6S}
\end{align}
と$S$の交線である円が対応する。

$(\ref{eq:6S})$を変形すると
\begin{align*}
    2b\xi+2c\eta+(a-d)\zeta+d=0
\end{align*}
とできるから、平面$(\ref{eq:6S})$の法線ベクトルは
\begin{align*}
    \bm{n}=\begin{bmatrix}
        2b \\ 2c \\ a-d
    \end{bmatrix}
\end{align*}
である。

$k=1,2$に対し、
\begin{align*}
    \bm{n}_k=\begin{bmatrix}
        2b_k \\ 2c_k \\ a_k-d_k
    \end{bmatrix}
\end{align*}
とする。
\begin{align*}
    \cos\theta
    =\frac{\bm{n}_1\cdot\bm{n}_2}{|\bm{n}_1||\bm{n}_2|}
    =\frac{4b_1b_2+4c_1c_2+(a_1-d_1)(a_2-d_2)}{\prod_{k=1,2}\sqrt{4b_k^2+4c_k^2+(a_k-d_k)^2}}
\end{align*}
$\theta$は二つの法線$\bm{n}_1,\bm{n}_2$の角度である。この角を$S$上の二つの円の交角とする\footnote{法線のなす角を二つの円の交角とするのは誤りであり、この証明は正しくない。}。ただし、二つの円が交わると仮定している。

$a\neq0$とすると、$(\ref{eq:6en})$から
\begin{align*}
    \left(x+\frac{b}{a}\right)^2
    +\left(y+\frac{c}{a}\right)^2
    =\frac{b^2+c^2-ad}{a^2}
\end{align*}
このような$\C$上の二つの円が交わるとし、その交点におけるそれぞれの接線のなす角を求める?

(保留)←この証明は間違っている。次の証明を参照。
\newpage
\paragraph{証明}
球面上の交わる円弧の一部を$AB$、$CD$とする。その交点を$P$とする。
さらに、$P,A,B,C,D$の立体射影による像を$P',A',B',C',D'$とする。
%\begin{align*}
%    \pi(P)&=P'\\
%    \pi(A)&=A'\\
%    \pi(B)&=B'\\
%    \pi(C)&=C'\\
%    \pi(D)&=D'
%\end{align*}
%とおく。

円弧$AB$の点$P$における接線と$\zeta=1$平面との交点を$Q$とする。同様に、円弧$CD$の点$P$における接線と$\zeta=1$平面との交点を$R$とする。

線分$PQ$と$NQ$はどちらも球面の接線であるから、$|PQ|=|NQ|$である。同様に、$|PR|=|NR|$である。

よって、$\triangle PQR\equiv\triangle NQR$であるから、$\angle QPR=\angle QNR$である。

また、
$\overrightarrow{QQ'}
=\overrightarrow{RR'}
=\overrightarrow{NP'}$とする。
$Q',R'$は$\zeta=0$平面上の点である。

$P,N,Q,P',Q'$は同一平面上にある。
同様に、$P,N,R,P',R'$は同一平面上にある。

$\angle Q'P'R'=\angle QNR$であるから、
$\angle Q'P'R'=\angle QPR$である。

直線$P'Q'$は円弧$A'B'$の$P'$における接線である。
同様に、直線$P'R'$は円弧$C'D'$の$P'$における接線である。

以上で、立体射影が等角写像であることが示された。
(証明終)

\begin{figure}[h]
    \centering
    \includegraphics[width=10cm]{chap6_fig/prob6.png}
    \caption{立体射影は等角写像}
    \label{fig:prob6}
\end{figure}

\newpage
\begin{mysimplebox}{問7}
    (44.1)、(44.2)を示せ。
\end{mysimplebox}
\paragraph{証明}
球面$S$の半径を$r$とおく。$r=\frac{1}{2}$である。

$S$の中心の座標は$(0,0,r)$である。

P$(\xi,\eta,\zeta)$は$S$上の点であるから
\begin{align*}
    \xi^2+\eta^2+(\zeta-r)^2=r^2
\end{align*}
が成り立つ。

\begin{figure}[h]
    \centering
    \includegraphics[width=5cm]{chap6_fig/ONP.jpg}
    \caption{ONP平面}
    \label{fig:ONP}
\end{figure}

図$\ref{fig:ONP}$から、次の比の計算ができる。
\begin{align*}
    (1-\zeta):1&=\sqrt{\xi^2+\eta^2}:\sqrt{x^2+y^2}\\
    \sqrt{x^2+y^2}&=\frac{\sqrt{\xi^2+\eta^2}}{1-\zeta}
    =\frac{\sqrt{r^2-(\zeta-r)^2}}{1-\zeta}
    =\frac{\sqrt{\zeta(1-\zeta)}}{1-\zeta}\\
    x^2+y^2&=\frac{\zeta}{1-\zeta}
\end{align*}

\begin{figure}[h]
    \centering
    \includegraphics[width=5cm]{chap6_fig/xON.jpg}
    \caption{xON平面}
    \label{fig:xON}
\end{figure}

$x$ON平面への射影である図$\ref{fig:xON}$から、次の比の計算ができる。
\begin{align*}
    (1-\zeta):1&=\xi:x\\
    \xi&=x(1-\zeta)\\
    x&=\frac{\xi}{1-\zeta}
\end{align*}

同様にして、
\begin{align*}
    (1-\zeta):1&=\eta:y\\
    \eta&=y(1-\zeta)\\
    y&=\frac{\eta}{1-\zeta}
\end{align*}

上で得られた式から、$\zeta,\xi,\eta$について次のように書ける。
\begin{align*}
    (x^2+y^2)(1-\zeta)&=\zeta\\
    x^2+y^2-(x^2+y^2)\zeta&=\zeta\\
    x^2+y^2&=(1+x^2+y^2)\zeta\\
    \zeta&=\frac{x^2+y^2}{1+x^2+y^2}\\
    1-\zeta&=\frac{1}{1+x^2+y^2}\\
    \xi&=x(1-\zeta)=\frac{x}{1+x^2+y^2}\\
    \eta&=y(1-\zeta)=\frac{y}{1+x^2+y^2}
\end{align*}
(証明終)
\newpage
\begin{mysimplebox}{問8}
    平面上の2点$z,z'$に対応する球面$S$上の2点の距離を$[z,z']$で表すと、$[z,z']$は次の式で表される:
    \begin{align*}
        [z,z']&=\frac{|z-z'|}{\sqrt{(1+|z|^2)(1+|z'|^2)}}
        \quad(z\neq\infty,z'\neq\infty)\\
        [z,z']&=\frac{1}{\sqrt{1+|z|^2}}
        \quad(z\neq\infty,z'=\infty)
    \end{align*}
\end{mysimplebox}
\paragraph{証明}

\begin{figure}[h]
    \centering
    \includegraphics[width=5cm]{chap6_fig/Sdis.jpg}
    \caption{$[z,z']$}
    \label{fig:Sdis}
\end{figure}

図$\ref{fig:Sdis}$のように、$\theta=\angle zNz'$とする。
\begin{align*}
    zN&=\sqrt{1+|z|^2}\\
    z'N&=\sqrt{1+|z'|^2}\\
    1:(1-\zeta)=\sqrt{1+x^2+y^2}:NP\\
    NP&=(1-\zeta)\sqrt{1+x^2+y^2}=\frac{1}{\sqrt{1+x^2+y^2}}=\frac{1}{\sqrt{1+|z|^2}}\\
    NP'&=\frac{1}{\sqrt{1+|z'|^2}}
\end{align*}
$\triangle zNz'$において、余弦定理から、
\begin{align*}
    \cos\theta=\frac{|z-z'|^2-(1+|z|^2)-(1+|z'|^2)}{2\sqrt{1+|z|^2}\sqrt{1+|z'|^2}}
\end{align*}

$\triangle PNP'$において、余弦定理から、
\begin{align*}
    PP'^2
    &=NP^2+NP'^2+2NP\cdot NP'\cos\theta\\
    &=\frac{1}{1+|z|^2}+\frac{1}{1+|z'|^2}+\frac{|z-z'|^2-(1+|z|^2)-(1+|z'|^2)}{(1+|z|^2)(1+|z'|^2)}\\
    &=\frac{|z-z'|^2}{(1+|z|^2)(1+|z'|^2)}
\end{align*}

よって、
\begin{align*}
    [z,z']&=PP'=\frac{|z-z'|}{\sqrt{(1+|z|^2)(1+|z'|^2)}}
    \quad(z\neq\infty,z'\neq\infty)
\end{align*}
である。

$z'=\infty$ならば、
\begin{align*}
    [z,z']&=NP=\frac{1}{\sqrt{1+|z|^2}}
\end{align*}
である。
(証明終)

\paragraph{別証明}
(44.1),(44.2)式を利用して示せる。
\begin{align*}
    z&\leftrightarrow(\xi,\eta,\zeta)
    =\left(\frac{1}{2}\frac{z+\overline{z}}{1+|z|^2},\frac{1}{2i}\frac{z-\overline{z}}{1+|z|^2},\frac{1}{2}\frac{z+\overline{z}}{1+|z|^2}\right)\\
    z'&\leftrightarrow(\xi',\eta',\zeta')
    =\left(\frac{1}{2}\frac{z'+\overline{z'}}{1+|z'|^2},\frac{1}{2i}\frac{z'-\overline{z'}}{1+|z'|^2},\frac{1}{2}\frac{z'+\overline{z'}}{1+|z|^2}\right)
\end{align*}
より、
\begin{align*}
    [z,z']=\sqrt{(\xi-\xi')^2+(\eta-\eta')^2+(\zeta-\zeta')^2}
\end{align*}
を素直に計算すればよい。具体的な計算は入力が大変であるため、省略する。(別証明終)


\newpage
\begin{mysimplebox}{問9}
    $\epsilon$がどの正数でも$m\ge N(\epsilon),n\ge N(\epsilon)$ならば$[z_m,z_n]<\epsilon$であるように自然数$N(\epsilon)$を定めうるとき、数列$\{z_n\}$は球面収束するという。$\{z_n\}$が球面収束するためには$\{z_n\}$が極限値$\alpha$($|\alpha|\le+\infty$)をもつことが必要十分であることを示せ。
\end{mysimplebox}
\paragraph{証明}
(必要性)

$\{|z_n|\mid n\in\N\}$が有界であるとき、$\sup_{n\in\N}|z_n|=M<\infty$とする。

$m,n\ge N(\epsilon)$ならば、
\begin{align*}
    [z_m,z_n]
    &=\frac{|z_m-z_n|}{\sqrt{(1+|z_m|^2)(1+|z_n|^2)}}
    <\epsilon
\end{align*}

よって、
\begin{align*}
    |z_m-z_n|<\epsilon\sqrt{(1+|z_m|^2)(1+|z_n|^2)}
    \le\epsilon\sqrt{(1+M^2)(1+M^2)}=\epsilon(1+M^2)
\end{align*}
となるが、$\epsilon$が任意の正数であるから、この最右辺はいくらでも小さくできる。よって、$\{z_n\}$は収束して極限値をもつ。

$\{|z_n|\mid n\in\N\}$が有界でないとき、部分列$\{z_{l_n}\}$で$z_{l_n}\longrightarrow\infty$であるものが存在する。

$[\cdot,\cdot]$は三角不等式を満たすため(問10で示す)、次が成り立つ。
\begin{align*}
    [z_n,\infty]
    =\frac{1}{\sqrt{(1+|z_n|^2)}}
    \le[z_n,z_{l_m}]+[z_{l_m},\infty]
\end{align*}
この最右辺は、十分大きな$n,m$をとればいくらでも小さくできるから、$z_n\longrightarrow\infty$が言える。

(十分性)

$z_n\longrightarrow\alpha$($|\alpha|<\infty$)のとき

\begin{align*}
    0\le[z_n,z_m]
    =\frac{|z_m-z_n|}{\sqrt{(1+|z_m|^2)(1+|z_n|^2)}}
    \le|z_m-z_n|\longrightarrow0\quad(m,n\longrightarrow\infty)
\end{align*}

$z_n\longrightarrow\infty$のときは

\begin{align*}
    [z_m,z_n]
    &=\frac{|z_m-z_n|}{\sqrt{(1+|z_m|^2)(1+|z_n|^2)}}
    \le\frac{|z_m|+|z_n|}{\sqrt{(1+|z_m|^2)(1+|z_n|^2)}}\\
    &\le\frac{\sqrt{1+|z_m|^2}+\sqrt{1+|z_n|^2}}{\sqrt{(1+|z_m|^2)(1+|z_n|^2)}}\\
    &=\frac{1}{\sqrt{1+|z_m|^2}}+\frac{1}{\sqrt{1+|z_n|^2}}
    =[z_m,\infty]+[z_n,\infty]\longrightarrow0
    \quad(m,n\longrightarrow\infty)
\end{align*}
(証明終)
\newpage
\paragraph{別証明}
$\{|z_n|\mid n\in\N\}$が有界であるときに関しては、次のようにも示せる。$\sup_{n\in\N}|z_n|=M<\infty$とする。

\begin{align*}
    [z_m,z_n]
    &=\frac{|z_m-z_n|}{\sqrt{(1+|z_m|^2)(1+|z_n|^2)}}
    \ge\frac{|z_m-z_n|}{1+M^2}
\end{align*}
であるから、
\begin{align*}
    \frac{|z_m-z_n|}{1+M^2}
    \le[z_m,z_n]
    \le|z_m-z_n|
\end{align*}
が成り立つ。よって、$[z_m,z_n]$に関してコーシー条件を満たすことと、$|z_m-z_n|$に関してコーシー条件を満たすことは同値である。したがって、$\{|z_n|\mid n\in\N\}$が有界であるときは、球面収束するためには極限値をもつことが必要十分であることが言えた。
(証明終)

\newpage
\begin{mysimplebox}{問10}
    点集合$E$で定義された函数$f_n(z)\ (n=1,2,\dots)$は$\infty$なる値をとり得るものとし、$E$の各点で数列$\{f_n(z)\}$が球面収束するときは、函数列$\{f_n(z)\}$は$E$で球面収束するという。

    また、$f(z)=\lim_{n\to\infty}f_n(z)$とおくと、$z$が$E$のどの点でも、また$\epsilon$がどの正数でも、$n\ge N(\epsilon)$に対しては$[f_n(z),f(z)]<\epsilon$であるように$z$に無関係な自然数$N(\epsilon)$を選び得るとき、函数列$\{f_n(z)\}$は$E$で``球面一様収束''するという。

    $f_n(z)\ (n=1,2,\dots)$が領域$D$で有理型で、$D$内のどの閉集合でも函数列$\{f_n(z)\}$が球面一様収束するとき($D$で``広義の球面一様収束''するとき)、極限函数$f(z)$は$D$で至る所$\infty$に等しいか、または$D$で有理型函数である。これを示せ。
\end{mysimplebox}
\paragraph{証明}
最初に、$[\cdot,\cdot]$は三角不等式を満たすことを説明しておく。$k=1,2,3$に対して、立体射影における球面$S$上の点$(\xi_k,\eta_k,\zeta_k)$と$\C$平面上の点$z_k$が対応するとき、
\begin{align*}
    [z_1,z_3]
    =&\frac{|z_1-z_3|}{\sqrt{(1+|z_1|^2)(1+|z_3|^2)}}\\
    =&\sqrt{(\xi_1-\xi_3)^2+(\eta_1-\eta_3)^2+(\zeta_1-\zeta_3)^2}\\
    \le&\sqrt{(\xi_1-\xi_2)^2+(\eta_1-\eta_2)^2+(\zeta_1-\zeta_2)^2}\\
    &+\sqrt{(\xi_2-\xi_3)^2+(\eta_2-\eta_3)^2+(\zeta_2-\zeta_3)^2}\\
    =&\frac{|z_1-z_2|}{\sqrt{(1+|z_1|^2)(1+|z_2|^2)}}+\frac{|z_2-z_3|}{\sqrt{(1+|z_2|^2)(1+|z_3|^2)}}\\
    =&[z_1,z_2]+[z_2,z_3]
\end{align*}
が成り立つ。

さて、領域$D$の2点$z_1,z_2$において、
\begin{align*}
    [f(z_1),f(z_2)]
    \le[f(z_1),f_n(z_1)]+[f_n(z_1),f_n(z_2)]+[f_n(z_2),f(z_2)]
\end{align*}
が成り立つ。$z_2$は閉円板$\{z\in\C\mid|z-z_1|\le r\}$に属するとし、この閉円板上で$\{f_n(z)\}$は球面一様収束するから、任意の正数$\epsilon$に対して、ある自然数$N$が存在し、$n>N$に対して、
\begin{align*}
    [f(z_1),f_n(z_1)]<\epsilon,\quad
    [f(z_2),f_n(z_2)]<\epsilon
\end{align*}
が成り立つ。

%また、$\{f_n(z)\}$は有理型函数列であるから閉円板上で$\infty$の値もとり得るが、$|z_1-z_2|$を十分小さくすれば$[f_n(z_1),f_n(z_2)]<\epsilon$を成り立たせることができる。
また、$f_n(z_1),f_n(z_2)$ともに$\infty$ではないとき、
\begin{align*}
    [f_n(z_1),f_n(z_2)]
    &=\frac{|f_n(z_1)-f_n(z_2)|}{\sqrt{(1+|f_n(z_1)|^2)(1+|f_n(z_2)|^2)}}
    \le|f_n(z_1)-f_n(z_2)|
\end{align*}
であるから$z_1$と$z_2$が十分近ければ、$[f_n(z_1),f_n(z_2)]$はいくらでも小さくできる。

さらに、また、$f_n(z_1)\neq\infty$、$f_n(z_2)=\infty$のとき、$z_2$は$f_n(z)$の極であるから、
\begin{align*}
    [f_n(z_1),f_n(z_2)]
    &=\frac{1}{\sqrt{(1+|f_n(z_1)|^2)}}
\end{align*}
についても$z_1$を$z_2$に十分近づければ、$[f_n(z_1),f_n(z_2)]$はいくらでも小さくできる。

よって、$f(z)$は$[\cdot,\cdot]$に関して連続である。

次に、$f(z)\not\equiv\infty$とする。$z_0\in D$において$f(z_0)\neq\infty$ならば、ある閉円板$A=\{z\in\C\mid|z-z_0|\le r\}$において、$|f(z)|\le L$(有界)としてよい。

$\{f_n(z)\}$は$A$において、球面一様収束する。
よって、任意の正数$\epsilon$、$z\in A$、十分大きい$n$に対して、
\begin{align*}
    [f_n(z),f(z)]
    =&\frac{|f_n(z)-f(z)|}{\sqrt{\left(1+\left|f_n(z)\right|^2\right)\left(1+\left|f(z)\right|^2\right)}}
    <\epsilon\\
    |f_n(z)-f(z)|
    <&\epsilon\sqrt{\left(1+\left|f_n(z)\right|^2\right)\left(1+\left|f(z)\right|^2\right)}\\
    <&\epsilon\sqrt{\left(1+2\left|f_n(z)\right|+\left|f_n(z)\right|^2\right)\left(1+2L+L^2\right)}\\
    =&\epsilon\left(1+\left|f_n(z)\right|\right)\left(1+L\right)\\
    |f_n(z)|-|f(z)|
    <&\epsilon\left(1+\left|f_n(z)\right|\right)\left(1+L\right)\\
    \{1-\epsilon(1+L)\}|f_n(z)|
    <&\epsilon(1+L)+|f(z)|<\epsilon(1+L)+L\\
    |f_n(z)|<&\frac{\epsilon(1+L)+L}{1-\epsilon(1+L)}
\end{align*}
が成り立つ。最後の最右辺は$\epsilon<\frac{1}{2(1+L)}$とすれば$|f_n(z)|<1+2L$とできる。

すなわち、十分大きい$n$に対して$|f_n(z)|$も$A$において有界であるから、極を持たない。ゆえに、$f_n(z)$は$A$の内部において正則であり、$A$において連続である。

以上より、$z\in A$、十分大きい$n$に対して、
\begin{align*}
    |f_n(z)-f(z)|
    <\epsilon(2+L)(1+L)
\end{align*}
が成り立つから、$A$において$\{f_n(z)\}$は通常の意味で$f(z)$に一様収束する。

ゆえに、$f(z)$は$A$の内部で正則である。

また、$f(z_0)=\infty$のとき、$\frac{1}{f(z_0)}=0$である。

\begin{align*}
    \left[\frac{1}{f_n(z)},\frac{1}{f(z)}\right]
    =&\frac{\left|\frac{1}{f_n(z)}-\frac{1}{f(z)}\right|}{\sqrt{1+\left|\frac{1}{f_n(z)}\right|^2}\sqrt{1+\left|\frac{1}{f(z)}\right|^2}}\\
    =&\frac{\left|f_n(z)-f(z)\right|}{\sqrt{1+\left|f_n(z)\right|^2}\sqrt{1+\left|f(z)\right|^2}}\\
    =&[f_n(z),f(z)]
\end{align*}
であるから、有理型函数列$\left\{\frac{1}{f_n(z)}\right\}$は$\frac{1}{f(z)}$に広義球面一様収束する。先ほどの議論と同様にして$\frac{1}{f(z)}$は正則であることが言えるから、$f(z)$は有理型である。(証明終)



%
%以下は最初に書こうとした証明だが、方針があまりよくなかったため、手放した。
%
%$c\in D$において$f(c)=\infty$とする。
%$D$内で$f(z)\neq\infty$となる点$z$が存在するならば、$f(z)$は有理型であることを示す。

%仮定により、$D$内の$c$を含む任意の有界閉集合$\Delta$において、有理型函数列$\{f_n(z)\}$は$f(z)$に球面一様収束する。
%
%次の2つの場合があり得る。
%\begin{enumerate}
%    \item 十分大きい$n$に対して、$f_n(z)$は$c$で極をもたないが、$\left|f_n(c)\right|\longrightarrow\infty\ (n\longrightarrow\infty)$\label{enum:1}
%    \item 十分大きい$n$に対して、$f_n(z)$が$c$で極をもつ。\label{enum:2}
%\end{enumerate}
%
%$\ref{enum:1}$の場合、$c$の近傍で$f_n(z)$は正則であるから、
%\begin{align*}
%    f_n(z)=a_0+a_1(z-c)+a_2(z-c)^2+\cdots
%\end{align*}
%と表される。
%
%任意の$R>0$に対して、十分大きな$n$をとれば、$z\in\Delta$において、$\left|f_n(z)\right|>R$とできる。よって、$\left|f_n(z)\right|\longrightarrow\infty\ (n\longrightarrow\infty)$であるから、$f(z)=\infty$である。ゆえに、$\Delta$において$f(z)=\infty$である。
%
%結局、$D$において$f(z)=\infty$である。
%
%次に、$\ref{enum:2}$の場合、ある正数$r$に対して$0<|z-c|<r$で$f_n(z)$は正則であるとしてよい。
%
%この円環領域において、$f_n(z)\longrightarrow f(z)$が広義一様収束であることを示せば、$f(z)$が$0<|z-c|<r$において正則であることが言えて、$f(z)$が有理型であると分かる。
%
%$f_n(z)\longrightarrow f(z)$が球面広義一様収束であることを用いて、広義一様収束であることを示す。
%(保留)
%

%\begin{align*}
%    [f(c),f_n(c)]=\frac{1}{\sqrt{1+\left|f_n(c)\right|^2}}
%\end{align*}
%であり、任意の正数$r$に対する閉集合$|z-c|\le r$において、任意の正数$\epsilon$に対して、ある自然数$N(\epsilon)$が存在して、$n\ge N(\epsilon)$に定しては$[f(z),f_n(z)]<\epsilon$である。
%
%特に、$[f(c),f_n(c)]<\epsilon$であるから、

\newpage
\begin{mysimplebox}{問11}
   前題において、特に、$f_n(z)\ (n=1,2,\cdots)$が$D$で正則な場合には、函数列$\{f_n(z)\}$が$D$で広義の一様収束するか、または函数列$\left\{\frac{1}{f_n(z)}\right\}$が$D$で定数0に広義の一様収束するか、そのいずれかである。
\end{mysimplebox}
\paragraph{証明}
$\{f_n(z)\}$が正則函数列である場合も、前題と同様にして、極限関数$\{f(z)\}$が存在して、$[\cdot,\cdot]$に関して連続であることがわかる。

$f(z)\not\equiv\infty$であるとき、前題と同様にして、$D$内の任意の閉集合$A$において、十分大きい$n$に対して$|f_n(z)|$は有界であり、$f_n(z)$は通常の意味で$f(z)$に広義一様収束する。

$f(z)\equiv\infty$であるとき、任意の閉集合$A$において球面一様収束するから、任意の正数$\epsilon$に対して、十分大きい$n$に対して、以下のようにできる。
\begin{align*}
    [f_n(z),\infty]
    =\frac{1}{\sqrt{1+|f_n(z)|^2}}
    &<\epsilon\\
    \frac{1}{1+|f_n(z)|^2}
    &<\epsilon^2\\
    1&<\epsilon^2+\epsilon^2|f_n(z)|^2\\
    \frac{1-\epsilon^2}{\epsilon^2}&<|f_n(z)|^2\\
    \frac{1}{|f_n(z)|^2}&<\frac{\epsilon^2}{1-\epsilon^2}
\end{align*}
最下行の右辺はいくらでも小さくできるから、$\left\{\frac{1}{f_n(z)}\right\}$は$A$において0に一様収束する。

したがって、$\left\{\frac{1}{f_n(z)}\right\}$は0に広義一様収束する。(証明終)