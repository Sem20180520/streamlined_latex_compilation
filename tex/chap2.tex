\chapter{複素函数}%第2章

\begin{mysimplebox}{問1}
    点集合$E\subset\C$が閉集合であるためには$E$の境界$\partial E$が$E$の部分であることが必要十分であることを示せ。 
\end{mysimplebox}
\paragraph{証明}
($\Rightarrow$)本の定義では$E$が閉であるとは$E$の集積点がすべて$E$に属することである(p.22)。

$E$は閉であるとしたとき、$\partial E\subset E$であることを示す。
すなわち、$x\in\partial E$ならば$x\in E$であることを示す。
$r\in\R$に対して
\begin{align*}
    B_r(x)=\left\{y\in\C\mid\|y-x|<r\right\}
\end{align*}
と定義する。$E$の境界$\partial E$の点$x$は$E$の内点でも外点でもない。特に、外点でないことから、任意の$n\in \N$に対して、$B_{1/n}(x)$は$E$と交わりをもつ。よって、任意の$n\in \N$に対して、$x_n\in B_{1/n}(x)\cap E$をとれる。

もし$|x-x_n|=0$となる$n\in\N$が存在すれば、$x=x_n\in E$である。

すべての$n\in\N$に対して$|x-x_n|\neq 0$であるとする。$|x-x_n|\longrightarrow 0 (n\longrightarrow 0)$であり、$\{x_n\}\subset E$であるから、$x$は$E$の集積点である。したがって、$E$は閉であると仮定していたから、閉であることの定義より$x\in E$である。

($\Leftarrow$)
$\partial E\subset E$とする。$x$が$E$の任意の集積点であるとき、$x\in E$であることを示せばよい。ここで、$\C$の任意の点は$E$の内点か外点か境界の点であることに注意する。

もし$E$の集積点$x$が$E$の外点であるとすると、ある$r\in\R$に対して$B_r(x)\subset E^c$($E^c$は$E$の補集合を表す)であるから、$x$が$E$の集積点であることに反する。

よって、$E$の集積点$x$は$E$の内点か境界の点である。内点ならば$x\in E$は明らかである。境界の点ならば$\partial E\subset E$という仮定から、やはり$x\in E$である。(証明終)

\begin{mysimplebox}{問2}
    点集合の境界は閉集合であることを示せ。 
\end{mysimplebox}
\paragraph{証明}
$E$を点集合とし、$x$を$\partial E$の集積点とする。$x\in\partial E$であることを示せばよい。

$x$を$E$の内点とすると、ある$n\in\N$に対して、$B_{1/n}(x)\subset E$である。一方、$x$は$\partial E$の集積点であるから、ある$y\in\partial E$が存在して、$y\in B_{1/n}(x)\subset E$である。十分小さい$n'\in\N$をとれば、$B_{1/n'}(y)\subset E$であるから、$y$は$E$の内点である。しかし、これは$y\in\partial E$に反する。よって$x$は$E$の内点ではない。

$x$を$E$の外点としても同様の議論ができるから、$x$は$E$の外点でもない。

したがって、$x$は$E$の境界の点である。(証明終)

\begin{mysimplebox}{問3(Borel--Lebesgueの被覆定理)}
    有界な閉集合$E\subset\C$の各点においてそれぞれ1つの近傍が与えられているとする。
    そのとき、これら無数の近傍の中から有限個を選び出して、
    $E$のいかなる点もこれら有限個の近傍のいずれかに含まれるようにできる。
    これを示せ。
\end{mysimplebox}
\paragraph{証明}
有界な閉集合$E$の各点においてそれぞれ1つの近傍が与えられているとする。
このとき、$E$が有限被覆可能でないと仮定して矛盾を導く。

$E$が4点$R+Ri, -R+Ri, -R-Ri, R-Ri$を頂点とする閉正方形$Q_0$
に包まれるように正数$R$を大きくとる。$E$が有界であることから、これは可能である。

$Q_0$の対辺の中点を結んで$Q_0$を4つの閉正方形に分けると、
そのうちの少なくとも1つと$E$との共通部分は有限被覆可能でない。
その閉正方形を$Q_1$とする。

この手続きを繰り返して、閉正方形の列
$Q_0, Q_1, Q_2,\dots$を作る。
各$Q_k$と$E$の共通部分は、最初に与えられた$E$の各点の近傍に関して、
有限被覆可能ではない。

ここで、Weierstrass--Bolzanoの定理により、$\{Q_k\}$の部分列の中に
収束するものが存在する。
例えば閉正方形$Q_k$の右上の頂点を$a_k+b_{k}i (a_k,b_k\in\R)$、
左下の頂点を$c_k+d_{k}i (c_k,d_k\in\R)$とすると、
$\{a_k\}, \{b_k\},\{c_k\},\{d_k\}$にも収束する部分列がある。
$a_k-c_k=b_k-d_k=\frac{2R}{2^k}$であり、$k\longrightarrow\infty$の極限で0に収束するため、
区間縮小法により、$\{Q_k\}$の収束する部分列は1点に収束する。
$\{Q_k\}$の収束するある部分列を$\{Q'_k\}$とし、その収束点を$q$とする。

$\{Q'_k\}$の各項は$E$との共通部分が有限被覆可能でないから、
任意の$k$に対して$Q'_k\cap E\neq\emptyset$である。
よって、各$k$に対して$x_k\in Q'_k\cap E$をとることができる。
これは$q$の任意の近傍に$E$の点が存在することを示している。
ゆでに$q$は$E$の集積点である。
$E$が閉であることから$q\in E$である。

したがって、最初に$E$の各点に与えられた近傍として、
$q$にも近傍$V_q$が与えられており、
十分大きい$k$ に対して$Q'_k\subset V_q$である。

これは$Q'_k$が有限被覆可能でないことに反する。(証明終)

