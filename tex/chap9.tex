\chapter{Picardの定理}%第9章

\begin{mysimplebox}{問1}
    領域$|z|>0$、$|\arg z|<\pi$で
    \begin{align*}
        \int_{1}^{z}\frac{d\zeta}{\zeta}=\log|z|+i\arg z
    \end{align*}
    である。
\end{mysimplebox}
\paragraph{証明}
$\frac{1}{\zeta}$は$\zeta=0$に極を有するが、問題の領域においては積分路は原点の周りを一周することはない。
よって、この領域において$\frac{1}{\zeta}$は正則であるからCauchyの積分定理より、領域内で積分路を自由にとってよく、
\begin{align*}
    \int_{1}^{z}\frac{d\zeta}{\zeta}
    =\left[\log\zeta\right]_1^{z}
    =\log z
    =\log |z|e^{i\arg z}
    =\log |z|+i\arg z
\end{align*}
が得られる。

または、実軸上を1から$|z|$まで動き、原点を中心とする半径$|z|$の円周上を偏角を0から$\arg z$まで動く積分路で考えると、
\begin{align*}
    \int_{1}^{z}\frac{d\zeta}{\zeta}
    =\int_{1}^{|z|}\frac{dx}{x}
    +\int_{0}^{\arg z}\frac{ire^{i\theta}d\theta}{re^{i\theta}}
    =\left[\log x\right]_1^{|z|}
    +i\int_{0}^{\arg z}d\theta
    =\log|z|+i\arg z
\end{align*}
(終)

\newpage
\begin{mysimplebox}{問2}
    \begin{align*}
        \binom{\alpha}{0}=1,\quad
        \binom{\alpha}{n}
        =\frac{\alpha(\alpha-1)(\alpha-2)\cdots(\alpha-n+1)}{n!}
    \end{align*}
    とおくと、
    \begin{align*}
        f(z)=\sum_{n=0}^{\infty}\binom{\alpha}{n}z^n
    \end{align*}
    は$|z|<1$で正則で、
    \begin{align*}
        f(z)=\exp[\alpha(\log|1+z|+i\arg(1+z))]
    \end{align*}
\end{mysimplebox}
\paragraph{証明}
\begin{align*}
    \frac{\binom{\alpha}{n}}{\binom{\alpha}{n+1}}
    &=\frac{\alpha(\alpha-1)(\alpha-2)\cdots(\alpha-n+1)}{n!}\cdot
    \frac{(n+1)!}{\alpha(\alpha-1)(\alpha-2)\cdots(\alpha-n)}\\
    &=\frac{n+1}{\alpha-n}
    \longrightarrow -1\quad(n\longrightarrow\infty)
\end{align*}
よって、$f(z)=\sum_{n=0}^{\infty}\binom{\alpha}{n}z^n$の収束半径は1である。よって、$|z|<1$で$f(z)$は正則である(p.47参照)。

ここで、$g(z):=(1+z)^\alpha$は次にように定義されているのであった。
\begin{align*}
    g(z)
    &=(1+z)^\alpha\\
    &=\exp(\alpha\log(1+z))\\
    &=\exp(\alpha(\log|1+z|+i\arg(1+z)))
\end{align*}

% $f(0)=1$であるから、
$\alpha$が整数でないとき、このままでは$g(z)$は多価性を有しているが、今は$\log(1+z)$の主値をとることとする。$g(z)$のテイラー展開を考えると、
\begin{align*}
    g^{(k)}(z)
    &=\alpha(\alpha-1)\cdots(\alpha-k+1)(1+z)^{\alpha-k}\\
    \frac{g^{(k)}(0)}{k!}&=\binom{\alpha}{k}
\end{align*}
また、$f(0)=1$であるから、
$g(z)=f(z)$である。

もし$\log(1+z)$の主値をとっていなければ、
$g(0)\neq1$となる得るため、一般には$g(z)\neq f(z)$である。

(終)

\newpage
\begin{mysimplebox}{問3}
$f(z)$が$|z|<1$で正則単葉であるとき、円$|z|=r$($0<r<1$)の写像$w=f(z)$による像曲線の長さ
\begin{align*}
    L(r)=\int_{0}^{2\pi}|f'(re^{\theta i})|rd\theta
\end{align*}
は$0<r<1$において変数$r$の増加関数である。
\end{mysimplebox}
\paragraph{証明}
問題の式が像曲線の長さを表すことは本のp.93参照。

$f(z)$は$|z|<1$で正則単葉であるため、$|c|<1$に対して$f'(c)\neq0$である。なぜなら、もし$f'(c)=0$とすると、正整数$k$に対して
\begin{align*}
    f'(z)=a_k(z-c)^k+a_{k+1}(z-c)^{k+1}+\cdots
\end{align*}
と書ける。ただし、$a_k\neq0$。よって、
\begin{align*}
    f(z)=a+\frac{a_k}{k+1}(z-c)^{k+1}+\frac{a_{k+1}}{k+2}(z-c)^{k+2}+\cdots
\end{align*}
となり、$z=a$において$k+1$葉である。これは単葉であることに反する。ゆえに、$|c|<1$に対して$f'(c)\neq0$である。

このとき、本のp.161より、$f'(z)=[g(z)]^2$を満足する正則函数$g(z)$が存在する。$g(z)=\sum_{n=0}^{\infty}c_nz^n$とすると、p.98と同じ計算により
\begin{align*}
    L(r)
    &=\int_{0}^{2\pi}|g(re^{\theta i})|^2rd\theta\\
    &=\int_{0}^{2\pi}g(re^{\theta i})\overline{g(re^{\theta i})}rd\theta\\
    &=r\int_{0}^{2\pi}\sum_{n=0}^{\infty}c_nr^ne^{n\theta i}\sum_{m=0}^{\infty}\overline{c_m}r^me^{-m\theta i}d\theta\\
    &=r\sum_{n=0}^{\infty}|c_n|^2r^{2n}2\pi\\
    &=2\pi\sum_{n=0}^{\infty}|c_n|^2r^{2n+1}
\end{align*}
これは$0<r<1$において$r$の増加関数である。(終)

\newpage
\begin{mysimplebox}{問4}
    $f(z)$が領域$D$で正則、$D$の1点$c$で十分小さい正数$r$に対し
    \begin{align*}
        |f(c)|=\frac{1}{2\pi}\int_{0}^{2\pi}|f(c+re^{\theta i})|d\theta
    \end{align*}
    ならば、$D$で$f(z)\equiv f(c)$である。
\end{mysimplebox}
\paragraph{証明}
$f(c)=0$ならば
\begin{align*}
    |f(c)|=0=\frac{1}{2\pi}\int_{0}^{2\pi}|f(c+re^{\theta i})|d\theta
\end{align*}
よって、$0\le\theta\le2\pi$において$|f(c+re^{\theta i})|=0$である。よって$f(z)\equiv0$である(p.88参照)。

$f(c)\neq0$ならば、$|z|$が十分小さいとき、
$f(z+c)=[g(z)]^2$を満足する正則函数$g(z)$が存在する。前問と同様の計算により、
\begin{align*}
    |f(c)|
    &=\frac{1}{2\pi}\int_{0}^{2\pi}|g(re^{\theta i})|^2d\theta\\
    &=\frac{1}{2\pi}\int_{0}^{2\pi}g(re^{\theta i})\overline{g(re^{\theta i})}d\theta\\
    &=\frac{1}{2\pi}\int_{0}^{2\pi}\sum_{n=0}^{\infty}c_nr^ne^{n\theta i}\sum_{m=0}^{\infty}\overline{c_m}r^me^{-m\theta i}d\theta\\
    &=\frac{1}{2\pi}\sum_{n=0}^{\infty}|c_n|^2r^{2n}2\pi\\
    &=\sum_{n=0}^{\infty}|c_n|^2r^{2n}
\end{align*}
$|f(c)|$は$r$に依存しないから、$n\ge1$に対して$c_n=0$である。

よって、
\begin{align*}
    f(z)\equiv c_0^2=f(c)
\end{align*}
(終)