\chapter{導函数}%第3章

\begin{mysimplebox}{問1}
    Cauchy--Riemannの関係式は次の関係式と同等であることを示せ:
    \begin{align*}
        \frac{\partial P}{\partial r}
        =\frac{1}{r}\frac{\partial Q}{\partial \theta},\quad
        \frac{\partial P}{\partial \theta}
        =-r\frac{\partial Q}{\partial r}\quad
        (z=re^{\theta i},r\neq0)
    \end{align*}
\end{mysimplebox}
\paragraph{証明}
$z=re^{\theta i}=r(\cos\theta+i\sin\theta)$、
$x=r\cos\theta,y=r\sin\theta$である。

$x^2+y^2=r^2, \frac{y}{x}=\tan\theta$より
\begin{align*}
    &\frac{\partial r}{\partial x}=\frac{x}{r}=\cos\theta,\quad
    \frac{\partial \theta}{\partial x}=-\frac{\sin\theta}{r}\\
    &\frac{\partial r}{\partial y}=\frac{y}{r}=\sin\theta,\quad
    \frac{\partial \theta}{\partial y}=\frac{\cos\theta}{r}
\end{align*}
である。

微分の変数変換によって以下のような計算ができる。
\begin{align*}
    \begin{bmatrix}
        P_x & Q_x\\
        P_y & Q_y
    \end{bmatrix}
    &=
    \begin{bmatrix}
        \frac{\partial r}{\partial x} & \frac{\partial \theta}{\partial x}\\
        \frac{\partial r}{\partial y} & \frac{\partial \theta}{\partial y}
    \end{bmatrix}
    \begin{bmatrix}
        P_r & Q_r\\
        P_\theta & Q_\theta
    \end{bmatrix}
    =
    \begin{bmatrix}
        \cos\theta & -\frac{\sin\theta}{r}\\
        \sin\theta & \frac{\cos\theta}{r}
    \end{bmatrix}
    \begin{bmatrix}
        P_r & Q_r\\
        P_\theta & Q_\theta
    \end{bmatrix}\\
    %%%%%%%%%%%%%%%%%%%%%
    \begin{bmatrix}
        P_r & Q_r\\
        P_\theta & Q_\theta
    \end{bmatrix}
    &=
    \begin{bmatrix}
        \cos\theta & -\frac{\sin\theta}{r}\\
        \sin\theta & \frac{\cos\theta}{r}
    \end{bmatrix}^{-1}
    \begin{bmatrix}
        P_x & Q_x\\
        P_y & Q_y
    \end{bmatrix}
    =
    \begin{bmatrix}
        \cos\theta & \sin\theta\\
        -r\sin\theta & r\cos\theta
    \end{bmatrix}
    \begin{bmatrix}
        P_x & Q_x\\
        P_y & Q_y
    \end{bmatrix}
\end{align*}
Cauchy--Riemannの関係式より,
$P_x=Q_y, P_y=-Q_x$であるから
\begin{align*}
    \begin{bmatrix}
        P_r & Q_r\\
        P_\theta & Q_\theta
    \end{bmatrix}
    =
    \begin{bmatrix}
        \cos\theta & \sin\theta\\
        -r\sin\theta & r\cos\theta
    \end{bmatrix}
    \begin{bmatrix}
        P_x & -P_y\\
        P_y & P_x
    \end{bmatrix}
\end{align*}

$P_x=P_y=0$のとき$P_r=P_\theta=Q_r=Q_\theta=0$であるから示すべき式が成り立つ\footnote{上の行列の式よりすぐに極座標でのCauchy--Riemannの関係式が導かれるが、ここではわざわざ回転行列で書いた。}。

$P_x^2+P_y^2\neq0$であるとき、
$\frac{P_x}{\sqrt{P_x^2+P_y^2}}=\cos\alpha,
\frac{P_y}{\sqrt{P_x^2+P_y^2}}=\sin\alpha$とする。
$R(\theta)=
\begin{bmatrix}
    \cos\theta & -\sin\theta\\
    \sin\theta & \cos\theta
\end{bmatrix}$とすると
\begin{align*}
    \begin{bmatrix}
        P_r & Q_r\\
        P_\theta & Q_\theta
    \end{bmatrix}
    &=
    \begin{bmatrix}
        1 & 0\\
        0 & r
    \end{bmatrix}
    R(-\theta)\cdot\sqrt{P_x^2+P_y^2}R(\alpha)
    =\sqrt{P_x^2+P_y^2}
    \begin{bmatrix}
        1 & 0\\
        0 & r
    \end{bmatrix}
    R(\alpha-\theta)
\end{align*}

よって、
\begin{align*}
    P_r&=\frac{1}{r}Q_\theta=\sqrt{P_x^2+P_y^2}\cos(\alpha-\theta)\\
    P_\theta&=-rQ_r=-\sqrt{P_x^2+P_y^2}\sin(\alpha-\theta)
\end{align*}

逆に、$P_r=\frac{1}{r}Q_\theta,P_\theta=-rQ_r$が成り立つとき
\begin{align*}
    \begin{bmatrix}
        P_x & Q_x\\
        P_y & Q_y
    \end{bmatrix}
    &=
    \begin{bmatrix}
        \cos\theta & -\frac{\sin\theta}{r}\\
        \sin\theta & \frac{\cos\theta}{r}
    \end{bmatrix}
    \begin{bmatrix}
        P_r & Q_r\\
        P_\theta & Q_\theta
    \end{bmatrix}
    =
   R(\theta)
    \begin{bmatrix}
        1 & 0\\
        0 & \frac{1}{r}
    \end{bmatrix}
    \begin{bmatrix}
        P_r & Q_r\\
        P_\theta & Q_\theta
    \end{bmatrix}\\
    &=
    R(\theta)
    \begin{bmatrix}
        P_r & Q_r\\
        \frac{1}{r}P_\theta & \frac{1}{r}Q_\theta
    \end{bmatrix}
    =
    R(\theta)
    \begin{bmatrix}
        P_r & Q_r\\
        -Q_r & P_r
    \end{bmatrix}
\end{align*}
これより$P_x=Q_y,P_y=-Q_x$が成り立つ。(証明終)

\begin{mysimplebox}{問2}
領域$|z|>0$、$|\arg z|<\pi$で$h(z)=\log|z|+i\arg z$は正則で、
$z=e^{h(z)}=\exp(\log|z|+i\arg z)$であることを示せ。
\end{mysimplebox}
\paragraph{証明}
$r=|z|$、$\theta=\arg z$であるから$h(z)=\log r+i\theta$である。
ただし、$-\pi<\theta<\pi$である。

$P(r,\theta)=\log r, Q(r,\theta)=\theta$とすると
\begin{align*}
    &P_r=\frac{1}{r},\quad
    Q_r=0\\
    &P_\theta=0,\quad
    Q_\theta=1
\end{align*}
である。
よって、$P_r=\frac{1}{r}Q_\theta,P_\theta=-rQ_r$が満たされるため、問1により$h(z)$は$r>0,|\theta|<\pi$で正則である。

また
\begin{align*}
    e^{h(z)}&=e^{\log r+i\theta}=e^{\log r}e^{i\theta}\\
    &=r(\cos\theta+i\sin\theta)=x+iy=z
\end{align*}
(証明終)

\begin{mysimplebox}{問3}
    $c\neq0,e^z=c$である$z$のなかには不等式$|z|\le|\log|c||+\pi$を満たすものが必ず存在することを示せ。
\end{mysimplebox}
\paragraph{証明}
複素平面上の$c\neq0$は$c=|c|e^{i\theta}\ (-\pi\le\theta<\pi)$と書くことができる。

$z=\log|c|+i\theta$とすると
\begin{align*}
    e^z&=e^{\log|c|+i\theta}=e^{\log|c|}e^{i\theta}\\
    &=|c|(\cos\theta+i\sin\theta)=c
\end{align*}
である。

この$z$は次を満たす。
\begin{align*}
    |z|&=|\log|c|+i\theta|\le|\log|c||+|\theta|\le|\log|c||+\pi
\end{align*}
(証明終)

\begin{mysimplebox}{問4}
    $z=c$で$f(z)$が微分可能ならば、
    $z=\overline{c}$で$\overline{f(\overline{z})}$
    は微分可能であることを示せ。
    ただし、$\Im c\neq0$である。
\end{mysimplebox}
\paragraph{証明}
次の極限値が存在するとする。
\begin{align*}
    \lim_{z\to c}\frac{f(z)-f(c)}{z-c}=f'(c)
\end{align*}    
%$\lim_{z\to c}\frac{f(z)-f(c)}{z-c}$は存在するとする。
このとき
\begin{align*}
    \lim_{z\to\overline{c}}
    \frac{\overline{f(\overline{z})}-\overline{f(\overline{\overline{c}})}}{z-\overline{c}}
    &=\lim_{z\to\overline{c}}
    \frac{\overline{f(\overline{z})}-\overline{f(c)}}{z-\overline{c}}
    =\lim_{w\to c}
    \frac{\overline{f(w)}-\overline{f(c)}}{\overline{w}-\overline{c}}
    =\overline{\lim_{w\to c}\frac{f(w)-f(c)}{w-c}}
    =\overline{f'(c)}
\end{align*}
である。ただし、途中で$w=\overline{z}$とおいた。
これで$z=\overline{c}$で$\overline{f(\overline{z})}$は微分可能であることを示せたうえに、その微分係数が$\overline{f'(c)}$であることもわかった。(証明終)

\begin{mysimplebox}{問5}
    $\overline{z}$を$z$の有理函数として表すことは不可能であることを示せ。
\end{mysimplebox}
\paragraph{証明}
$z=x+iy\ (x,y\in\R)$とする。

実函数$P(x,y),Q(x,y)$によって$\overline{z}=P(x,y)+iQ(x,y)$とすると
\begin{align*}
    P(x,y)=x,\quad Q(x,y)=-y
\end{align*}
である。
\begin{align*}
    &P_x=1,\quad Q_x=0\\
    &P_y=0,\quad Q_y=-1
\end{align*}
であるから、$\overline{z}$はCauchy--Riemannの関係式を満たさない。
ゆえに$\overline{z}$は正則でない。

ここで
\begin{align*}
    \overline{z}=\frac{a_0+a_1z+\cdots+a_nz^n}{b_0+b_1z+\cdots+b_mz^m}
\end{align*}
と書けるとする。
この右辺は分母の零点を除いた領域で正則である。
しかし左辺は正則でないから矛盾である。
よって、$\overline{z}$を$z$の有理函数として表すことは不可能である。(証明終)

\begin{mysimplebox}{問6}
    $f(z)$は領域$D$で正則な函数、$C$は$D$の2点$c_1, c_2$を結ぶ$D$内の線分であるとき、$C$上の2点$c, c'$を適当に選べば
    \begin{align*}
        f(c_2)-f(c_1)=(c_2-c_1)[\Re (f'(c))+i\Im(f'(c'))]
    \end{align*}
    とできることを示せ。
\end{mysimplebox}
\paragraph{証明}
$f(z)$の実部と虚部を$P(x,y),Q(x,y)$とする。
つまり、$z=x+yi$($x,y\in\R$)に対して実数関数$P(x,y),Q(x,y)$によって$f(z)=P(x,y)+iQ(x,y)$と書けるとする。
さらに
\begin{align*}
    c_1&=a+bi\\
    c_2&=a+h+(b+k)i
\end{align*}
とする。
ただし、$a,b,h,k\in\R$である。

線分$C$上の点は$t\in[0,1]$によって、
\begin{align*}
    c_1+t(c_2-c_1)=a+bi+t(h+ki)=a+th+(b+tk)i
\end{align*}
と表される。

ここで、次のように$t$に関する函数$g(t)$を定義する。
\begin{align*}
    g(t)&=\frac{f(c_1+t(c_2-c_1))}{c_2-c_1}\\
    &=\frac{P(a+th,b+tk)+iQ(a+th,b+tk)}{h+ki}\\
    &=\frac{(P+iQ)(h-ki)}{h^2+k^2}=\frac{Ph+Qk+i(Qh-Pk)}{h^2+k^2}
\end{align*}

$g(t)$の実部と虚部を$U(t),V(t)$とする。次が成り立つ。
\begin{align*}
    U(t)&=\frac{P(a+th,b+tk)h+Q(a+th,b+tk)k}{h^2+k^2}\\
    V(t)&=\frac{Q(a+th,b+tk)h-P(a+th,b+tk)k}{h^2+k^2}
\end{align*}

$U(t),V(t)$は$t$に関する連続函数であるから、平均値の定理により次が成り立つ。
\begin{align*}
    g(1)-g(0)&=\frac{f(c_2)-f(c_1)}{c_2-c_1}=U(1)-U(0)+i(V(1)-V(0))\\
    &=U'(\theta_1)+iV'(\theta_2)
\end{align*}
ただし、$\theta_1,\theta_2\in(0,1)$である。

Cauchy--Riemannの関係式から、次が成り立つ。
\begin{align*}
    U'(t)=\frac{1}{h^2+k^2}
    &\left\{P_x(a+th,b+tk)h^2+P_y(a+th,b+tk)hk\right.\\
    &\left.+Q_x(a+th,b+tk)hk+Q_y(a+th,b+tk)k^2\right\}\\
    =\frac{1}{h^2+k^2}
    &\left\{P_x(a+th,b+tk)h^2
    +Q_y(a+th,b+tk)k^2\right\}\\
    =\frac{1}{h^2+k^2}
    &P_x(a+th,b+tk)(h^2+k^2)=P_x(a+th,b+tk)\\
    %%%%%%
    V'(t)=\frac{1}{h^2+k^2}
    &\left\{Q_x(a+th,b+tk)h^2+Q_y(a+th,b+tk)hk\right.\\
    &\left.-P_x(a+th,b+tk)hk-P_y(a+th,b+tk)k^2\right\}\\
    =\frac{1}{h^2+k^2}
    &\left\{Q_x(a+th,b+tk)h^2
    -P_y(a+th,b+tk)k^2\right\}\\
    =\frac{1}{h^2+k^2}
    &Q_x(a+th,b+tk)(h^2+k^2)=Q_x(a+th,b+tk)
\end{align*}

よって、
\begin{align*}
    c&=c_1+\theta_1(c_2-c_1)=a+\theta_1 h+(b+\theta_1 k)i\\
    c'&=c_1+\theta_2(c_2-c_1)=a+\theta_2 h+(b+\theta_2 k)i
\end{align*}
とすると、
\begin{align*}
    \frac{f(c_2)-f(c_1)}{c_2-c_1}
    &=P_x(a+\theta_1 h,b+\theta_1 k)+iQ_x(a+\theta_2 h,b+\theta_2 k)\\
    &=\Re(f'(c))+i\Im(f'(c'))
\end{align*}
ここで$f'(z)=P_x+iQ_x$であることを用いた。

以上より、
\begin{align*}
    f(c_2)-f(c_1)=(c_2-c_1)[\Re(f'(c))+i\Im(f'(c'))]
\end{align*}
が成り立つ。
(証明終)