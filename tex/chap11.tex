\chapter{調和函数}%第11章

\begin{mysimplebox}{問1}
    $f(z)$が領域$D$で正則で$f(z)\neq0$ならば、$\log|f(z)|$が$D$で調和函数であることを示せ。
\end{mysimplebox}
\paragraph{証明}
$f(z)=P(x,y)+iQ(x,y)$で実部と虚部を表すとする。
正則性よりコーシー・リーマンの関係式$P_x=Q_y,P_y=-Q_x$が成り立つ。
\begin{align*}
    \frac{\partial}{\partial x}\log|f(z)|
    &=\frac{\partial}{\partial x}\frac{1}{2}\log(P^2+Q^2)
    =\frac{PP_x+QQ_x}{P^2+Q^2}\\
    \frac{\partial^2}{\partial x^2}\log|f(z)|
    &=\frac{1}{(P^2+Q^2)^2}\{(P_x^2+PP_{xx}+Q_x^2+QQ_{xx})(P^2+Q^2)-2(PP_x+QQ_x)^2\}
\end{align*}
$y$についても偏微分の計算をし、コーシー・リーマンの関係式を利用すると、次のようになる(途中計算は省略)。
\begin{align*}
    &\Delta\log|f(z)|\\
    &=\frac{1}{(P^2+Q^2)^2}\{2(P_x^2+P_y^2)(P^2+Q^2)-2(PP_x-QP_y)^2-2(PP_y+QP_x)^2\}\\
    &=\frac{2}{(P^2+Q^2)^2}\{|(P_x,P_y)|^2|(P,Q)|^2-|(P_x,P_y)\cdot(P,-Q)|^2-|(P_x,P_y)\cdot(Q,P)|^2\}\\
    &=\frac{2}{(P^2+Q^2)^2}|(P_x,P_y)|^2|(P,Q)|^2(1-\sin^2\theta-\cos^2\theta)=0
\end{align*}
よって、$\log|f(z)|$が$D$で調和函数であることが示された。
(証明終)

\newpage
\begin{mysimplebox}{問2}
    $x(x^2-3y^2)$を実数部とする整函数を求めよ。
\end{mysimplebox}
\paragraph{解答}
$u(x,y)=x(x^2-3y^2)$とする。

$u(x,y)=x^3-3xy^2$

$\Delta u(x,y)=6x-6x=0$であるから、$u(x,y)$は調和函数である。

$\phi(z)=u_x-iu_y$とすると
\begin{align*}
    \phi(z)&=3x^2-3y^2-i(-6xy)=3(x^2+2ixy-y^2)\\
    &=3(x+iy)^2=3z^2
\end{align*}
よって、次のような積分を計算する。ただし、定数分の違いは無視するため、定数部分の記載は省略する。
\begin{align*}
    \int^{z}\phi(\zeta)d\zeta
    &=z^3=(x+iy)^3=x^3+3x^2iy-3xy^2-iy^3\\
    &=x^3-3xy^2+i(3x^2y-y^3)
\end{align*}
(解答終)

\paragraph{別解}
\begin{align*}
    U=\int u(x,y)dx
    =\int (x^3-3xy^2)dx
    =\frac{x^4}{4}-\frac{3}{2}x^2y^2+g(y)
\end{align*}
$g(y)$は積分定数。
$U_x=u(x,y)$、
$U_y=-3x^2y+g'(y)$。

$h(z)=U_x-iU_y$が全平面で正則になるように$g(y)$を定めればよい。
そのために、コーシー・リーマンの関係式が成り立つようにする。

$U_{xy}=-(-U_{yx})=U_{yx}$は成り立っているから、
$U_{xx}=-U_{yy}$(すなわち$\Delta U=0$)が成り立つようにする。
\begin{align*}
    U_{xx}&=3x^2-3y^2\\
    -U_{yy}&=-(-3x^2+g''(y))=3x^2-g''(y)\\
    g''(y)&=3y^2\\
    g'(y)&=y^3+C_1\\
    g(y)&=\frac{y^4}{4}+C_1y+C_2
\end{align*}
よって、$U_y=-3x^2y+g'(y)=-3x^2y+y^3+C_1$。
\begin{align*}
    h(z)&=U_x-iU_y
    =x^3-3xy^2-i(-3x^2y+y^3+C_1)\\
    &=x^3+3x^2iy-3xy^2-iy^3+iC_1=(x+iy)^3+iC_1\\
    &=z^3+iC_1
\end{align*}
(別解終)

\newpage
\begin{mysimplebox}{問3}
    (67.4)を証明せよ。
\end{mysimplebox}
\paragraph{証明}
$u(z)$は$|z|<R$で調和函数であるとする。$z=re^{i\theta}$、$0\le r<R'<R$とすると、
\begin{align*}
    u(re^{i\theta})
    =\frac{1}{2\pi}\int_{0}^{2\pi}u(R'e^{i\phi})\frac{R'^2-r^2}{R'^2+r^2-2R'r\cos(\theta-\phi)}d\phi
\end{align*}
が成り立っている。

$u(z)$が$|z|\le R$で連続な場合に、次が成り立つことを示す。
\begin{align*}
    u(re^{i\theta})
    =\frac{1}{2\pi}\int_{0}^{2\pi}u(Re^{i\phi})\frac{R^2-r^2}{R^2+r^2-2Rr\cos(\theta-\phi)}d\phi
\end{align*}

(31.1)を示したのと同様に、$R'=\frac{n-1}{n}R$とし、$n\longrightarrow\infty$の極限で、第1式の被積分函数が第2式の被積分函数に$z=|R|$上で一様収束することを見ればよい。

これらの被積分函数は、有界閉領域$\epsilon\le R'\le|R|,0\le\phi\le2\pi$で連続であるから、一様連続である(解析概論p.29の定理14)。よって、$R'\longrightarrow R$の極限をとっても積分は成り立つ。(証明終)

\newpage
\begin{mysimplebox}{問4}
    $u(z)$は単連結な領域$D$で連続な偏導関数を有する実函数、$C:z=z(t)$($a\le t\le b$)は$D$内の正則なJordan閉曲線であるとき、$C$の点$z(t)$において
    \begin{align*}
        \frac{\partial u}{\partial n}
        =\frac{\partial u}{\partial x}\cos\alpha
        +\frac{\partial u}{\partial y}\cos\alpha
        \quad(\alpha=\arg z'(t)-\pi/2)
    \end{align*}
    を$u(z)$の外法線方向の微分係数という。特に、$u(z)$が$D$で調和函数であるときは$\int_C\frac{\partial u}{\partial n}ds=0$である。
\end{mysimplebox}
\paragraph{証明}
単連結領域$D$の境界である$C$の外向きの単位法線を$\bm{n}$とすると
\begin{align*}
    \bm{n}=\begin{bmatrix}
        \cos\alpha\\\sin\alpha
    \end{bmatrix}
\end{align*}
よって、
\begin{align*}
    \int_C\frac{\partial u}{\partial n}ds
    &=\int_C(\frac{\partial u}{\partial x}\cos\alpha
    +\frac{\partial u}{\partial y}\cos\alpha)ds\\
    &=\int_C
    \begin{bmatrix}
        u_x\\u_y
    \end{bmatrix}
    \cdot\begin{bmatrix}
        \cos\alpha\\\sin\alpha
    \end{bmatrix}ds\\
    &=\int_C\nabla u\cdot\bm{u}ds=\int_D\Delta udxdy=0
\end{align*}
ここで、ガウスの定理(解析概論p.411の(G)式、またはp.413の一番下の式)を用いた。(証明終)


\newpage
\begin{mysimplebox}{問5}
    領域$D$で$u(z)$は劣調和函数であるとする。$h(z)$が$D$内の閉領域$\overline{\delta}$で連続な函数で、領域$\delta$では調和函数、$\delta$の境界$\partial\delta$では$h(z)\ge u(z)$ならば、$\delta$の各点で$h(z)\ge u(z)$であることを示せ。
\end{mysimplebox}
\paragraph{証明}
\begin{figure}[h]
    \centering
    \includegraphics[width=11cm]{chap11_fig/prob11-5.jpg}
    \caption{問5と6:入力中}
    \label{fig:chap11_5}
\end{figure}

\newpage
\begin{mysimplebox}{問6}
    $u(z)$が領域$D$で連続であるとき、$D$内のどの閉領域$\delta$をとっても、また、前提の条件を満たすようなどの函数$h(z)$をとっても、$\delta$で$h(z)\ge u(z)$ならば、$u(z)$は$D$で劣調和函数であることを示せ。
\end{mysimplebox}
\paragraph{証明}

\newpage
\begin{mysimplebox}{問7}
    領域$D$で$u(z)$が連続な二次偏導関数を有し、$D$の各点$z$で$\Delta u(z)\ge0$ならば、$u(z)$は$D$で劣調和函数であることを示せ。
\end{mysimplebox}
\paragraph{証明}
\begin{figure}[h]
    \centering
    \includegraphics[width=11cm]{chap11_fig/prob11-7-1.jpg}
    \caption{問7-1:入力中}
    \label{fig:chap11_7-1}
\end{figure}
\begin{figure}[h]
    \centering
    \includegraphics[width=11cm]{chap11_fig/prob11-7-2.jpg}
    \caption{問7-2:入力中}
    \label{fig:chap11_7-2}
\end{figure}