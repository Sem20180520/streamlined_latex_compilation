\chapter{留数}%第7章

\begin{mysimplebox}{問1}
    $\frac{1}{\sin z}$の留数を求める。
\end{mysimplebox}
\paragraph{解答}
$\frac{1}{\sin z}$は$z=n\pi\ (n\in\Z)$に極をもつ。
\begin{align*}
    \lim_{z\to n\pi}\frac{z-n\pi}{\sin z}
    &=\lim_{z\to n\pi}\left(\frac{\sin z}{z-n\pi}\right)^{-1}
    =\lim_{z\to n\pi}\left\{\frac{\sin z-\sin n\pi}{z-n\pi}\right\}^{-1}\\
    &=\left\{\left.\left(\sin z\right)'\right|_{z=n\pi}\right\}^{-1}
    =\left\{\cos n\pi\right\}^{-1}
    =\left\{(-1)^n\right\}^{-1}
    =(-1)^n
\end{align*}
が成り立つから、$\frac{1}{\sin z}$の$z=n\pi\ (n\in\Z)$における留数は$(-1)^n$である。(終)

\newpage
\begin{mysimplebox}{問2}
    $\int_{0}^{+\infty}\frac{\sin x}{x}dx=\frac{\pi}{2}$を証明せよ。
    (ヒント:$\lim_{R\to+\infty}\int_{-R}^{R}\frac{e^{xi}}{x}dx=\pi i$)
\end{mysimplebox}
\paragraph{証明}
上半平面における半径$R$の半円の弧$C_R$と、半径$\epsilon$の半円の弧$C_\epsilon$を考える。ただし、$0<\epsilon<R$である。

$\frac{e^{iz}}{z}$の極は$z=0$であるから、
積分路$\int_C=\int_{C_R}+\int_{-R}^{-\epsilon}+\int_{C_\epsilon}+\int_{\epsilon}^{R}$とすると、
\begin{align}
    \int_C\frac{e^{iz}}{z}dz=0\label{eq:ch7-toi2-1}
\end{align}
である。

\begin{align}
    \int_{C_R}\frac{e^{iz}}{z}dz
    =&\int_{0}^{\pi}\frac{e^{iR(\cos\theta+i\sin\theta)}}{Re^{i\theta}}iRe^{i\theta}d\theta\nonumber\\
    =&i\int_{0}^{\pi}e^{iR\cos\theta}e^{-R\sin\theta}d\theta\nonumber\\
    \left|\int_{C_R}\frac{e^{iz}}{z}dz\right|
    \le&\int_{0}^{\pi}e^{-R\sin\theta}d\theta
    =2\int_{0}^{\pi/2}e^{-R\sin\theta}d\theta\nonumber\\
    \le&2\int_{0}^{\pi/2}e^{-R\frac{2}{\pi}\theta}d\theta
    =2\left[\frac{e^{-2R\theta/\pi}}{-2R/\pi}\right]_0^{\pi/2}\nonumber\\
    =&\frac{\pi}{R}(1-e^{-R})
    \longrightarrow0\quad(R\longrightarrow\infty)\label{eq:ch7-toi2-2}
\end{align}

次に、$C_\epsilon$上において、正則函数$P(z)$として、
\begin{align*}
    \frac{e^{iz}}{z}=\frac{1}{z}+P(z)
\end{align*}
と書ける。
ゆえに、
\begin{align*}
    \int_{C_\epsilon}\frac{e^{iz}}{z}dz
    =&\int_{C_\epsilon}\left(\frac{1}{z}+P(z)\right)dz
    =\int_{\pi}^{0}\frac{i\epsilon e^{i\theta}}{\epsilon e^{i\theta}}d\theta+\int_{C_\epsilon}P(z)dz\\
    =&-\pi i+\int_{C_\epsilon}P(z)dz\\
    \left|\int_{C_\epsilon}P(z)dz\right|
    \le&M\int_{C_\epsilon}dz
    \le M\pi\epsilon\longrightarrow0
    \quad(\epsilon\longrightarrow0)
\end{align*}
よって、
\begin{align}
    \int_{C_\epsilon}\frac{e^{iz}}{z}dz
    \longrightarrow-\pi i\label{eq:ch7-toi2-3}
\end{align}

また、
\begin{align*}
    &\int_{-R}^{-\epsilon}\frac{e^{ix}}{x}dx+\int_{\epsilon}^{R}\frac{e^{ix}}{x}dx\\
    =&\int_{R}^{\epsilon}\frac{e^{-ix}}{-x}(-1)dx+\int_{\epsilon}^{R}\frac{e^{ix}}{x}dx\\
    =&-\int_{\epsilon}^{R}\frac{e^{-ix}}{x}dx+\int_{\epsilon}^{R}\frac{e^{ix}}{x}dx\\
    =&2i\int_{\epsilon}^{R}\frac{\sin x}{x}dx
    \longrightarrow
    2i\int_{0}^{\infty}\frac{\sin x}{x}dx
    \quad(\epsilon\longrightarrow0,R\longrightarrow\infty)
\end{align*}
これは最初の行の2つの積分に現れる$\epsilon$について同時に極限をとっている。$R$についても同様である。このような積分をコーシーの主値積分と呼び、$\mathrm{p.v.}$をつける。

すなわち、
\begin{align}
    \mathrm{p.v.}\int_{-\infty}^{\infty}\frac{e^{ix}}{x}dx=2i\int_{0}^{\infty}\frac{\sin x}{x}dx\label{eq:ch7-toi2-4}
\end{align}
である。

式$(\ref{eq:ch7-toi2-1}),(\ref{eq:ch7-toi2-2}),(\ref{eq:ch7-toi2-3}),(\ref{eq:ch7-toi2-4})$から、
\begin{align*}
    -\pi i+2i\int_{0}^{\infty}\frac{\sin x}{x}dx=0
\end{align*}
したがって、
\begin{align*}
    \int_{0}^{\infty}\frac{\sin x}{x}dx
    =\frac{\pi}{2}
\end{align*}
が成り立つ。(証明終)

\newpage
\begin{mysimplebox}{問3}
    $\int_{0}^{2\pi}\frac{d\theta}{a+\cos\theta}=\frac{2\pi}{\sqrt{a^2-1}}\ (a>1)$を証明せよ。

    (ヒント:$2\cos\theta=z+z^{-1}$、ただし$z=e^{\theta i}$。)
\end{mysimplebox}
\paragraph{証明}
\begin{align*}
    z=&e^{i\theta}=\cos\theta+i\sin\theta\\
    z^{-1}=&e^{-i\theta}=\cos\theta-i\sin\theta\\
    z+z^{-1}=&2\cos\theta\\
    dz=&ie^{i\theta}d\theta=izd\theta
\end{align*}
以上から、
\begin{align*}
    \int_{0}^{2\pi}\frac{d\theta}{a+\cos\theta}
    =&\int_{|z|=1}\frac{1}{a+(z+1/z)/2}\frac{dz}{iz}
    =\frac{2}{i}\int_{|z|=1}\frac{dz}{2az+z^2+1}\\
    =&\frac{2}{i}\int_{|z|=1}\frac{1}{(z-z_+)(z-z_-)}dz\\
    =&\frac{2}{i}\frac{1}{2\sqrt{a^2-1}}\int_{|z|=1}\left(\frac{1}{z-z_+}-\frac{1}{z-z_-}\right)dz\\
    =&\frac{1}{i\sqrt{a^2-1}}(2\pi i-0)
    =\frac{2\pi}{\sqrt{a^2-1}}
\end{align*}
が成り立つ。
積分の計算にはコーシーの積分公式を適用している。

ただし、
\begin{align*}
    z^2+2az+1=0
\end{align*}
を解いて、
\begin{align*}
    z=-a\pm\sqrt{a^2-1}
\end{align*}
であるから、
\begin{align*}
    z_+&=-a+\sqrt{a^2-1}\\
    z_-&=-a-\sqrt{a^2-1}
\end{align*}
としている。
$z_+z_-=1$であるから、
\begin{align*}
    z_-<-1<z_+<0
\end{align*}
である。ゆえに、単位円内にある極は$z_+$のみである。

留数定理を用いるならば、
\begin{align*}
    2\pi i\Res(a_+)
    =&2\pi i\lim_{z\to z_+}(z-z_+)\frac{2}{i}\frac{1}{(z-z_+)(z-z_-)}
    =4\pi\lim_{z\to z_+}\frac{1}{z-z_-}\\
    =&4\pi\frac{1}{z_+-z_-}
    =4\pi\frac{1}{2\sqrt{a^2-1}}
    =\frac{2\pi}{\sqrt{a^2-1}}
\end{align*}
となる。

不定積分を利用すると、
\begin{align*}
    \int\left(\frac{1}{z-z_+}-\frac{1}{z-z_-}\right)dz
    =&\log(z-z_+)-\log(z-z_-)
    =\log\frac{z-z_+}{z-z_-}
\end{align*}
が得られる。

\begin{align*}
    w=f(z)=\frac{z-z_+}{z-z_-}
\end{align*}
とすると、$w$は単位円内では正則である。

単位円周$|z|=1$を$f$で写してできる閉曲線を$\Gamma$とする。単位円周内における$f(z)=0$の根は$z=z_+$の1個だけであるから、
\begin{align*}
    1
    =&\frac{1}{2\pi i}\int_{|z|=1}\frac{f'(z)}{f(z)}dz
    =\frac{1}{2\pi i}\int_{\Gamma}\frac{dw}{w}
\end{align*}
である。
これは$z$が単位円周を半時計回りに1周するとき、$w$も$\Gamma$を反時計回りに1周することを意味する。これは書籍のp.145(下から6行目)の内容である。

念のため、確認しておく。$|z|=1$から、
\begin{align*}
    |z_-w-z_+|=|w-1|
\end{align*}
両辺2乗して整理すると(途中計算は省略する)、
\begin{align*}
    |w|=a-\sqrt{a^2-1}=-z_+
\end{align*}
が得られる。

また、
\begin{align*}
    z&=1& &w=\frac{1-z_+}{1-z_-}=a-\sqrt{a^2-1}=-z_+>0\\
    z&=i& &w=\frac{i-z_+}{i-z_-}=\frac{2+i(z_+-z_-)}{z_-^2+1}\\
    z&=-1& &w=\frac{-1-z_+}{-1-z_-}=-a+\sqrt{a^2-1}=z_+<0\\
    z&=-i& &w=\frac{-i-z_+}{-i-z_-}=\frac{2-i(z_+-z_-)}{z_-^2+1}
\end{align*}
であるから、$w$が反時計回りに動くことがわかる。

以上から、
\begin{align*}
    &\int_{|z|=1}\left(\frac{1}{z-z_+}-\frac{1}{z-z_-}\right)dz\\
    =&\left[\log w\right]_{\theta=0}^{\theta=2\pi}
    =\left[\log(-z_+e^{i\theta})\right]_{\theta=0}^{\theta=2\pi}
    =\left[\log(-z_+)+i\theta\right]_{\theta=0}^{\theta=2\pi}
    =2\pi i
\end{align*}
であるから、再び同じ結果が得られた。(証明終)

\newpage
\begin{mysimplebox}{問4}
    $f(z)$が$|z|<R$で正則、$|z|=r\ (<R)$で$f(z)\neq0$であるとき、$|z|<r$における$f(z)$の零点の数は
    \begin{align*}
        \frac{1}{2\pi}\int_{0}^{2\pi}\Re\left(z\frac{f'(z)}{f(z)}\right)d\theta
    \end{align*}
    であることを示せ。ただし、$z=re^{\theta i}$である。
\end{mysimplebox}
\paragraph{証明}
$|z|<r$における$f(z)$の零点の数を$N$とすると、次が成り立つ。
\begin{align*}
    \frac{1}{2\pi i}\int_{|z|=r}\frac{f'(z)}{f(z)}dz=N\\
\end{align*}
この左辺を変形すると、
\begin{align*}
    \frac{1}{2\pi i}\int_{|z|=r}\frac{f'(z)}{f(z)}dz
    =&\frac{1}{2\pi i}\int_{0}^{2\pi}\frac{f'(z)}{f(z)}izd\theta
    =\frac{1}{2\pi}\int_{0}^{2\pi}z\frac{f'(z)}{f(z)}d\theta\\
\end{align*}
であるが、これは実数であるため、実部だけをとっても値は同じである。

よって、
\begin{align*}
    \frac{1}{2\pi}\int_{0}^{2\pi}\Re\left(z\frac{f'(z)}{f(z)}\right)d\theta=N
\end{align*}
が成り立つ。(証明終)

\newpage
\begin{mysimplebox}{問5}
    $f(z)=\sum_{n=0}^{\infty}c_nz^n$が収束円の内部$|z|<R$で単葉ならば、写像$w=f(z)$による$|z|\le r<R$の像の面積は$\sum_{n=0}^{\infty}n|c_n|^2r^{2n}$であることを示せ。
\end{mysimplebox}
\paragraph{証明}
\begin{align*}
    w&=f(z)=u(x,y)+iv(x,y)\\
    \frac{\partial f(z)}{\partial x}&=f_x(z)\\
    \frac{\partial f(z)}{\partial y}&=f_y(z)\\
\end{align*}
とすると、
\begin{align*}
    f_xdx&=(u_x+iv_x)dx\leftrightarrow
    \begin{bmatrix}
        u_x\\v_x
    \end{bmatrix}dx\\
    f_ydy&=(u_y+iv_y)dy\leftrightarrow
    \begin{bmatrix}
        u_y\\v_y
    \end{bmatrix}dy\\
\end{align*}
のように2次元ベクトルと見ることで、微小面積$dxdy$は$f$により、
\begin{align*}
    f_xdx\times f_ydy
    =&(u_xv_y-v_xu_y)dxdy\\
    =&(u_x^2+v_x^2)dxdy=|f_x|^2dxdy\\
    =&(v_y^2+u_y^2)dxdy=|f_y|^2dxdy
\end{align*}
となる。ここで$f$が正則であることから、コーシー--リーマンの関係式が成り立つことを利用している。

また、微分可能な複素函数は、任意の方向微分が等しいから、
\begin{align*}
    f'(z)&=\lim_{\Delta x\to0}\frac{f(x+\Delta x+iy)-f(x+iy)}{\Delta x}=f_x(z)\\
    &=\lim_{\Delta y\to0}\frac{f(x+i(y+\Delta y))-f(x+iy)}{i\Delta y}=-if_y(z)
\end{align*}
が成り立つ(書籍のp.42も参照)。

よって、\footnote{これは要するにヤコビアンを計算している。『定本 解析概論』のp.218も参照。}
\begin{align*}
    f_xdx\times f_ydy
    =&|f'(z)|^2dxdy
\end{align*}
であるから、求める像の面積は
\begin{align*}
    &\int_{|z|\le r}|f'(z)|^2dxdy\\
    =&\int_{0}^{r}\int_{0}^{2\pi}\left(\sum_{n=1}^{\infty}nc_nz^{n-1}\right)\left(\sum_{m=1}^{\infty}m\overline{c_mz^{m-1}}\right)rdrd\theta\\
    =&\sum_{n,m}nmc_n\overline{c_m}\int_{0}^{r}
    \int_{0}^{2\pi}r^{n+m-2}e^{i(n-m)\theta}rdrd\theta\\
    =&\sum_{n,m}nmc_n\overline{c_m}\int_{0}^{r}
    r^{n+m-2}2\pi\delta_{nm}rdr\\
    =&2\pi\sum_{n}n^2|c_n|^2\int_{0}^{r}
    r^{2n-1}dr\\
    =&2\pi\sum_{n}n^2|c_n|^2\left[\frac{r^{2n}}{2n}\right]_0^r\\
    =&2\pi\sum_{n}n^2|c_n|^2\frac{r^{2n}}{2n}\\
    =&\pi\sum_{n}n|c_n|^2r^{2n}
\end{align*}
(証明終)

\paragraph{疑問}
問3で現れた、
\begin{align*}
    w=f(z)=\frac{z-z_+}{z-z_-}
\end{align*}
も単葉函数である。

$|z|\le1$を$f$で写すと、像は原点を中心とする半径$-z_+$の円であったから、
\begin{align*}
    \int_{|z|\le 1}|f'(z)|^2dxdy=\pi z_+^2
\end{align*}
となると思うが、計算できるだろうか?

$\alpha=z_+$、$\beta=z_-$と置き直す。
$\alpha+\beta=-2a,\ \alpha\beta=1$が成り立つ。

\begin{align*}
    f(z)&=\frac{z-\alpha}{z-\beta}\\
    f'(z)&=\frac{(z-\beta)-(z-\alpha)}{(z-\beta)^2}
    =\frac{\alpha-\beta}{(z-\beta)^2}=\frac{2\sqrt{a^2-1}}{(z-\beta)^2}\\
    \left|f'(z)\right|^2&=f'(z)\overline{f'(z)}
    =\frac{4(a^2-1)}{(z-\beta)^2\overline{(z-\beta)^2}}\\
    z&=re^{i\theta}\\
    (z-\beta)\overline{(z-\beta)}&=|z|^2-(z+\overline{z})\beta+\beta^2\\
    &=r^2-2r\beta\cos\theta+\beta^2\\
    I=\int_{|z|\le 1}|f'(z)|^2dxdy
    &=\int_{0}^{2\pi}\int_{0}^{1}\frac{4(a^2-1)}{(r^2-2r\beta\cos\theta+\beta^2)^2}rdrd\theta
\end{align*}
$r-2\beta\cos\theta=t$とすると、$dr=dt$であり、
\begin{align*}
    &r^2-2r\beta\cos\theta+\beta^2\\
    &=(r-\beta\cos\theta)^2-\beta^2\cos^2\theta+\beta^2\\
    &=t^2+\beta^2\sin^2\theta\\
    I&=4(a^2-1)\int_{0}^{2\pi}d\theta\int_{1-\beta\cos\theta}^{-\beta\cos\theta}\frac{t+\beta\cos\theta}{(t^2+\beta^2\sin^2\theta)^2}dt
\end{align*}
$t$に関する積分は、次の不定積分を利用すればよい。
\begin{align*}
    b\in\R,\ b\neq0&\\
    \int\frac{xdx}{(x^2+b^2)^n}&=\begin{cases}
        \frac{-1}{2(n-1)}\cdot\frac{1}{(x^2+b^2)^{n-1}}, & n>1\\
        \frac{1}{2}\log(x^2+b^2), & n=1
    \end{cases}\\
    I_n=\int\frac{dx}{(x^2+b^2)^n}
    &=\begin{cases}
        \frac{1}{b^2}\left\{\frac{x}{(2n-2)(x^2+b^2)^{n-1}}+\frac{2n-3}{2n-2}I_{n-1}\right\}, & n>1\\
        \frac{1}{b}\arctan\frac{x}{b}, & n=1
    \end{cases}
\end{align*}

\begin{align*}
    &\int_{0}^{2\pi}d\theta\int_{1-\beta\cos\theta}^{-\beta\cos\theta}\frac{t}{(t^2+\beta^2\sin^2\theta)^2}dt\\
    &=\int_{0}^{2\pi}d\theta\left[-\frac{1}{2}\frac{1}{t^2+\beta^2\sin^2\theta}\right]_{1-\beta\cos\theta}^{-\beta\cos\theta}\\
    &=-\frac{1}{2}\int_{0}^{2\pi}\left(\frac{1}{1-2\beta\cos\theta+\beta^2}-\frac{1}{\beta^2}\right)d\theta
\end{align*}

(入力保留中)


\newpage
\begin{mysimplebox}{問6}
    $z+\frac{1}{z}$は$|z|<1$および$|z|>1$においてそれぞれ単葉である。写像$w=z+\frac{1}{z}$による$|z|<1$、$|z|>1$の像領域を求めよ。また、円$|z|=r$、半直線$\arg z=\theta$の像曲線を求めよ。
\end{mysimplebox}
\paragraph{解答}
\begin{align*}
    z&=re^{i\theta}\\
    z+\frac{1}{z}&=re^{i\theta}+\frac{1}{re^{i\theta}}
    =\left(r+\frac{1}{r}\right)\cos\theta+i\left(r-\frac{1}{r}\right)\sin\theta
\end{align*}

\begin{align*}
    u&=\left(r+\frac{1}{r}\right)\cos\theta,\ v=\left(r-\frac{1}{r}\right)\sin\theta
\end{align*}
とすると、円$|z|=r$(ただし$r\neq1$)に対して、
\begin{align*}
    &\left\{\frac{u}{\left(r+\frac{1}{r}\right)}\right\}^2
    +\left\{\frac{v}{\left(r-\frac{1}{r}\right)}\right\}^2
    =\cos^2\theta+\sin^2\theta=1
\end{align*}
である。
よって、円$|z|=r$(ただし$r\neq1$)の像曲線は、
長軸の長さが$2\left(r+\frac{1}{r}\right)\ge2$、
短軸の長さが$2\left|r-\frac{1}{r}\right|$の楕円である。

なお、$|z|=r$の反時計回りに対して、
$0<r<1$のときは$r-\frac{1}{r}<0$であるから、
楕円の時計回りが対応する。
$r>1$のときは$r-\frac{1}{r}>0$であるから、
楕円の反時計回りが対応する。

$r=1$のとき、
$z+\frac{1}{z}=2\cos\theta$
である。これは、点$(2,0)$と点$(-2,0)$
を端点とする線分である。すなわち、単位円は長さ4の線分にうつされる。
\begin{figure}[h]
    \centering
    \begin{minipage}{0.4\columnwidth}
        \centering
        \includegraphics[width=4cm]{chap7_fig/chap7-prob6-1.jpg}
        \caption{$|z|=r,r\neq1$の像曲線}
        \label{fig:elliptic1}
    \end{minipage}
    \begin{minipage}{0.4\columnwidth}
        \centering
        \includegraphics[width=4cm]{chap7_fig/chap7-prob6-2.jpg}
        \caption{$|z|=r$の像曲線の変化}
        \label{fig:elliptic2}
    \end{minipage}
\end{figure}


\newpage
(再掲)
\begin{align*}
    z&=re^{i\theta}\\
    z+\frac{1}{z}&=re^{i\theta}+\frac{1}{re^{i\theta}}
    =\left(r+\frac{1}{r}\right)\cos\theta+i\left(r-\frac{1}{r}\right)\sin\theta\\
    u&=\left(r+\frac{1}{r}\right)\cos\theta,\ v=\left(r-\frac{1}{r}\right)\sin\theta
\end{align*}
次に、半直線$\arg z=\theta$(ただし$\theta\neq n\pi/2$)に対して、
\begin{align*}
    &\frac{u}{\cos\theta}=r+\frac{1}{r},
    \ \frac{v}{\sin\theta}=r-\frac{1}{r}\\
    &\left(\frac{v}{\sin\theta}\right)^2
    =r^2-2+\frac{1}{r^2}
    =\left(r+\frac{1}{r}\right)^2-4
    =\left(\frac{u}{\cos\theta}\right)^2-4\\
    &\left(\frac{u}{\cos\theta}\right)^2-\left(\frac{v}{\sin\theta}\right)^2=4\\
    &\left(\frac{u}{2\cos\theta}\right)^2-\left(\frac{v}{2\sin\theta}\right)^2=1
\end{align*}
であるから、半直線$\arg z=\theta$(ただし$\theta\neq n\pi/2$)の像曲線は、
\begin{align*}
(\pm\sqrt{(2\cos\theta)^2+(2\cos\theta)^2},0)=(\pm2,0)   
\end{align*}
を焦点とする双曲線の右側($\cos\theta>0$のとき)、または左側($\cos\theta<0$のとき)である。
\begin{figure}[h]
    \centering
    \includegraphics[width=7cm]{chap7_fig/chap7-prob6-3.jpg}
    \caption{$\arg z=\theta$の像曲線}
    \label{fig:hyperbola}
\end{figure}

$\theta=0,\pi,2\pi$のとき、$u=\pm\left(r+\frac{1}{r}\right),v=0$であるから、像曲線は
\begin{align*}
    &\{x\in\R\mid x\ge2\},\quad(\theta=0,2\pi)\\
    &\{x\in\R\mid x\le -2\},\quad(\theta=\pi)
\end{align*}
である。

$\theta=\pi/2,3\pi/2$のとき、$u=0,v=\pm\left(r-\frac{1}{r}\right)$であるから、像曲線は虚軸($x=0$)である。


$|z|<1$、$|z|>1$の像領域はともに、複素平面から実軸の長さ4の線分を除いた部分、
\begin{align*}
    \C\setminus\{z\in\C\mid -2\le \Re(z)\le2,\Im(z)=0\}
\end{align*}
である。(解答終)

\newpage
\begin{mysimplebox}{問7}
    Rouch\'{e}の定理を用いて代数学の基本定理を証明せよ。
\end{mysimplebox}
\paragraph{証明}
\begin{align*}
    z^n+a_{n-1}z^{n-1}+\cdots+a_1z+a_0=0\quad(a_k\in\C,k=1,\dots,n-1)
\end{align*}
が重複度も込めて$n$個の根をもつことを示す。
左辺を$\phi(z)$とする。

$z^n=0$は0を$n$重根としてもつことは既知とする。
\begin{align*}
    a:=\max_{k=1,\dots,n-1}|a_k|
\end{align*}
とする。
$|z|=R>0$のとき、
\begin{align*}
    |\phi(z)-z^n|=&|a_{n-1}z^{n-1}+\cdots+a_1z+a_0|\\
    \le&a|z|^{n-1}+\dots+a|z|+a\\
    \le&a(R^{n-1}+\dots+R+1)=a\frac{R^n-1}{R-1}
\end{align*}
が成り立つ。ここで、
\begin{align*}
    R^n-a\frac{R^n-1}{R-1}
    &=\frac{1}{R-1}(R^{n+1}-R^n-aR^n+a)\\
    &=\frac{1}{R-1}\left[R^{n}\{R-(a+1)\}+a\right]\\
\end{align*}
であるから、$R>\max\{1,a+1\}$ならば、
\begin{align*}
    R^n>a\frac{R^n-1}{R-1}
\end{align*}
が成り立つ。

よって、このように$R$をとれば、$|z|=R$の円周上において、
\begin{align*}
    |z|^n>|\phi(z)-z^n|
\end{align*}
が成り立つから、
\begin{align*}
    z^n+(\phi(z)-z^n)=\phi(z)
\end{align*}
は$|z|=R$において零点をもたない。

ゆえに、Rouch\'{e}の定理から、
円$|z|<R$の内部にある$z^n$の零点の数と\\
$z^n+(\phi(z)-z^n)=\phi(z)$
の零点の個数は等しい。

したがって、$\phi(z)=0$は$n$個の根をもつ。(証明終)


\newpage
\begin{mysimplebox}{問8}
    $f(z)$が領域$D$で正則、$|z|\le1$が$D$に包まれて、$|z|=1$では$|f(z)|<1$のとき、$|z|<1$における$f(z)$の零点の数を求めよ。
\end{mysimplebox}
\paragraph{解答}
$k=1,2,\dots,n$に対して$a_k\in\C$、$|a_k|<1$とする。
$|\xi|=1$ならば、
\begin{align*}
    |(\xi-a_1)(\xi-a_2)\cdots(\xi-a_n)|
    \le2^n
\end{align*}
であるから、
\begin{align*}
    f(z)=\frac{1}{2^n}(z-a_1)(z-a_2)\cdots(z-a_n)
\end{align*}
とすると、$f(z)$は$\C$において正則であり、
$|z|=1$において$|f(z)|<1$が成り立つ。

以上のことから、この問題の条件を満たす正則函数で任意個の零点をもつものが構成できる。よって、零点の個数は一意に定まらない。(解答終)

\paragraph{考察}
この問題に意味をもたせるにはどうすればよいか。

たとえば、$f(z)$はこの問題の条件を満たす正則函数であるとし、
単位円$|z|=1$で$|g(z)|\ge1$となる函数$g(z)$に対して、
$g(z)$と$g(z)+f(z)$の単位円内における零点の個数は等しい。

例を挙げれば、$z^n+f(z)$の単位円内の零点の個数は$n$個である。

しかし、これでは問題として簡単すぎる気がする。

\newpage
\begin{mysimplebox}{問9}
    整函数$f_n(z)$($n=1,2,\dots$)の零点がすべて実軸上にあるとき、
    $\{f_n(z)\}$が$|z|<\infty$で$f(z)$に広義一様収束すれば、
    $f(z)$が定数でない限り、$f(z)$の零点はすべて実軸上にあることを示せ。
\end{mysimplebox}
\paragraph{証明}
Hurwitzの定理の証明にならって示す。

整函数列が広義一様収束するとき、極限函数$f(z)$は$\C$で正則である。
$f(z)$が実軸上以外に零点$\alpha$をもつと仮定し、矛盾を導く。

曲線$C$は$\alpha$を内部に含み、実軸と交わらないJordan閉曲線であるとする。
$f(z)$が定数でないとすると、$C$を実軸と交わらないまま変形して、$C$上に$f(z)$の零点がないようにできる。

なぜなら、もし$\alpha$を内部に含み実軸と交わらないJordan閉曲線をどのようにとっても、その上に$f(z)$の零点が存在するならば、$\alpha$の任意の近傍に収まるJordan閉曲線をとり続けることにより、$f(z)$は$\alpha$に収束する点列において$f(z)=0$を満たし、$f(z)\equiv0$が導かれるからである。

%$f_n(z)$は実軸にしか零点をもたないから、$C$上で$f_n(z)\neq0$である。
%%このことは使わない。

よって、$C$上で$f(z)\neq0$としてよく、
\begin{align*}
    m:=\min_{z\in C}|f(z)|
\end{align*}
とすると、$m>0$である。

$\{f_n(z)\}$は広義一様収束するから、$C$上で、
$n>N$ならば$|f_n(z)-f(z)|<m$が成り立つような
$N>0$が存在する。よって、$C$上で、
\begin{align*}
    |f_n(z)-f(z)|<m\le f(z)
\end{align*}
であるから、
\begin{align*}
    f(z)+\lambda (f_n(z)-f(z))\quad(0\le\lambda\le1)
\end{align*}
は$C$上で零点をもたない。

よって、Rouch\'{e}の定理から、
$C$の内部において、$f(z)$と\\
$f(z)+(f_n(z)-f(z))=f_n(z)$の零点の個数は等しい。

すなわち、$C$内で$f(z)$は零点をもたないことになるが、これは矛盾である。以上から、$f(z)$は実軸上以外に零点をもたない。(証明終)

\newpage
\begin{mysimplebox}{問10}
    領域$D$で零点をもたない正則函数から成る函数列が$D$で広義一様収束するとき、
    もし極限函数$f(z)$が零点をもてば、$D$で$f(z)\equiv0$である。
\end{mysimplebox}
\paragraph{証明}
$f(z)\not\equiv0$とし、$f(z)$の零点を$a$とする。
$D$に含まれるJordan閉曲線$C$は内部に$a$を含むとする。

さらに、前問と同様に、$C$上で$f(z)\neq0$としてよい。

\begin{align*}
    m:=\min_{z\in C}|f(z)|
\end{align*}
とすると、$m>0$である。

十分大きい$N>0$に対して、$n>N$ならば$C$上で$|f_n(z)-f(z)|<m$であるから、
\begin{align*}
    f(z)+(f_n(z)-f(z))=f_n(z)
\end{align*}
は$C$上で零点をもたない。

よって、Rouch\'{e}の定理より、
$C$内における$f(z)$と$f_n(z)$の零点の個数は等しいが、これは矛盾である。

したがって、$D$で$f(z)\equiv0$である。(証明終)