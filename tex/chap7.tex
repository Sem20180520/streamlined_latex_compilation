\chapter{留数}%第7章

\begin{mysimplebox}{問1}
    $\frac{1}{\sin z}$の留数を求める。
\end{mysimplebox}
\paragraph{解答}
$\frac{1}{\sin z}$は$z=n\pi\ (n\in\Z)$に極をもつ。
\begin{align*}
    \lim_{z\to n\pi}\frac{z-n\pi}{\sin z}
    &=\lim_{z\to n\pi}\left(\frac{\sin z}{z-n\pi}\right)^{-1}
    =\lim_{z\to n\pi}\left\{\frac{\sin z-\sin n\pi}{z-n\pi}\right\}^{-1}\\
    &=\left\{\left.\left(\sin z\right)'\right|_{z=n\pi}\right\}^{-1}
    =\left\{\cos n\pi\right\}^{-1}
    =\left\{(-1)^n\right\}^{-1}
    =(-1)^n
\end{align*}
が成り立つから、$\frac{1}{\sin z}$の$z=n\pi\ (n\in\Z)$における留数は$(-1)^n$である。(終)

\newpage
\begin{mysimplebox}{問2}
    $\int_{0}^{+\infty}\frac{\sin x}{x}dx=\frac{\pi}{2}$を証明せよ。
    (ヒント:$\lim_{R\to+\infty}\int_{-R}^{R}\frac{e^{xi}}{x}dx=\pi i$)
\end{mysimplebox}
\paragraph{証明}
上半平面における半径$R$の半円の弧$C_R$と、半径$\epsilon$の半円の弧$C_\epsilon$を考える。ただし、$0<\epsilon<R$である。

$\frac{e^{iz}}{z}$の極は$z=0$であるから、
積分路$\int_C=\int_{C_R}+\int_{-R}^{-\epsilon}+\int_{C_\epsilon}+\int_{\epsilon}^{R}$とすると、
\begin{align}
    \int_C\frac{e^{iz}}{z}dz=0\label{eq:ch7-toi2-1}
\end{align}
である。

\begin{align}
    \int_{C_R}\frac{e^{iz}}{z}dz
    =&\int_{0}^{\pi}\frac{e^{iR(\cos\theta+i\sin\theta)}}{Re^{i\theta}}iRe^{i\theta}d\theta\nonumber\\
    =&i\int_{0}^{\pi}e^{iR\cos\theta}e^{-R\sin\theta}d\theta\nonumber\\
    \left|\int_{C_R}\frac{e^{iz}}{z}dz\right|
    \le&\int_{0}^{\pi}e^{-R\sin\theta}d\theta
    =2\int_{0}^{\pi/2}e^{-R\sin\theta}d\theta\nonumber\\
    \le&2\int_{0}^{\pi/2}e^{-R\frac{2}{\pi}\theta}d\theta
    =2\left[\frac{e^{-2R\theta/\pi}}{-2R/\pi}\right]_0^{\pi/2}\nonumber\\
    =&\frac{\pi}{R}(1-e^{-R})
    \longrightarrow0\quad(R\longrightarrow\infty)\label{eq:ch7-toi2-2}
\end{align}

次に、$C_\epsilon$上において、正則函数$P(z)$として、
\begin{align*}
    \frac{e^{iz}}{z}=\frac{1}{z}+P(z)
\end{align*}
と書ける。
ゆえに、
\begin{align*}
    \int_{C_\epsilon}\frac{e^{iz}}{z}dz
    =&\int_{C_\epsilon}\left(\frac{1}{z}+P(z)\right)dz
    =\int_{\pi}^{0}\frac{i\epsilon e^{i\theta}}{\epsilon e^{i\theta}}d\theta+\int_{C_\epsilon}P(z)dz\\
    =&-\pi i+\int_{C_\epsilon}P(z)dz\\
    \left|\int_{C_\epsilon}P(z)dz\right|
    \le&M\int_{C_\epsilon}dz
    \le M\pi\epsilon\longrightarrow0
    \quad(\epsilon\longrightarrow0)
\end{align*}
よって、
\begin{align}
    \int_{C_\epsilon}\frac{e^{iz}}{z}dz
    \longrightarrow-\pi i\label{eq:ch7-toi2-3}
\end{align}

また、
\begin{align*}
    &\int_{-R}^{-\epsilon}\frac{e^{ix}}{x}dx+\int_{\epsilon}^{R}\frac{e^{ix}}{x}dx\\
    =&\int_{R}^{\epsilon}\frac{e^{-ix}}{-x}(-1)dx+\int_{\epsilon}^{R}\frac{e^{ix}}{x}dx\\
    =&-\int_{\epsilon}^{R}\frac{e^{-ix}}{x}dx+\int_{\epsilon}^{R}\frac{e^{ix}}{x}dx\\
    =&2i\int_{\epsilon}^{R}\frac{\sin x}{x}dx
    \longrightarrow
    2i\int_{0}^{\infty}\frac{\sin x}{x}dx
    \quad(\epsilon\longrightarrow0,R\longrightarrow\infty)
\end{align*}
これは最初の行の2つの積分に現れる$\epsilon$について同時に極限をとっている。$R$についても同様である。このような積分をコーシーの主値積分と呼び、$\mathrm{p.v.}$をつける。

すなわち、
\begin{align}
    \mathrm{p.v.}\int_{-\infty}^{\infty}\frac{e^{ix}}{x}dx=2i\int_{0}^{\infty}\frac{\sin x}{x}dx\label{eq:ch7-toi2-4}
\end{align}
である。

式$(\ref{eq:ch7-toi2-1}),(\ref{eq:ch7-toi2-2}),(\ref{eq:ch7-toi2-3}),(\ref{eq:ch7-toi2-4})$から、
\begin{align*}
    -\pi i+2i\int_{0}^{\infty}\frac{\sin x}{x}dx=0
\end{align*}
したがって、
\begin{align*}
    \int_{0}^{\infty}\frac{\sin x}{x}dx
    =\frac{\pi}{2}
\end{align*}
が成り立つ。(証明終)

\newpage
\begin{mysimplebox}{問3}
    $\int_{0}^{2\pi}\frac{d\theta}{a+\cos\theta}=\frac{2\pi}{\sqrt{a^2-1}}\ (a>1)$を証明せよ。

    (ヒント:$2\cos\theta=z+z^{-1}$、ただし$z=e^{\theta i}$。)
\end{mysimplebox}
\paragraph{証明}
\begin{align*}
    z=&e^{i\theta}=\cos\theta+i\sin\theta\\
    z^{-1}=&e^{-i\theta}=\cos\theta-i\sin\theta\\
    z+z^{-1}=&2\cos\theta\\
    dz=&ie^{i\theta}d\theta=izd\theta
\end{align*}
以上から、
\begin{align*}
    \int_{0}^{2\pi}\frac{d\theta}{a+\cos\theta}
    =&\int_{|z|=1}\frac{1}{a+(z+1/z)/2}\frac{dz}{iz}
    =\frac{2}{i}\int_{|z|=1}\frac{dz}{2az+z^2+1}\\
    =&\frac{2}{i}\int_{|z|=1}\frac{1}{(z-z_+)(z-z_-)}dz\\
    =&\frac{2}{i}\frac{1}{2\sqrt{a^2-1}}\int_{|z|=1}\left(\frac{1}{z-z_+}-\frac{1}{z-z_-}\right)dz\\
    =&\frac{1}{i\sqrt{a^2-1}}(2\pi i-0)
    =\frac{2\pi}{\sqrt{a^2-1}}
\end{align*}
が成り立つ。
積分の計算にはコーシーの積分公式を適用している。

ただし、
\begin{align*}
    z^2+2az+1=0
\end{align*}
を解いて、
\begin{align*}
    z=-a\pm\sqrt{a^2-1}
\end{align*}
であるから、
\begin{align*}
    z_+&=-a+\sqrt{a^2-1}\\
    z_-&=-a-\sqrt{a^2-1}
\end{align*}
としている。
$z_+z_-=1$であるから、
\begin{align*}
    z_-<-1<z_+<0
\end{align*}
である。ゆえに、単位円内にある極は$z_+$のみである。

留数定理を用いるならば、
\begin{align*}
    2\pi i\Res(a_+)
    =&2\pi i\lim_{z\to z_+}(z-z_+)\frac{2}{i}\frac{1}{(z-z_+)(z-z_-)}
    =4\pi\lim_{z\to z_+}\frac{1}{z-z_-}\\
    =&4\pi\frac{1}{z_+-z_-}
    =4\pi\frac{1}{2\sqrt{a^2-1}}
    =\frac{2\pi}{\sqrt{a^2-1}}
\end{align*}
となる。

不定積分を利用すると、
\begin{align*}
    \int\left(\frac{1}{z-z_+}-\frac{1}{z-z_-}\right)dz
    =&\log(z-z_+)-\log(z-z_-)
    =\log\frac{z-z_+}{z-z_-}
\end{align*}
が得られる。

\begin{align*}
    w=f(z)=\frac{z-z_+}{z-z_-}
\end{align*}
とすると、$w$は単位円内では正則である。

単位円周$|z|=1$を$f$で写してできる閉曲線を$\Gamma$とする。単位円周内における$f(z)=0$の根は$z=z_+$の1個だけであるから、
\begin{align*}
    1
    =&\frac{1}{2\pi i}\int_{|z|=1}\frac{f'(z)}{f(z)}dz
    =\frac{1}{2\pi i}\int_{\Gamma}\frac{dw}{w}
\end{align*}
である。
これは$z$が単位円周を半時計回りに1周するとき、$w$も$\Gamma$を反時計回りに1周することを意味する。これは書籍のp.145(下から6行目)の内容である。

念のため、確認しておく。$|z|=1$から、
\begin{align*}
    |z_-w-z_+|=|w-1|
\end{align*}
両辺2乗して整理すると(途中計算は省略する)、
\begin{align*}
    |w|=a-\sqrt{a^2-1}=-z_+
\end{align*}
が得られる。

また、
\begin{align*}
    z&=1& &w=\frac{1-z_+}{1-z_-}=a-\sqrt{a^2-1}=-z_+>0\\
    z&=i& &w=\frac{i-z_+}{i-z_-}=\frac{2+i(z_+-z_-)}{z_-^2+1}\\
    z&=-1& &w=\frac{-1-z_+}{-1-z_-}=-a+\sqrt{a^2-1}=z_+<0\\
    z&=-i& &w=\frac{-i-z_+}{-i-z_-}=\frac{2-i(z_+-z_-)}{z_-^2+1}
\end{align*}
であるから、$w$が反時計回りに動くことがわかる。

以上から、
\begin{align*}
    &\int_{|z|=1}\left(\frac{1}{z-z_+}-\frac{1}{z-z_-}\right)dz\\
    =&\left[\log w\right]_{\theta=0}^{\theta=2\pi}
    =\left[\log(-z_+e^{i\theta})\right]_{\theta=0}^{\theta=2\pi}
    =\left[\log(-z_+)+i\theta\right]_{\theta=0}^{\theta=2\pi}
    =2\pi i
\end{align*}
であるから、再び同じ結果が得られた。(証明終)

\newpage
\begin{mysimplebox}{問4}
    $f(z)$が$|z|<R$で正則、$|z|=r\ (<R)$で$f(z)\neq0$であるとき、$|z|<r$における$f(z)$の零点の数は
    \begin{align*}
        \frac{1}{2\pi}\int_{0}^{2\pi}\Re\left(z\frac{f'(z)}{f(z)}\right)d\theta
    \end{align*}
    であることを示せ。ただし、$z=re^{\theta i}$である。
\end{mysimplebox}
\paragraph{証明}
$|z|<r$における$f(z)$の零点の数を$N$とすると、次が成り立つ。
\begin{align*}
    \frac{1}{2\pi i}\int_{|z|=r}\frac{f'(z)}{f(z)}dz=N\\
\end{align*}
この左辺を変形すると、
\begin{align*}
    \frac{1}{2\pi i}\int_{|z|=r}\frac{f'(z)}{f(z)}dz
    =&\frac{1}{2\pi i}\int_{0}^{2\pi}\frac{f'(z)}{f(z)}izd\theta
    =\frac{1}{2\pi}\int_{0}^{2\pi}z\frac{f'(z)}{f(z)}d\theta\\
\end{align*}
であるが、これは実数であるため、実部だけをとっても値は同じである。

よって、
\begin{align*}
    \frac{1}{2\pi}\int_{0}^{2\pi}\Re\left(z\frac{f'(z)}{f(z)}\right)d\theta=N
\end{align*}
が成り立つ。(証明終)

\newpage
\begin{mysimplebox}{問5}
    $f(z)=\sum_{n=0}^{\infty}c_nz^n$が収束円の内部$|z|<R$で単葉ならば、写像$w=f(z)$による$|z|\le r<R$の像の面積は$\sum_{n=0}^{\infty}n|c_n|^2r^{2n}$であることを示せ。
\end{mysimplebox}
\paragraph{証明}
\begin{align*}
    w&=f(z)=u(x,y)+iv(x,y)\\
    \frac{\partial f(z)}{\partial x}&=f_x(z)\\
    \frac{\partial f(z)}{\partial y}&=f_y(z)\\
\end{align*}
とすると、
\begin{align*}
    f_xdx&=(u_x+iv_x)dx\leftrightarrow
    \begin{bmatrix}
        u_x\\v_x
    \end{bmatrix}dx\\
    f_ydy&=(u_y+iv_y)dy\leftrightarrow
    \begin{bmatrix}
        u_y\\v_y
    \end{bmatrix}dy\\
\end{align*}
のように2次元ベクトルと見ることで、微小面積$dxdy$は$f$により、
\begin{align*}
    f_xdx\times f_ydy
    =&(u_xv_y-v_xu_y)dxdy\\
    =&(u_x^2+v_x^2)dxdy=|f_x|^2dxdy\\
    =&(v_y^2+u_y^2)dxdy=|f_y|^2dxdy
\end{align*}
となる。ここで$f$が正則であることから、コーシー--リーマンの関係式が成り立つことを利用している。

また、微分可能な複素函数は、任意の方向微分が等しいから、
\begin{align*}
    f'(z)&=\lim_{\Delta x\to0}\frac{f(x+\Delta x+iy)-f(x+iy)}{\Delta x}=f_x(z)\\
    &=\lim_{\Delta y\to0}\frac{f(x+i(y+\Delta y))-f(x+iy)}{i\Delta y}=-if_y(z)
\end{align*}
が成り立つ(書籍のp.42も参照)。

よって、
\begin{align*}
    f_xdx\times f_ydy
    =&|f'(z)|^2dxdy
\end{align*}
であるから、求める像の面積は
\begin{align*}
    &\int_{|z|\le r}|f'(z)|^2dxdy\\
    =&\int_{0}^{r}\int_{0}^{2\pi}\left(\sum_{n=1}^{\infty}nc_nz^{n-1}\right)\left(\sum_{m=1}^{\infty}m\overline{c_mz^{m-1}}\right)rdrd\theta\\
    =&\sum_{n,m}nmc_n\overline{c_m}\int_{0}^{r}
    \int_{0}^{2\pi}r^{n+m-2}e^{i(n-m)\theta}rdrd\theta\\
    =&\sum_{n,m}nmc_n\overline{c_m}\int_{0}^{r}
    r^{n+m-2}2\pi\delta_{nm}rdr\\
    =&2\pi\sum_{n}n^2|c_n|^2\int_{0}^{r}
    r^{2n-1}dr\\
    =&2\pi\sum_{n}n^2|c_n|^2\left[\frac{r^{2n}}{2n}\right]_0^r\\
    =&2\pi\sum_{n}n^2|c_n|^2\frac{r^{2n}}{2n}\\
    =&\pi\sum_{n}n|c_n|^2r^{2n}
\end{align*}
(証明終)

\paragraph{疑問}
問3で現れた、
\begin{align*}
    w=f(z)=\frac{z-z_+}{z-z_-}
\end{align*}
も単葉函数である。

$|z|\le1$を$f$で写すと、像は原点を中心とする半径$-z_+$の円であったから、
\begin{align*}
    \int_{|z|\le 1}|f'(z)|^2dxdy=\pi z_+^2
\end{align*}
となると思うが、計算できるだろうか?