\chapter{有理型函数の展開}%第12章

\begin{mysimplebox}{問1}
    $z=n$($n=1,2,\dots$)において主要部が
    \begin{align*}
        \sum_{\nu=1}^{\infty}\frac{1}{\nu!(z-n)^\nu}
    \end{align*}
    であるような特異点をもち、その他の点では正則であるような函数を求めよ。
\end{mysimplebox}
\paragraph{解答}
\begin{align*}
    \sum_{\nu=1}^{\infty}\frac{1}{\nu!(z-n)^\nu}=e^{\frac{1}{z-n}}-1
\end{align*}
であるから、求める函数は
\begin{align*}
    \sum_{n=1}^{\infty}(e^{\frac{1}{z-n}}-1)
\end{align*}
である。これが収束することを見ておく。

\begin{align*}
    f(w)=\sum_{n=1}^{\infty}(e^{\frac{1}{z-n}}-1)w^n
\end{align*}
とする。
\begin{align*}
    \frac{e^{\frac{1}{z-n}}-1}{e^{\frac{1}{z-(n+1)}}-1}
    =
\end{align*}

\begin{enumerate}
    \item 本のp.210の(74.1)式の具体例に対して、Mittag--Lefflerの定理で存在が保証された函数$\phi(z)$を求める問題。
    \item $\phi(z)$は$\C$から$\{1,2,\dots\}$を除いた領域で正則。
    \item p.213の(74.5)式の$g(z)\equiv0$、$g_n(z)\equiv0$の場合か?
    \item この問題では何をすればいいのかが分からない。
\end{enumerate}

\newpage
\begin{mysimplebox}{問2}
    \begin{align*}
        \frac{\pi}{\sin\pi z}=\frac1z
        +\sum_{n=1}^{\infty}(-1)^n\left(\frac{1}{z-n}+\frac1n\right)
        +\sum_{n=1}^{\infty}(-1)^n\left(\frac{1}{z+n}-\frac1n\right)
    \end{align*}
\end{mysimplebox}
\paragraph{証明}
$\dfrac{\pi}{\sin\pi z}$は$z=n\in\Z$において一位の極をもつ。
\begin{align*}
    (z-n)\frac{\pi}{\sin\pi z}
    &=t\frac{\pi}{\sin\pi(t+n)}
    =\frac{\pi t}{\sin\pi t\cos\pi n+\cos\pi t\sin\pi n}\\
    &=(-1)^n \frac{\pi t}{\sin\pi t}
    \longrightarrow (-1)^n
\end{align*} 
よって、$z=n\in\Z$における留数は$(-1)^n$である。

$x=\pm (n+\frac{1}{2}), y=\pm (n+\frac{1}{2})$で囲まれた正方形の周を反時計回りに通る積分路を$C$とすると、$\zeta\notin\Z, |\zeta|<|n|+\frac{1}{2}$のとき
\begin{align*}
    \frac{1}{2\pi i}\int_{C}\frac{\pi}{(z-\zeta)\sin\pi z}dz
    =\frac{\pi}{\sin\pi \zeta}
     +\sum_{k=-n}^{n}\frac{(-1)^k}{k-\zeta}
\end{align*}
積分路の縦辺上では
\begin{align*}
    z&=\pm (n+\frac{1}{2})+yi\\
    \frac{\pi}{\sin\pi z}
    &=\frac{\pi}{\sin\pi(n+1/2+yi)}
    =\frac{(-1)^n\pi}{\cos\pi yi}
    =\frac{(-1)^n\pi}{e^{-\pi y}+e^{\pi y}}\\
    \left|\frac{\pi}{\sin\pi z}\right|&\le\frac{\pi}{2}
\end{align*}
横辺上では
\begin{align*}
    z&=x+\pm i(n+\frac{1}{2})\\
    \frac{\pi}{\sin\pi z}
    &=\frac{\pi}{\sin\pi(x+i(n+1/2))}
    =\frac{2i\pi}{e^{i\pi x}e^{-(n+1/2)}-e^{-i\pi x}e^{n+1/2}}\\
    \left|\frac{\pi}{\sin\pi z}\right|&\longrightarrow 0
\end{align*}
よって、
\begin{align*}
    &\left|\int_{C}\frac{\pi}{(z-\zeta)\sin\pi z}dz\right|\\
    &\le \left|\int_{C}\frac{\pi}{z\sin\pi z}dz\right|
    +\left|\zeta\int_{C}\frac{\pi}{z(z-\zeta)\sin\pi z}dz\right|\\
    &\le 0+\frac{\pi}{2}\cdot\frac{1}{(|n|+1/2)(|n|+1/2-|\zeta|)}\int_{C}ds\\
    &=\frac{\pi}{2}\cdot\frac{4(2|n|+1)}{(|n|+1/2)(|n|+1/2-|\zeta|)}\longrightarrow0
\end{align*}
したがって
\begin{align*}
    \frac{\pi}{\sin\pi \zeta}
     &=-\sum_{k=-\infty}^{\infty}\frac{(-1)^k}{k-\zeta}
    =\sum_{k=-\infty}^{\infty}\frac{(-1)^k}{\zeta-k}
\end{align*}
(終)



\newpage
\begin{mysimplebox}{問3}
    \begin{align*}
        \frac{\pi^2}{\sin^2\pi z}=\sum_{n=-\infty}^{\infty}\frac{1}{(z-n)^2}
    \end{align*}
\end{mysimplebox}
\paragraph{証明}
吉田先生の本のp.223の式より、
\begin{align*}
    \pi\cot \pi z=\frac{1}{z}+\sum_{n=1}^{\infty}
    \left(\frac{1}{z-n}+\frac{1}{z+n}\right)
\end{align*}

右辺は広義一様収束する(nの2乗の逆数だから)からWeierstrassの二重級数定理より、項別微分可能。

左辺の微分は$\dfrac{-\pi^2}{\sin^2\pi z}$(終)


\newpage
\begin{mysimplebox}{問4}
    \begin{align*}
        \cos\pi z=\prod_{n=1}^{\infty}\left(1-\frac{4z^2}{(2n-1)^2}\right)
    \end{align*}
\end{mysimplebox}
\paragraph{証明}
$\cos\pi z$の零点は$\dfrac{2n-1}{2}\pi$($n\in\Z$)。

p.218の(76.8)式を利用する。

ここで、位数$\rho$はp.102より、
\begin{align*}
    \rho=\overline{\lim_{r\to+\infty}}\frac{\log\log M(r)}{\log r}
\end{align*}
\begin{align*}
    &|\cos\pi z|=\left|\frac{e^{i\pi z}+e^{-i\pi z}}{2}\right|
    \le e^{\pi r}\\
    &\frac{\log\log|\cos\pi z|}{\log r}
    \le\frac{\log\log e^{\pi r}}{\log r}
    =\frac{\log \pi+\log r}{\log r}\longrightarrow1
\end{align*}
よって、$\cos\pi z$の位数$\rho=1$である。

ゆえに、(76.8)式において$q=1$としてよい。
また、$g(z)\equiv\alpha$(定数)、$k=0$。

したがって、
\begin{align*}
    \cos\pi z
    &=e^\alpha\prod_{n=-\infty}^{\infty}\left(1-\frac{2z}{2n-1}\exp\left(\frac{2z}{2n-1}\right)\right)\\
    &=e^\alpha\prod_{n=1}^{\infty}\left\{1-\left(\frac{2z}{2n-1}\right)^2\right\}
\end{align*}
$\cos\pi0=1$から$e^\alpha=1$が分かる。(終)

\paragraph{別解1}
\begin{align*}
    \frac{\prod_{n=1}^{\infty}\left\{1-\frac{4z^2}{(2n-1)^2}\right\}}{\cos\pi z}
\end{align*}
は全平面において正則であるから定数。$z=0$を代入すると1。(終)

\paragraph{別解2}
解析概論(最新の定本版)p.254の(23)式を利用する(吉田先生の本のp.223の式も同じ)。
\begin{align*}
    \tan z=-\sum_{n=0}^{\infty}
    \left\{\frac{1}{z-(n+1/2)\pi}-\frac{1}{z+(n+1/2)\pi}\right\}
\end{align*}
さらに、
\begin{align*}
    \frac{d}{dz}\log\cos z=-\tan z
\end{align*}
0から$z$($0<z<\pi/2$)まで積分する。
\begin{align*}
    \log\cos z&=\sum_{n=0}^{\infty}(\log(z-(n+1/2)\pi)-\log(-(n+1/2)\pi)\\
    &+\log(z+(n+1/2)\pi)-\log((n+1/2)\pi))\\
    &=\sum_{n=0}^{\infty}(\log(1-\frac{z}{(n+1/2)\pi})+\log(1+\frac{z}{(n+1/2)\pi}))\\
    &=\sum_{n=0}^{\infty}\log(1-\frac{4z^2}{(2n+1)^2\pi^2})\\
    &=\log\prod_{n=0}^{\infty}(1-\frac{4z^2}{(2n+1)^2\pi^2})\\
    \cos z&=\prod_{n=0}^{\infty}(1-\frac{4z^2}{(2n+1)^2\pi^2})
\end{align*}
(終)