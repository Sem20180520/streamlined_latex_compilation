\chapter{一次函数}%第8章

\begin{mysimplebox}{問1}
    一次函数$\frac{az+b}{cz+d}$($a,b,c,d\in\C,ad-bc\neq0$)
    は3つの一次函数
    $z+h$($h\neq0$)、
    $\alpha z$($\alpha\neq0$)、
    $\frac{1}{z}$
    を合成したものと考えられることを示せ。
\end{mysimplebox}
\paragraph{解答}
平行移動、定数倍、逆数をとる、再び平行移動の順に合成すると、
\begin{align*}
    z
    \mapsto z+h
    \mapsto \alpha(z+h)=\alpha z+\alpha h
    \mapsto \frac{1}{\alpha z+\alpha h}
    \mapsto \frac{1}{\alpha z+\alpha h}+h'
    =\frac{\alpha h'z+\alpha hh'+1}{\alpha z+\alpha h}
\end{align*}
となる。行列で表すと、
\begin{align*}
    \begin{bmatrix}
        1&h'\\
        0&1
    \end{bmatrix}
    \begin{bmatrix}
        0&1\\
        1&0
    \end{bmatrix}
    \begin{bmatrix}
        \alpha&0\\
        0&1
    \end{bmatrix}
    \begin{bmatrix}
        1&h\\
        0&1
    \end{bmatrix}
    =\begin{bmatrix}
        \alpha h'&\alpha hh'+1\\
        \alpha&\alpha h
    \end{bmatrix}
\end{align*}
である。この左辺の行列がすべて正則であるから、右辺の行列も正則である。

$c=0$のとき、ある$k\in\C,k\neq0$によって、
\begin{align*}
    a&=k\alpha h'\\
    b&=k(\alpha hh'+1)\\
    c&=k\alpha\\
    d&=k\alpha h
\end{align*}
と書けることを示せばよい。

$h=d/c$、$h'=a/c$がすぐにわかる。

また、$b=\frac{ad}{c}+k$であるから、
\begin{align*}
    k=b-\frac{ad}{c}=-\frac{ad-bc}{c}
\end{align*}
である。
よって、$\alpha=c/k=-(ad-bc)^{-1}$とすればよい。

次に、$c=0$のとき、$\frac{az+b}{cz+d}=(a/d)z+(b/d)$であるから、
\begin{align*}
    z\mapsto \frac{a}{d}z\mapsto \frac{a}{d}z+\frac{b}{d}
\end{align*}
とすればよい。(終)
\paragraph{補足}
$c\neq0$のとき、
\begin{align*}
    \frac{az+b}{cz+d}
    =\frac{\frac{a}{c}z+\frac{b}{c}}{z+\frac{d}{c}}
    =\frac{\frac{a}{c}\left(z+\frac{d}{c}\right)-\frac{ad}{c^2}+\frac{b}{c}}{z+\frac{d}{c}}
    =\frac{a}{c}-\frac{ad-bc}{c^2}\cdot\frac{1}{z+\frac{d}{c}}
\end{align*}
であるから、$\frac{d}{c}$だけ平行移動する、逆数をとる、$-\frac{ad-bc}{c^2}$倍する、$\frac{a}{c}$だけ平行移動する、という操作で$\frac{az+b}{cz+d}$という一次函数が得られる。

$c=0$のとき、
\begin{align*}
    \frac{az+b}{d}
    =\frac{a}{d}z+\frac{b}{d}
\end{align*}
であるから、$\frac{a}{d}$倍する、$\frac{b}{d}$だけ平行移動する、という操作で$\frac{az+b}{d}$という一次函数が得られる。または、$a$倍する、$b$だけ平行移動する、$\frac{1}{d}$倍する、という操作でもよい。これは、
\begin{align*}
    \begin{bmatrix}
        1&b/d\\0&1
    \end{bmatrix}
    \begin{bmatrix}
        a/d&0\\0&1
    \end{bmatrix}
    =
    \begin{bmatrix}
        1/d&0\\0&1
    \end{bmatrix}
    \begin{bmatrix}
        1&b\\0&1
    \end{bmatrix}
    \begin{bmatrix}
        a&0\\0&1
    \end{bmatrix}
    =
    \begin{bmatrix}
        a/d&b/d\\0&1
    \end{bmatrix}
\end{align*}
という行列計算が対応する。


\newpage
\begin{mysimplebox}{問2}
    一次函数は円(または直線)に関し偶数回反転することによってできたものと考えられることを示せ。
\end{mysimplebox}
%\paragraph{解答}
%書籍p.155で説明されているように、点$c$を中心とする半径$r$の円に関する反転は、
%\begin{align*}
%    z'=c+\frac{r^2}{\overline{z}-\overline{c}}
%\end{align*}
%と書ける。ここで、$z,z'$は互いの反転である。
%
%この式は、
%\begin{align*}
%    z'=\frac{c\overline{z}-|c|^2+r^2}{\overline{z}-\overline{c}}
%    =\overline{\left(\frac{\overline{c}z-|c|^2+r^2}{z-c}\right)}
%\end{align*}
%と書ける。
%
%さらに、点$c'$を中心とする半径$r'$の円に関する反転を続けて行うと、
%\begin{align*}
%    z''=\overline{\left(\frac{\overline{c'}z'-|c'|^2+r'^2}{z'-c'}\right)}
%    =\frac{c'\overline{z'}-|c'|^2+r'^2}{\overline{z'}-\overline{c'}}
%\end{align*}
%である。これは、
%\begin{align*}
%    \begin{bmatrix}
%        c'&-|c'|^2+r'^2\\
%        1&-\overline{c'}
%    \end{bmatrix}
%    \begin{bmatrix}
%        \overline{c}&-|c|^2+r^2\\
%        1&-c
%    \end{bmatrix}
%\end{align*}
%に対応する一次函数である。
%実際、この行列の行列式は、
%\begin{align*}
%    (-|c'|^2+|c'|^2-r'^2)(-|c|^2+|c|^2-r^2)=r^2r'^2\neq0
%\end{align*}
%である。(終)

\paragraph{解答}
前問から、一次函数は平行移動($z+\alpha$)、定数倍($\alpha z$)、逆数($\dfrac{1}{z}$)をとる、という操作で生成されることが分かった。
よって、円か直線に関する反転を偶数回行うことにより、平行移動、定数倍、逆数をとる、という操作を実現できればよい。

書籍p.155で説明されているように、点$c$を中心とする半径$r$の円に関する反転は、
\begin{align*}
    z'=c+\frac{r^2}{\overline{z}-\overline{c}}
\end{align*}
と書ける。ここで、$z,z'$は互いの反転である。

点$b$を通り、傾き$a$の直線に関する反転は
\begin{align*}
    \overline{\left(\frac{z-b}{a}\right)}
    &=\frac{z'-b}{a}\\
    z'&=a\overline{\left(\frac{z-b}{a}\right)}+b
\end{align*}
と書ける。
\begin{figure}[h]
    \centering
    \includegraphics[width=8cm]{chap8_fig/chap8_fig002.png}
    \caption{直線に関する反転}
    \label{fig:chap8_line}
\end{figure}

\newpage
まず、平行移動($z+\alpha$)について考える。
原点が$\alpha$まで移動することから、原点と$\alpha$を結ぶ線分$\mathrm{O}\alpha$の垂直二等分線で反転をし、
その後に、$\alpha$を通り線分$\mathrm{O}\alpha$に垂直な直線で反転をすればよい。
実際、
\begin{align*}
    \overline{\left(\frac{z-\alpha/2}{i\alpha}\right)}
    &=\frac{z'-\alpha/2}{i\alpha}\\
    \overline{\left(\frac{z'-\alpha}{i\alpha}\right)}
    &=\frac{z''-\alpha}{i\alpha}
\end{align*}
から、
\begin{align*}
    \frac{z''-\alpha}{i\alpha}
    &=\overline{\left(\frac{z'-\alpha/2-\alpha/2}{i\alpha}\right)}
    =\frac{z-\alpha/2}{i\alpha}+\frac{\overline{\alpha}/2}{i\overline{\alpha}}
    =\frac{z}{i\alpha}\\
    z''&=z+\alpha
\end{align*}
が得られる。
\begin{figure}[h]
    \centering
    \includegraphics[width=8cm]{chap8_fig/chap8_fig003.png}
    \caption{平行移動を実現する直線反転}
    \label{fig:chap8_parallel}
\end{figure}

\newpage
次に、定数倍($\alpha z$)について考える。
$1$が$\alpha$まで移ることから、原点と$\sqrt{\alpha}$を通る直線で反転をしてから、
原点と$\alpha$を通る直線で反転をしてみる。
\begin{align*}
    \overline{\left(\frac{z}{\sqrt{\alpha}}\right)}
    &=\frac{z'}{\sqrt{\alpha}}\\
    \overline{\left(\frac{z'}{\alpha}\right)}
    &=\frac{z''}{\alpha}
\end{align*}
から、
\begin{align*}
    z''&=\alpha\overline{\left(\frac{z'}{\alpha}\right)}
    =\alpha\overline{\left(\frac{1}{\sqrt{\alpha}}\cdot\frac{z'}{\sqrt{\alpha}}\right)}
    =\alpha\frac{1}{\overline{\sqrt{\alpha}}}\cdot\frac{z}{\sqrt{\alpha}}
    =\frac{\alpha}{|\alpha|}z
\end{align*}
が得られる。

さらに、原点を中心とする半径$1$、$\sqrt{|\alpha|}$の円に関する反転を続けて行うと
\begin{align*}
    z'''&=\frac{\sqrt{|\alpha|}^2}{\overline{\frac{1}{\overline{z''}}}}
    =|\alpha|z''
    =|\alpha|\frac{\alpha}{|\alpha|}z
    =\alpha z
\end{align*}
が得られる。
\begin{figure}[h]
    \centering
    \includegraphics[width=10cm]{chap8_fig/chap8_fig004.png}
    \caption{定数倍を実現する直線反転と円反転}
    \label{fig:chap8_mult}
\end{figure}

\newpage
最後に、逆数($\dfrac{1}{z}$)について考える。
実軸に関する反転、すなわち複素共役をとってから、原点を中心とする半径1の円に関する反転を行えばよい。操作の順番は逆でもよい。
よって、
\begin{align*}
    z'&=\overline{z}\\
    z''&=\frac{1}{\overline{z'}}=\frac{1}{z}
\end{align*}
が得られる。
\begin{figure}[h]
    \centering
    \includegraphics[width=10cm]{chap8_fig/chap8_fig005.png}
    \caption{逆数を実現する円反転}
    \label{fig:chap8_inverse}
\end{figure}
(解答終)



\newpage
\begin{mysimplebox}{問3}
    半平面$\Im(z)>0$を$|w|<1$に写像する一次函数は
    \begin{align*}
        w=\frac{az+b}{\overline{a}z+\overline{b}}
        \quad\left(\Im\left(\frac{b}{a}\right)<0\right)
    \end{align*}
    であることを示せ。
\end{mysimplebox}
\paragraph{証明}
\begin{align*}
    f(z)=\frac{az+b}{\overline{a}z+\overline{b}}
\end{align*}
とする。円円対応により、$f$は実軸を円または直線に写像する。
$x\in\R$とすると、
\begin{align*}
    \left|f(x)\right|
    =\left|\frac{ax+b}{\overline{a}x+\overline{b}}\right|
    =\left|\frac{ax+b}{\overline{(ax+b)}}\right|
    =1
\end{align*}
であるから、$f$は実軸を単位円に写像することがわかる。

条件から、$\Im\left(-\frac{b}{a}\right)>0$、
すなわち、$-\frac{b}{a}$は上半平面の点であるが、
\begin{align*}
    f\left(-\frac{b}{a}\right)
    =\frac{-a\frac{b}{a}+b}{-\overline{a}\frac{b}{a}+\overline{b}}
    =\frac{0}{-\overline{a}b+a\overline{b}}
    =0
\end{align*}
であるから、$f$は上半平面を単位円内に写像する。(証明終)

\newpage
\begin{mysimplebox}{問4}
    一次函数
    \begin{align*}
        f(z)=\frac{az+b}{cz+d}
        \quad\left(a,b,c,d\in\R,ad-bc>0\right)
    \end{align*}
    は半平面$\Im(z)>0$を半平面$\Im(w)>0$に写像することを示せ。
\end{mysimplebox}
\paragraph{証明}
$a\neq0,c\neq0$のとき、
\begin{align*}
    f(z)=&\frac{a}{c}\cdot\frac{cz+\frac{bc}{a}}{cz+d}
    =\frac{a}{c}\cdot\frac{cz+d-d+\frac{bc}{a}}{cz+d}
    =\frac{a}{c}\left(1-\frac{1}{a}\cdot\frac{ad-bc}{cz+d}\right)
\end{align*}
と書けるから、$f$は実軸を実軸に写像する。
\footnote{わざわざこの式変形をしなくても$f$が実軸を実軸に写像することは分かる。}

$a=0,c\neq0$のとき、
\begin{align*}
    f(z)=\frac{b}{cz+d}
\end{align*}
であるから、やはり$f$は実軸を実軸に写像する。

$a\neq0,c=0$ならば、
\begin{align*}
    f(z)=\frac{az+b}{d}
\end{align*}
であるから、やはり$f$は実軸を実軸に写像する。

\begin{align*}
    f(i)&=\frac{ai+b}{ci+d}
    =\frac{(ai+b)(-ci+d)}{(ci+d)(-ci+d)}
    =\frac{ac+bd+(ad-bc)i}{c^2+d^2}
\end{align*}
であり、$ad-bc>0$であるから、$\Im f(i)>0$である。
よって、$f$は半平面$\Im(z)>0$を半平面$\Im(w)>0$に写像する。(証明終)

\newpage
\begin{mysimplebox}{問5}
    $z=f(z)$なる$z$を函数$f(z)$の不動点という。
    2個より多くの不動点をもつ一次函数は$z$以外にはないことを示せ。
\end{mysimplebox}
\paragraph{証明}
\begin{align*}
    f(z)=\frac{az+b}{cz+d}
\end{align*}
とする。ただし、$ad-bc\neq0$である。

$f(z)=z$から、
\begin{align}
    z&=\frac{az+b}{cz+d}\nonumber\\
    cz^2+dz&=az+b\nonumber\\
    cz^2+(d-a)z-b&=0\label{eq:ch8-4-1}
\end{align}
である。
\begin{itemize}
    \item $c\neq0$ならば、方程式$(\ref{eq:ch8-4-1})$は、重複度も込めて2個の解をもつ。
    \item $c=0,d-a\neq0$ならば、方程式$(\ref{eq:ch8-4-1})$は、1個の解をもつ。このとき、$f(z)=(az+b)/d$である。不動点は$b/(d-a)$である。$\infty$も不動点であると考える。
    \item $c=0,d-a=0,b\neq0$ならば、方程式$(\ref{eq:ch8-4-1})$は、解をもたない。このとき、$f(z)=z+b/a$である。$\infty$が不動点であると考える。
    \item $c=0,d-a=0,b=0$ならば、方程式$(\ref{eq:ch8-4-1})$は、$\C$全体が解である。このとき、$f(z)\equiv z$である。
\end{itemize}
以上から、不動点を2個より多くもつ一次函数は$f(z)\equiv z$のみである。(証明終)

\newpage
\begin{mysimplebox}{問6}
    $w=\dfrac{az+b}{cz+d}$
    は$\dfrac{w-\alpha}{w-\beta}=k\dfrac{z-\alpha}{z-\beta}$
    または
    $\dfrac{1}{w-\alpha}=\dfrac{1}{z-\alpha}+h$
    なる形に書き直すことができることを示せ。
    ただし、$\alpha,\beta,k,h$は定数である。
\end{mysimplebox}
\paragraph{証明}
\begin{align*}
    \begin{vmatrix}
        a&b\\
        c&d
    \end{vmatrix}
    =ad-bc=1
\end{align*}
としてよい。

前問の場合分けで順に見ていく。

$c\neq0$で、方程式$(\ref{eq:ch8-4-1})$が重根をもたないとき、
判別式$D$とすると、
\begin{align*}
    D=(d-a)^2+4bc
    =(d-a)^2+4(ad-1)
    =(a+d)^2-4\neq0
\end{align*}
である。

方程式$(\ref{eq:ch8-4-1})$の異なる2個の解を、$\alpha,\beta$とする。
\begin{align*}
    \alpha+\beta=&\frac{a-d}{c}\\
    \alpha\beta=&-\frac{b}{c}
\end{align*}
が成り立つ。

$\dfrac{z-\alpha}{z-\beta}$と$\dfrac{w-\alpha}{w-\beta}$の関係を考える。これは、次のような行列計算を見ればよい。
\begin{align*}
    \begin{bmatrix}
        1&-\alpha\\
        1&-\beta
    \end{bmatrix}
    \begin{bmatrix}
        a&b\\
        c&d
    \end{bmatrix}
    \begin{bmatrix}
        1&-\alpha\\
        1&-\beta
    \end{bmatrix}^{-1}
\end{align*}
計算を行うと、次の行列が得られる。
\begin{align*}
    \frac{1}{\alpha-\beta}
    \begin{bmatrix}
        d\alpha-a\beta-2b&0\\
        0&a\alpha-d\beta+2b
    \end{bmatrix}
\end{align*}
なお、
\begin{align*}
    \begin{vmatrix}
        1&-\alpha\\
        1&-\beta
    \end{vmatrix}
    =\alpha-\beta\neq0
\end{align*}
である。

以上で、
\begin{align*}
    \frac{w-\alpha}{w-\beta}
    =k\frac{z-\alpha}{z-\beta},\quad
    k=\frac{d\alpha-a\beta-2b}{a\alpha-d\beta+2b}
\end{align*}
であることが分かる。
\footnote{$k$は行列の固有値の比であることも分かった。}

次に、$D=(a+d)^2-4=0$のとき、重根を$\alpha=\dfrac{a-d}{2c}$とする。$\dfrac{1}{z-\alpha}$と$\dfrac{1}{w-\alpha}$の関係を考える。これは、次のような行列計算を見ればよい。
\begin{align*}
    \begin{bmatrix}
        0&1\\
        1&-\alpha
    \end{bmatrix}
    \begin{bmatrix}
        a&b\\
        c&d
    \end{bmatrix}
    \begin{bmatrix}
        0&1\\
        1&-\alpha
    \end{bmatrix}^{-1}
\end{align*}
計算を行うと、次の行列が得られる。
\begin{align*}
    \begin{bmatrix}
        (a+d)/2&c\\
        0&(a+d)/2
    \end{bmatrix}
\end{align*}
よって、
\begin{align*}
    \frac{1}{w-\alpha}
    =\frac{\{(a+d)/2\}\frac{1}{z-\alpha}+c}{\{(a+d)/2\}}
    =\frac{1}{z-\alpha}+\frac{2c}{a+d}
\end{align*}
であることが分かる。

不動点が$\alpha:=\dfrac{b}{d-a}$と$\infty$のとき($d-a\neq0,c=0$)、
\begin{align*}
    w-\alpha
    =\frac{az+b}{d}-\frac{b}{d-a}
    =\frac{(az+b)(d-a)-bd}{d(d-a)}
    =\frac{a(d-a)z-ab}{d(d-a)}
    =\frac{a}{d}(z-\alpha)
\end{align*}
となり、$\dfrac{w-\alpha}{w-\beta}=k\dfrac{z-\alpha}{z-\beta}$の亜種と思うことができる(追記参照)。

不動点が$\infty$のみの場合、$w=f(z)=z+h$($h\neq0$)
と書ける。$z=\dfrac{1}{\zeta}$、$w=\dfrac{1}{\xi}$とすると、$\dfrac{1}{\xi}=\dfrac{1}{\zeta}+h$であるから、
$\dfrac{1}{w-\alpha}=\dfrac{1}{z-\alpha}+h$の
$\alpha=0,h\neq0$の場合と思うことができる。

$f(z)=\dfrac{az+b}{cz+d}$
の不動点が2個より多いとき、$w=f(z)\equiv z$である。
これは
$\dfrac{1}{w-\alpha}=\dfrac{1}{z-\alpha}+h$
で$\alpha=0,h=0$の場合である。(証明終)

\paragraph{追記}
亜種と書いたタイプについては、その下のタイプと同様に$z=\dfrac{1}{\zeta}$、$w=\dfrac{1}{\xi}$としてみると、
\begin{align*}
    \xi
    =\frac{1}{w}
    =\frac{d}{az+b}
    =\frac{d/z}{a+b/z}
    =\frac{d\zeta}{b\zeta+a}
\end{align*}
であり、$\zeta=\dfrac{1}{\alpha},0$が不動点である。$\alpha=0$のときの扱いが面倒である。

その代わり、$z$のみ$z=\dfrac{1}{\zeta}$と変換し、$w$は$w$のままとすると、
\begin{align*}
    w=\frac{az+b}{d}=\frac{a+b/z}{d/z}=\frac{b\zeta+a}{d\zeta}
\end{align*}
である。不動点は、$\dfrac{b\zeta+a}{d\zeta}=\zeta$より、
$d\zeta^2-b\zeta-a=0$の解である。
$d\neq0$であるから重複度も込めて2個の不動点を持つことが分かる。

異なる2個の不動点を持つときは、上記の議論と同じように、不動点を$\alpha,\beta$とすると、$\dfrac{w-\alpha}{w-\beta}=k\dfrac{\zeta-\alpha}{\zeta-\beta}$の形にできる。次の行列計算をすればよい。
\begin{align*}
    \begin{bmatrix}
        1&-\alpha\\1&-\beta
    \end{bmatrix}
    \begin{bmatrix}
        a&b\\0&d
    \end{bmatrix}
    \begin{bmatrix}
        0&1\\1&0
    \end{bmatrix}^{-1}
    \begin{bmatrix}
        1&-\alpha\\1&-\beta
    \end{bmatrix}^{-1}
    =\frac{1}{\alpha-\beta}
    \begin{bmatrix}
        -2a-\beta b&0\\0&2a+\alpha b
    \end{bmatrix}
\end{align*}
これにより、$k=\dfrac{-2a-\beta b}{2a+\alpha b}$も分かる。
片方だけ逆数にすると不動点が変わるため、有限の異なる2個の不動点の場合に帰着できるという話である。$w$を$w=\dfrac{1}{\xi}$とし、$z$は$z$のままとしても同様の結果が得られる。不動点が1個のときも同様なので省略する。

\newpage
\begin{mysimplebox}{問7}
    $f(z)$が$|z|<1$で正則で$|f(z)|<M$ならば
    \begin{align*}
        \frac{M(|f(0)|-Mr)}{M-r|f(0)|}
        \le|f(z)|
        \le\frac{M(|f(0)|+Mr)}{M+r|f(0)|}
        \quad(|z|=r<1).
    \end{align*}
    ヒント: $\dfrac{M(f(0)-f(z))}{M^2-\overline{f(0)}f(z)}$にSchwarzの定理を応用する。
    または、$\dfrac{f(z)}{M}$を$f(z)$と取り直し、$\dfrac{f(0)-f(z)}{1-\overline{f(0)}f(z)}$にSchwarzの定理を応用する。
\end{mysimplebox}
\paragraph{証明}
ヒントの後半にあるように、$\dfrac{f(z)}{M}$を$f(z)$と取り直し、$g(z)=\dfrac{f(0)-f(z)}{1-\overline{f(0)}f(z)}$にSchwarzの定理を応用する。
まずは、Schwarzの定理(p.102)を適用できることを確認する。

$|z|<1$において$|f(z)|<1$であるから、$g(z)$は$|z|<1$において正則である。
また、$g(0)=0$である。さらに、
\begin{align*}
    &|1-\overline{f(0)}f(z)|^2-|f(0)-f(z)|^2\\
    =&(1-\overline{f(0)}f(z))(1-f(0)\overline{f(z)})-(f(0)-f(z))(\overline{f(0)}-\overline{f(z)})\\
    =&1-\overline{f(0)}f(z)-f(0)\overline{f(z)}+|f(0)|^2|f(z)|^2-(|f(0)|^2-f(z)\overline{f(0)}-f(0)\overline{f(z)}+|f(z)|^2)\\
    =&1-|f(0)|^2-|f(z)|^2+|f(0)|^2|f(z)|^2\\
    =&(1-|f(0)|^2)(1-|f(z)|^2)>0
\end{align*}
である。よって、$|g(z)|<1$である。

以上から、$g(z)$にSchwarzの定理を適用できることが分かった。次の不等式が得られる。
\begin{align*}
    |g(z)|&\le|z|\\
    \left|\frac{f(0)-f(z)}{1-\overline{f(0)}f(z)}\right|&\le|z|
\end{align*}

ここで、
\begin{align*}
    \zeta=\frac{c-w}{1-\overline{c}w}
\end{align*}
を考える。
ただし、$|c|<1$である。$c=f(0)$、$w=f(z)$とすれば$\zeta=g(z)$である。

\begin{align*}
    \det\begin{bmatrix}
        -1&c\\-\overline{c}&1
    \end{bmatrix}
    =-1+|c|^2<0
\end{align*}
であるから、$\zeta$は一次函数である。

$|\zeta|=r(<1)$とすると、
\begin{align*}
    r&=\left|\frac{c-w}{1-\overline{c}w}\right|\\
    r&=\left|\frac{c-w}{\overline{c}(1/\overline{c}-w)}\right|\\
    r|\overline{c}|=r|c|&=\left|\frac{c-w}{1/\overline{c}-w}\right|
\end{align*}
となる。なお、$c=0$、すなわち$f(0)=0$のときは、$f(z)$にSchwarzの定理を適用すれば求める不等式が得られる。ゆえに、以下では$c\neq0$とする。
よって、$w$は$1/\overline{c}$と$c$からの距離の比が
$1:r|c|$となる。すなわち、アポロニウスの円上を動く。
ゆえに、$|\zeta|\le r$のとき、$w$はこのアポロニウスの円による閉円板上に存在する(円周を含む)。
\begin{figure}[h]
    \centering
    \includegraphics[width=15cm]{chap8_fig/chap8_fig001.png}
    \caption{アポロニウスの円}
    \label{fig:chap8_apo}
\end{figure}

実際に円の方程式を求めると、
\begin{align*}
    \left|w-\frac{1-r^2}{1-r^2|c|^2}c\right|
    =\frac{(1-|c|^2)r}{1-r^2|c|^2}
\end{align*}
である(計算は省略する)。

このアポロニウスの円上で$w=0$からの距離が最小となる点は、$c$と$\dfrac{1}{\overline{c}}$を結ぶ線分を$r|c|:1$の比に外分するから、
\begin{align*}
    &\frac{-r|c|\frac{1}{\overline{c}}+c}{1-r|c|}
\end{align*}
である。原点と$c$と$\dfrac{1}{\overline{c}}$が半直線上にあることから、
\begin{align*}
    &\left|\frac{-r|c|\frac{1}{\overline{c}}+c}{1-r|c|}\right|
    =\frac{\left||c|-r\right|}{1+r|c|}
\end{align*}
である。

このアポロニウスの円上で$w=0$からの距離が最大となる点は、$c$と$\dfrac{1}{\overline{c}}$を結ぶ線分を$r|c|:1$の比に内分するから、
\begin{align*}
    \frac{r|c|\frac{1}{\overline{c}}+c}{1+r|c|}
\end{align*}
である。
上記と同様に、
\begin{align*}
    \left|\frac{r|c|\frac{1}{\overline{c}}+c}{1+r|c|}\right|
    =\frac{|c|+r}{1+r|c|} 
\end{align*}
である。

よって、
\begin{align*}
    \frac{|c|-r}{1-r|c|}
    \le|w|
    \le\frac{|c|+r}{1+r|c|}
\end{align*}
である。\footnote{この不等式の左側は$|c|-r<0$のときも成り立つので、絶対値を外した。}
もとの問題の記号で表すと、
\begin{align*}
    \frac{|f(0)|-r}{1-r|f(0)|}
    \le|f(z)|
    \le\frac{|f(0)|+r}{1+r|f(0)|}
\end{align*}
である。

さらに、$|z|<1$で$|f(z)|<M$のときは、$f(z)/M$が今示した不等式を満たすから、
\begin{align*}
    \frac{|f(0)|/M-r}{1-r|f(0)|/M}
    &\le|f(z)|/M
    \le\frac{|f(0)|/M+r}{1+r|f(0)|/M}\\
    \frac{M(|f(0)|-Mr)}{M-r|f(0)|}
    &\le|f(z)|
    \le\frac{M(|f(0)|+Mr)}{M+r|f(0)|}
\end{align*}
が得られる。(証明終)

\paragraph{別証明}
\begin{align*}
    \left|\frac{c-w}{1-\overline{c}w}\right|&\le r\\
    1-\left|\frac{c-w}{1-\overline{c}w}\right|^2&\ge 1-r^2\\
    |1-\overline{c}w|^2-|c-w|^2&\ge(1-r^2)|1-\overline{c}w|^2\\
    &\ge(1-r^2)(1-|c||w|)^2\\
    1-\overline{c}w-\overline{w}c+|c|^2|w|^2
    -(|c|^2-\overline{c}w-\overline{w}c+|w|^2)
    &\ge(1-r^2)(1-2|c||w|+|c|^2|w|^2)\\
    (1-r^2|c|^2)|w|^2-2(1-r^2)|c||w|+|c|^2-r^2&\le0\\
    \{(1+r|c|)|w|-(|c|+r)\}\{(1-r|c|)|w|-(|c|-r)\}&\le0\\
    \frac{|c|-r}{1-r|c|}&\le|w|\le\frac{|c|+r}{1+r|c|}
\end{align*}
(別証明終)

\newpage
\begin{mysimplebox}{問8}
    $|z_1|<1$,$|z_2|<1$のとき
    \begin{align*}
        D(z_1,z_2)=\log\frac{1+r}{1-r}\quad\left(r=\left|\frac{z_1-z_2}{1-\overline{z_1}z_2}\right|\right)
    \end{align*}
    とおくと、
    \begin{enumerate}
        \item $D(z_1,z_2)\ge0.$ $D(z_1,z_2)=0$であるのは$z_1=z_2$のときに限る。
        \item $D(z_1,z_2)=D(z_2,z_1).$
        \item $D(z_1.z_3)\le D(z_1,z_2)+D(z_2,z_3).$
    \end{enumerate}
    $D(z_z,z_2)$を$z_1,z_2$の非ユークリッド距離という。
\end{mysimplebox}
\paragraph{証明}
1.について。
$|c|<1$に対して、
\begin{align*}
    w&=f(z)=\frac{z-c}{\overline{c}z-1}
\end{align*}
とすると、$|z|=1$のとき$1/z=\overline{z}$であるから
\begin{align*}
    |w|&=\left|\frac{z-c}{\overline{c}z-1}\right|
    =\frac{|z-c|}{|\overline{c}z-1|}
    =\frac{|z-c|}{|z||\overline{c}-1/z|}
    =\frac{|z-c|}{|z||\overline{c}-\overline{z}|}
    =\frac{|z-c|}{|z||z-c|}=1
\end{align*}
である。一次変換には円円対応があるから、$w=f(z)$は単位円を単位円に写像する。p.16の問3も参照。

ゆえに、$r=\left|\frac{z_1-z_2}{1-\overline{z_1}z_2}\right|$とすると$|z_1|<1, |z_2|<1$に対して$0\le r<1$である。

\begin{align*}
    \frac{1+r}{1-r}
    =\frac{2}{1-r}-1
\end{align*}
は、$0\le r<1$において単調増加であり、1から$\infty$まで動く。

したがって、
\begin{align*}
    D(z_1,z_2)=\log\frac{1+r}{1-r}\ge0
\end{align*}
である。
等号が成り立つのは、$r=0$のとき、すなわち$z_1=z_2$のときである。

2.について。
\begin{align*}
    \left|\frac{z_1-z_2}{1-\overline{z_1}z_2}\right|
    =\left|\frac{z_2-z_1}{1-\overline{z_2}z_1}\right|
\end{align*}
であるから、$D(z_1,z_2)=D(z_2,z_1)$である。

\newpage
3.について。

\paragraph{$D(z,w)$が非調和比を使って定義できることを見る。}

単位円内の$z,w$について、この2点を通り単位円と直交する円がただ一つ存在する。

\begin{figure}[h]
    \centering
    \includegraphics[width=8cm]{chap8_fig/chap8_fig006.jpg}
    \caption{非調和比による距離の定義}
    \label{fig:chap8_dist}
\end{figure}

図$\ref{fig:chap8_dist}$のように、その円と単位円の交点を$a,b$とする。

p.152で定義した非調和比を改めて、
\begin{align*}
    [a,b,w,z]=\frac{a-w}{b-w}\Bigg/\frac{a-z}{b-z}
\end{align*}
と定義する。円周角の性質から$[a,b,w,z]\in\R_{\ge0}$である。

\begin{align*}
    D(z,w)=\left|\log[a,b,w,z]\right|
\end{align*}
と定義する。

ある$\theta\in\R$に対して、
\begin{align*}
    f(z)=e^{i\theta}\frac{z-w}{1-\overline{w}z}
\end{align*}
と定めると、$f(w)=0$であるから$w$は原点に移される。
原点と$f(z)$を通る直線は単位円と直交する(リーマン球面では直線は無限遠点を通る円と考える。p.153参照。)

よって、$\theta$を適当に選べば、$f(z)$が実軸の正の部分にあるようにできる。
$f(z)=r$($r<1$)とすると、一次函数$f$によって$[a,b,w,z]$は$[1,-1,0,r]$となるが、非調和比は一次変換によって不変であることから(p.152参照)、
\begin{align*}
    [a,b,w,z]=[1,-1,0,r]=\frac{1-0}{-1-0}\Bigg/\frac{1-r}{-1-r}=\frac{1+r}{1-r}
\end{align*}
である。
したがって、
\begin{align*}
    D(z,w)=\log\frac{1+r}{1-r}
\end{align*}
が成り立つ。

% さらに、一次函数によって単位円内$|z|<1$を上半平面$\{w\in\C\mid\Re w>0\}$に移すことができる(問3より)。よって、上半平面において三角不等式を示せばよい。

\paragraph{三角不等式を示す}
単位円内の$z_1,z_2,z_3$に対して、改めて
\begin{align*}
    f(z)=e^{i\theta}\frac{z-z_1}{1-\overline{z_1}z}
\end{align*}
とすると$f(z_1)=0$である。
さらに、$\theta$を適当に選ぶことで$f(z_3)$が実軸の正の部分にあるようにできる。

$f$は一次変換であるから非調和比を保ち、$D(f(z_i),f(z_j))=D(z_i,z_j)$
であるから、最初から$z_1=0$、$z_3=t$($0\le t<1$)としてよい。

$z_2=w$とすると、示すべき不等式は
\begin{align*}
    D(0,t)\le D(0,w)+D(w,t)
\end{align*}
である。

$w$から実軸に垂線を下ろすと交点は$\Re w$である。
$s:=\Re w$とする。

\begin{figure}[h]
    \centering
    \includegraphics[width=7cm]{chap8_fig/chap8_fig007.jpg}
    \caption{三角不等式の証明}
    \label{fig:chap8_tri}
\end{figure}

まず、
\begin{align*}
    D(0,t)\le D(0,s)+D(s,t)
\end{align*}
を示す。

原点と$s$と$t$は実軸上にあることから
\begin{align*}
    D(0,t)&=|\log[1,-1,0,t]|
    =\left|\log\frac{1-0}{-1-0}\Bigg/\frac{1-t}{-1-t}\right|
    =\left|\log\frac{1+t}{1-t}\right|\\
    &=\log\frac{1+t}{1-t}\\
    D(0,s)&=\left|\log\frac{1+s}{1-s}\right|=\log\frac{1+s}{1-s}\\
    D(s,t)&=|\log[1,-1,s,t]|
    =\left|\log\frac{1-s}{-1-s}\Bigg/\frac{1-t}{-1-t}\right|\\
    &=\left|\log\frac{1+s}{1-s}-\log\frac{1+t}{1-t}\right|
    \ge\left|D(0,s)-D(0,t)\right|
\end{align*}
よって、$D(0,t)\le D(0,s)+D(s,t)$が成り立つ。

次に、
\begin{align*}
    &D(0,s)\le D(0,w)\\
    &D(s,t)\le D(w,t)
\end{align*}
をそれぞれ示せば証明が完了する。

$D(0,s)\le D(0,w)$について。

一次変換である回転によって、$w$は実軸の正の部分に移すことができるから
\begin{align*}
    D(0,w)=\left|\log\frac{1+|w|}{1-|w|}\right|
\end{align*}
$|s|\le|w|$であり、$\log\frac{1+x}{1-x}$が$0\le x<1$において単調増加であることから、$D(0,s)\le D(0,w)$がわかる。


$D(s,t)\le D(w,t)$について。
\begin{align*}
    g(z)=\frac{z-t}{1-tz}
\end{align*}
という一次変換を定めると、$g(t)=0$である。
また、$g(s)\in\R$である。
一次変換が等角写像であることから$\Re g(w)=g(s)$である。
\begin{figure}[h]
    \centering
    \includegraphics[width=8cm]{chap8_fig/chap8_fig008.jpg}
    \caption{三角不等式の証明2}
    \label{fig:chap8_tri2}
\end{figure}

よって、$|g(s)|\le|g(w)|$である。ゆえに上記の議論と同様にして、
\begin{align*}
    D(s,t)&=D(g(s),g(t))=D(g(s),0)=\left|\log\frac{1+|g(s)|}{1-|g(s)|}\right|\\
    D(w,t)&=D(g(w),g(t))=D(g(w),0)=\left|\log\frac{1+|g(w)|}{1-|g(w)|}\right|
\end{align*}
から$D(s,t)\le D(w,t)$がわかる。(証明終)
\paragraph{訂正}
$|g(s)|\le|g(w)|$は,
$w=s+iy$として、次の2式から分かる。
\begin{align*}
    |g(s)|
    &=\left|\frac{s-t}{1-st}\right|\\
    |g(w)|
    &=\left|\frac{s+iy-t}{1-(s+iy)t}\right|
\end{align*}

\newpage
\begin{mysimplebox}{問9}
    $f(z)$が$|z|<1$で正則で$|f(z)|<1$ならば
    \begin{align*}
        D(f(z_1),f(z_2))\le D(z_1,z_2)
    \end{align*}
    が成り立つ。等号が成り立つのは、$f(z)$が$|z|<1$をそれ自身に写像する一次函数の場合に限る。
\end{mysimplebox}
\paragraph{証明}
% 問題の条件下で、次の定理が成り立つ。

% \paragraph{Schwarz--Pickの定理}
% $|z_1|<1, |z_2|<1$に対して
% \begin{align*}
%     \left|\frac{f(z_1)-f(z_2)}{1-\overline{f(z_1)}f(z_2)}\right|
%     \le\left|\frac{z_1-z_2}{1-\overline{z_1}z_2}\right|
% \end{align*}
% $z_1=z_2$以外で等号が成り立つのは、$\theta\in\R$(定数)に対して
% \begin{align*}
%     \frac{f(z_1)-f(z_2)}{1-\overline{f(z_1)}f(z_2)}=e^{i\theta}\frac{z_1-z_2}{1-\overline{z_1}z_2}
% \end{align*}
% のときに限る。

% この定理を認めれば、$\log\frac{1+r}{1-r}$は$0\le r<1$において単調増加であるから、
% $D(f(z_1),f(z_2))\le D(z_1,z_2)$が成り立つ。

% また、$z_1=z_2$以外で等号が成り立つときを考える。
% $z_2$を固定し$z_1$を変数と見ると、$e^{i\theta}\frac{z_1-z_2}{1-\overline{z_1}z_2}$は単位円を単位円に写像する。よって、$f(z)$は単位円を単位円に写像する。

% したがって、Schwarz--Pickの定理を証明すればよい。それは現在、検討中である。

$|\alpha|<1$を満たす$\alpha$をパラメータとする、$z\in\C$についての関数を
\begin{align*}
    \psi_\alpha(z)=\frac{z-\alpha}{1-\overline{\alpha}z}
\end{align*}
と定める。これは
\begin{align*}
    \begin{bmatrix}
        1&-\alpha\\
        -\overline{\alpha}&1
    \end{bmatrix}
\end{align*}
に対応する一次函数である。この一次函数が$|z|<1$を$|\sigma_\alpha(z)|<1$に移すことはp.156で示した。

ここで、
% \begin{align*}
%     \psi_\alpha(\alpha)=0
% \end{align*}
% であるから
\begin{align*}
    g=\psi_{f(\alpha)}\circ f\circ\psi_\alpha^{-1}
\end{align*}
とすると、$g(z)$は$|z|<1$で正則であり、$|g(z)|<1$、$\psi_\alpha(\alpha)=0$であるから、
\begin{align*}
    g(0)&=\psi_{f(\alpha)}\circ f\circ\psi_\alpha^{-1}(0)\\
    &=\psi_{f(\alpha)}(f(\alpha))=0
\end{align*}
である。

よって、Schwarzの定理(p.102)から、$|g(z)|\le|z|$が成り立つ。

ゆえに、$|z|<1$に対して、
\begin{align*}
    |g(\psi_\alpha(z))|&\le|\psi_\alpha(z)|\\
    |\psi_{f(\alpha)}(f(z))|&\le|\psi_\alpha(z)|\\
    \left|\frac{f(z)-f(\alpha)}{1-\overline{f(\alpha)}f(z)}\right|&\le\left|\frac{z-\alpha}{1-\overline{\alpha}z}\right|
\end{align*}
が成り立つ。
したがって、$\log\frac{1+r}{1-r}$が$0\le r<1$において単調増加であることから、$D(f(z),f(\alpha))\le D(z,\alpha)$が成り立つ。

等号が成り立つのは、p.102より
\begin{align*}
    g(z)=e^{i\theta}z
\end{align*}
のときである。このとき、
\begin{align*}
    f(z)&=\psi_{f(\alpha)}^{-1}\circ g\circ\psi_\alpha(z)
\end{align*}
であり、これは
\begin{align*}
    \begin{bmatrix}
        1&f(\alpha)\\
        \overline{f(\alpha)}&1
    \end{bmatrix}
    \begin{bmatrix}
        e^{i\theta}&0\\
        0&1
    \end{bmatrix}
    \begin{bmatrix}
        1&-\alpha\\
        -\overline{\alpha}&1
    \end{bmatrix}
\end{align*}
に対応する一次函数である。(証明終)