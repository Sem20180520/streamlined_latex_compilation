\chapter{正則函数}%第5章

\begin{mysimplebox}{問1}
   $f(z)=u(r,\theta)+iv(r,\theta)$($z=re^{\theta i}$)が$|z|<R$で正則ならば
   \begin{align*}
    \frac{f^{(n)}(0)}{n!}
    =\frac{1}{\pi r^n}\int_{0}^{2\pi}u(r,\theta)e^{-in\theta}d\theta\quad(0<r<R)
   \end{align*}
\end{mysimplebox}
\paragraph{証明1}(泥臭い方法)
書籍のp.87の§28より、原点を中心とする半径$R$の円の内部において$f(z)$はべき級数で表される。
すなわち
\begin{align*}
    f(z)=c_0+c_1z+c_2+\dots+c_kz^k+\dots
    =\sum_{k=0}^{\infty}c_kz^k
\end{align*}
$\frac{f^{(n)}(0)}{n!}=c_n$である。

ここで$c_k=a_k+ib_k$($a_k,b_k\in\R$)とし、$f(z)$に$z=re^{\theta i}$を代入すると
\begin{align*}
    f(re^{\theta i})
    &=\sum_{k=0}^{\infty}(a_k+ib_k)r^ke^{k\theta i}\\
    &=\sum_{k=0}^{\infty}r^k(a_k+ib_k)(\cos k\theta+i\sin k\theta)\\
    &=\sum_{k=0}^{\infty}r^k\{a_k\cos k\theta-b_k\sin k\theta+i(a_k\sin k\theta+b_k\cos k\theta)\}
\end{align*}
$f(z)=u(r,\theta)+iv(r,\theta)$であるから
\begin{align*}
    u(r,\theta)
    =\sum_{k=0}^{\infty}r^k(a_k\cos k\theta-b_k\sin k\theta)
\end{align*}
$n=0$のときは$\int_{0}^{2\pi}u(r,\theta)d\theta=a_0\int_{0}^{2\pi}d\theta=2\pi a_0\neq \pi c_0$であり、成り立たない。よって$n\neq0$とする。

\begin{align*}
    u(r,\theta)e^{-in\theta}
    =&\sum_{k=0}^{\infty}r^k(a_k\cos k\theta-b_k\sin k\theta)(\cos n\theta-i\sin n\theta)\\
    =&\sum_{k=0}^{\infty}r^k
    \{a_k\cos k\theta\cos n\theta-b_k\sin k\theta\cos n\theta
    \\
    &-i(a_k\cos k\theta\sin n\theta-b_k\sin k\theta\sin n\theta)\}
\end{align*}
$k\neq n$のとき
\begin{align*}
    %%%cos cos
    \int_{0}^{2\pi}\cos k\theta\cos n\theta d\theta
    &=\int_{0}^{2\pi}\frac{1}{2}\{\cos(k+n)\theta+\cos(k-n)\theta\} d\theta\\
    &=\frac{1}{2}\left[\frac{\sin(k+n)\theta}{k+n}+\frac{\sin(k-n)\theta}{k-n}\right]_0^{2\pi}=0\\
    %%%sin cos
    \int_{0}^{2\pi}\sin k\theta\cos n\theta d\theta
    &=\int_{0}^{2\pi}\frac{1}{2}\{\sin(k+n)\theta+\sin(k-n)\theta\} d\theta\\
    &=-\frac{1}{2}\left[\frac{\cos(k+n)\theta}{k+n}+\frac{\cos(k-n)\theta}{k-n}\right]_0^{2\pi}=0\\
    %%%sin sin
    \int_{0}^{2\pi}\sin k\theta\sin n\theta d\theta
    &=\int_{0}^{2\pi}\frac{1}{2}\{\cos(k+n)\theta-\cos(k-n)\theta\} d\theta\\
    &=\frac{1}{2}\left[\frac{\sin(k+n)\theta}{k+n}+\frac{\sin(k-n)\theta}{k-n}\right]_0^{2\pi}=0
\end{align*}
一方、$k=n\neq0$のとき
\begin{align*}
    %%%cos cos
    \int_{0}^{2\pi}\cos n\theta\cos n\theta d\theta
    &=\int_{0}^{2\pi}\frac{1}{2}(\cos 2n\theta+1) d\theta\\
    &=\frac{1}{2}\left[\frac{\sin 2n\theta}{2n}+\theta\right]_0^{2\pi}=\pi\\
    %%%sin cos
    \int_{0}^{2\pi}\sin n\theta\cos n\theta d\theta
    &=\int_{0}^{2\pi}\frac{1}{2}\sin 2n\theta d\theta\\
    &=-\frac{1}{2}\left[\frac{\cos 2n\theta}{2n}\right]_0^{2\pi}=0\\
    %%%sin sin
    \int_{0}^{2\pi}\sin n\theta\sin n\theta d\theta
    &=\int_{0}^{2\pi}\frac{1}{2}\{1-\cos 2n\theta\} d\theta\\
    &=\frac{1}{2}\left[\theta-\frac{\sin 2n\theta}{2n}\right]_0^{2\pi}=\pi
\end{align*}
したがって
\begin{align*}
    \int_{0}^{2\pi}u(r,\theta)e^{-in\theta}d\theta
    &=r^n(\pi a_n+i\pi b_n)=\pi r^nc_n\\
    \frac{1}{\pi r^n}\int_{0}^{2\pi}u(r,\theta)e^{-in\theta}d\theta
    &=c_n
\end{align*}
これで$n\neq0$のときは示せた。
\paragraph{補足}
\begin{align*}
    \int_{0}^{2\pi}f(re^{\theta i})e^{-in\theta}d\theta
    &=\int_{0}^{2\pi}u(r,\theta)e^{-in\theta}d\theta
    +i\int_{0}^{2\pi}v(r,\theta)e^{-in\theta}d\theta\\
    &=2\pi c_nr^n
\end{align*}
よって、$n\neq0$に対して
\begin{align*}
    i\int_{0}^{2\pi}v(r,\theta)e^{-in\theta}d\theta
    &=\pi c_nr^n
\end{align*}
また、$n=0$のとき
\begin{align*}
    i\int_{0}^{2\pi}v(r,\theta)d\theta
    &=2\pi ib_0
\end{align*}

\paragraph{証明2}
$u(r,\theta)=\frac{1}{2}\{f(z)+\overline{f(z)}\}$であるから、$n\ge1$のとき、
\begin{align*}
    \frac{1}{\pi r^n}\int_{0}^{2\pi}u(r,\theta)e^{-in\theta}d\theta
    =\frac{1}{2\pi r^n}\int_{0}^{2\pi}f(z)e^{-in\theta}d\theta
     +\frac{1}{2\pi r^n}\int_{0}^{2\pi}\overline{f(z)}e^{-in\theta}d\theta
\end{align*}
右辺の第1項の積分は、
\begin{align*}
    \frac{1}{2\pi r^n}\int_{0}^{2\pi}f(z)e^{-in\theta}d\theta
    &=\frac{1}{2\pi}\int_{0}^{2\pi}\frac{f(z)}{(re^{i\theta})^n}d\theta
    =\frac{1}{2\pi}\int_{0}^{2\pi}\frac{f(z)}{z^n}\frac{d(re^{i\theta})}{ire^{i\theta}}\\
    &=\frac{1}{2\pi i}\oint\frac{f(z)}{z^{n+1}}dz=\frac{f^{(n)}(0)}{n!}
\end{align*}
第2項の積分は、
\begin{align*}
    \int_{0}^{2\pi}\overline{f(z)}e^{-in\theta}d\theta
    &=\overline{\int_{0}^{2\pi}f(z)e^{in\theta}d\theta}
    =\overline{\oint f(z)\frac{z^n}{r^n}\frac{dz}{iz}}\\
    &=-\frac{1}{ir^n}\overline{\oint f(z)z^{n-1}dz}=0
\end{align*}
最後の部分は、この被積分函数が積分路内で正則であるため、積分の値は0だからである。

したがって、
\begin{align*}
    \frac{1}{\pi r^n}\int_{0}^{2\pi}u(r,\theta)e^{-in\theta}d\theta
    =\frac{f^{(n)}(0)}{n!}
\end{align*}

一方、$n=0$のとき、
\begin{align*}
    \frac{1}{\pi}\int_{0}^{2\pi}u(r,\theta)d\theta
    &=\frac{1}{2\pi}\int_{0}^{2\pi}f(z)d\theta
     +\frac{1}{2\pi}\int_{0}^{2\pi}\overline{f(z)}d\theta\\
    &=\frac{1}{2\pi}\oint f(z)\frac{dz}{iz}
    +\frac{1}{2\pi}\overline{\oint f(z)\frac{dz}{iz}}\\
    &=\frac{1}{2\pi i}2\pi if(0)-\frac{1}{2\pi i}\overline{2\pi if(0)}\\
    &=f(0)+\overline{f(0)}=2\Re(f(0))
\end{align*}
よって、$n=0$のとき、問題の等式は一般には成り立たない。(証明終)

\paragraph{例}
$f(z)=z=r\cos\theta+ir\sin\theta$のとき
\begin{align*}
    \frac{f^{(n)}(0)}{n!}=
    \begin{cases}
        0 & (n=0)\\
        1 & (n=1)\\
        0 & (n\ge2)
    \end{cases}
\end{align*}
\begin{align*}
    \frac{1}{\pi r^n}\int_{0}^{2\pi}r\cos\theta e^{-in\theta}d\theta
    &=\frac{1}{\pi r^n}\int_{0}^{2\pi}r\frac{e^{i\theta}+e^{-i\theta}}{2} e^{-in\theta}d\theta\\
    &=\begin{cases}
        0 & (n=0)\\
        1 & (n=1)\\
        0 & (n\ge2)
    \end{cases}
\end{align*}
よって、この場合は$n=0$でも偶然成り立つ。

しかし、$g(z)=z+1=r\cos\theta+1+ir\sin\theta$のとき
\begin{align*}
    \frac{g^{(n)}(0)}{n!}=
    \begin{cases}
        1 & (n=0)\\
        1 & (n=1)\\
        0 & (n\ge2)
    \end{cases}
\end{align*}
\begin{align*}
    \frac{1}{\pi r^n}\int_{0}^{2\pi}(r\cos\theta+1) e^{-in\theta}d\theta
    &=\frac{1}{\pi r^n}\int_{0}^{2\pi}r\left\{\frac{e^{i\theta}+e^{-i\theta}}{2}+1\right\} e^{-in\theta}d\theta\\
    &=\begin{cases}
        2 & (n=0)\\
        1 & (n=1)\\
        0 & (n\ge2)
    \end{cases}
\end{align*}
よって、この場合は$n=0$で成り立たない。(例終)

\begin{mysimplebox}{問2}
    $f(z)$が$D$で正則ならば$D$の各点$z$で
    \begin{align*}
     \frac{\partial^2|f(z)|^2}{\partial x^2}
     +\frac{\partial^2|f(z)|^2}{\partial y^2}
     =4|f'(z)|^2\quad(z=x+iy)
    \end{align*}
\end{mysimplebox}
\paragraph{証明}
$f(z)=P(x,y)+iQ(x,y)$($P(x,y),Q(x,y)$は実函数)とすると
\begin{align*}
    \frac{\partial^2|f(z)|^2}{\partial x^2}
   &=\frac{\partial^2}{\partial x^2}\left\{P(x,y)^2+Q(x,y)^2\right\}\\
   &=\frac{\partial}{\partial x}\left\{2PP_x+2QQ_x\right\}\\
   &=2\left\{{P_x}^2+PP_{xx}+{Q_x}^2+QQ_{xx}\right\}\\
\end{align*}
よって
\begin{align*}
    \frac{\partial^2|f(z)|^2}{\partial x^2}
     +\frac{\partial^2|f(z)|^2}{\partial y^2}
     &=2\left\{P_x^2+P_y^2+P(P_{xx}+P_{yy})+Q_x^2+Q_y^2+Q(Q_{xx}+Q_{yy})\right\}\\
     &=4(P_x^2+P_y^2)=4|f'(z)|^2
\end{align*}
これで示せた。
ただし、$f(z)$が$D$において正則であることから、次のことを利用した。
\begin{align*}
    &P_x=Q_y,& &P_y=-Q_x\\
    &P_{xx}=Q_{yx}=Q_{xy}=-P_{yy},    & &Q_{xx}=-P_{yx}=-P_{xy}=-Q_{yy}\\
    &f'(z)=P_x+iQ_x=P_x-iP_y,& &|f'(z)|^2=P_x^2+P_y^2
\end{align*}
(証明終)

\paragraph{補足}
この証明中で正則函数の実部と虚部はそれぞれ調和函数であることを示した。

$P_{xx}+P_{yy}=0$、$Q_{xx}+Q_{yy}=0$

\begin{mysimplebox}{問3}
    領域$D$で$f(z),g(z)$が正則で$f(z)\cdot g(z)\equiv0$ならば、
    $f(z)\equiv0$または$g(z)\equiv0$でなければならない。
\end{mysimplebox}
\paragraph{証明1}
$c\in D$を固定し、$z\in D$は$c$の十分近くにある任意の点とする。
$f(z),g(z)$が正則であることから
\begin{align*}
    f(z)&=a_0+a_1(z-c)+a_2(z-c)^2+\cdots\\
    g(z)&=b_0+b_1(z-c)+b_2(z-c)^2+\cdots
\end{align*}
と表される。

$g(z)\equiv0$ではないと仮定すると、$b_n\neq0$となる最小の$n\in\N$が存在する。
よって
\begin{align*}
    g(z)&=b_n(z-c)^n+b_{n+1}(z-c)^{n+1}+b_{n+2}(z-c)^{n+2}+\cdots
\end{align*}
ゆえに
\begin{align*}
    f(z)g(z)=&a_0b_n(z-c)^n+(a_0b_{n+1}+a_1b_n)(z-c)^{n+1}
    +(a_0b_{n+2}+a_1b_{n+1}+a_2b_n)(z-c)^{n+2}\\
    &+\cdots+(a_0b_{n+k}+a_1b_{n+k-1}+\cdots+a_{n+k-1}b_1+a_{n+k}b_0)(z-c)^{n+k}+\cdots
\end{align*}
ここで、$f(z)g(z)=0,b_n\neq0$であることから
\begin{align*}
    a_0=0,a_1=0,a_2=0,\dots,a_{n+k}=0,\dots
\end{align*}
となり、$f(z)=0$であることがわかる。
$c\in D$は任意であったから、結局$D$において$f(z)\equiv0$であることとなる。(証明終)

%※p.89の内容に則して修正が必要
\paragraph{証明2}
$f(z)g(z)\equiv0$であるとき、$g(z)\not\equiv0$ならば$f(z)\equiv0$であることを示す。

$f(z)g(z)\equiv0$より、$z_k\to c$、$z_k\neq c$、$z_k\in D$、$c\in D$である点列$\{z_k\}$の各点において$f(z_k)g(z_k)=0$である。

$g\not\equiv0$の仮定より、$g(z_k)\neq0$である正の整数$k$が無限に存在する。そのような$k$を選んで$\{z_k\}$の部分列$\{z_{k_l}\}$を用意する。

$f(z_{k_l})g(z_{k_l})=0$かつ$g(z_k)\neq0$であるから$f(z_k)=0$である。また、$z_{k_l}\to c$、$z_{k_l}\neq c$、$z_{k_l}\in D$、$c\in D$
である。

したがって、p.88の定理から$f(z)\equiv0$である。(証明終)

\begin{mysimplebox}{問4}
    $f(z)$が$|z|<R$で正則であるとき、円$|z|=r$($r<R$)の写像$w=f(z)$による像曲線の長さを$L(z)$で表せば
    \begin{align*}
        L(r)\ge2\pi r|f'(0)|
    \end{align*}
\end{mysimplebox}
\paragraph{証明1}
%\begin{align*}
%    L(r)=\int_{0}^{2\pi}|f'(z(r\theta))|rd\theta
%\end{align*}
%ここで、$|z|=r$において、$f(z)=c_0+c_1z+c_2z^2+\dots$と表される。
%よって、
%\begin{align*}
%    f'(z)&=c_1+2c_2z+\dots\\
%    |f'(z)|&\ge|c_1|=|f'(0)|
%\end{align*}※この不等式は普通に間違っている。
%ゆえに、
%\begin{align*}
%    L(r)&\ge|f'(0)|r\int_{0}^{2\pi}d\theta=2\pi r|f'(0)|
%\end{align*}
半径$r$の円周$C_r$上の点は、弧の長さ$s=r\theta$によるパラメータ表示で、$z=z(s)=r\left(\cos\frac{s}{r}+i\sin\frac{s}{r}\right)=re^{i\frac{s}{r}}$($0\le s \le2\pi r$)と書ける。

\begin{align*}
    dz
    =i\frac{1}{r}re^{i\frac{s}{r}}ds
\end{align*}
であるから、$ds=\frac{r}{iz}dz$である。

写像$w=f(z)$による円周の像は$w=f(e^{i\frac{s}{r}})$($0\le s\le2\pi r$)である。その長さは以下のようになる。
\begin{align*}
    L(r)&=\int_{0}^{2\pi r}|f'(z(s))|ds
    \ge \left|\int_{0}^{2\pi r}f'(z(s))ds\right|
    =\left|\int_{C_r}f'(z)\frac{r}{iz}dz\right|\\
    &=r\left|\int_{C_r}\frac{f'(z)}{iz}dz\right|
    =r\left|2\pi i \frac{f'(0)}{i}\right|
    =2\pi r|f'(0)|
\end{align*}
(証明終)
\paragraph{証明2}(根本から計算する方法)
$f(z)=u(x,y)+iv(x,y)$と実部、虚部に分けて書く。極座標表示で
\begin{align*}
    u&=u(r\cos\theta,r\sin\theta)\\
    v&=v(r\cos\theta,r\sin\theta)
\end{align*}
である。よって、以下のように微分を計算できる。
\begin{align*}
    du&=-ru_x\sin\theta d\theta+ru_y\cos\theta d\theta\\
    &=r(-u_x\sin\theta+u_y\cos\theta)d\theta\\
    &=r(-v_y\sin\theta-v_x\cos\theta)d\theta\\
    dv&=r(-v_x\sin\theta+v_y\cos\theta)d\theta
\end{align*}
ここで、コーシー・リーマンの関係式を利用した。以上より、
\begin{align*}
    ds^2=du^2+dv^2=r^2(v_x^2+v_y^2)d\theta^2
\end{align*}
よって、
\begin{align*}
    ds=r\sqrt{v_x^2+v_y^2}d\theta=r|f'(re^{i\theta})|d\theta
\end{align*}
ゆえに
\begin{align*}
    L(r)&=\int_{\Gamma}ds=r\int_{0}^{2\pi}|f'(re^{i\theta})|d\theta
\end{align*}
この後は前述の計算と同じである。(証明終)
\paragraph{例}
$f(z)=a+bz$($a,b\in\C$)のとき、
\begin{align*}
    L(r)&=\int_{0}^{2\pi}|f'(z)|rd\theta=\int_{0}^{2\pi}|b|rd\theta\\
    &=2\pi r|b|=2\pi r|f'(0)|
\end{align*}
この場合は等号が成立する。
この$f(z)$は$b$倍して$a$だけ平行移動する写像であるから、$2\pi r$の円周は$2\pi r|b|$の円周に移される。(例終)

\begin{mysimplebox}{問5}
    領域$D$で正則な函数$f(z)$が$D$の境界の各点$\zeta$で境界値$\phi(\zeta)$を有するとき、
    $|\phi(\zeta)|=M$($M$は定数)ならば、$f(z)$が$D$で定数でない限り、$f(z)$は$D$で零点を有する。
\end{mysimplebox}
\paragraph{証明}
\paragraph{$D$が有界であるとき}

$f(z)$が$D$で零点を持たないと仮定する。
$g(z)=\frac{1}{f(z)}$とすると、$g(z)$は$D$で正則である。
最大値の原理により、$|g(z)|$は$\partial D$において最大値を持つ。
よって、$|f(z)|$は$\partial D$において最小値を持つ(p.100参照)。

しかし、$\zeta\in\partial D$において$|\phi(\zeta)|=M$であるから、
$|f(z)|$の最大値と最小値はともに$M$である。
よって、$D$の任意の点$z$において、$|f(z)|=M$であるから、
$f(z)$は$D$で定数である(p.99参照)。

したがって、$f(z)$が$D$で定数でないならば、$f(z)$は$D$で零点を有する。

\paragraph{$D$が有界でないとき}
$D$の外点$a$をとり、$\frac{1}{z-a}=w$とする。
この変換によって、$a$は$\infty$に、$\infty$は0に移る。

$D$において$z\neq a$であるから、
$D$と$a$の距離を$d$とすると、$|z-a|>d>0$である。
よって、$|w|=\frac{1}{|z-a|}<\frac{1}{d}<\infty$である。
ゆえに、$w$は有界な領域内を動く。その領域を$D'$とする。
\begin{align*}
    f(z)=f\left(\frac{1}{w}+a\right)
\end{align*}
であるから、$f\left(\frac{1}{w}+a\right)=h(w)$とすると、
$h(w)$は有界な領域$D'$で正則であり、境界値が$M$であるから、
前半の場合に帰着できる。(証明終)

\paragraph{例}
$f(z)=z$は単位円内において正則であり、境界値$e^{i\theta}$を有する。
境界において$|z|=1$である。原点が零点である。

(あまり良い例が思い浮かばない)

\begin{mysimplebox}{問6}
    $f(z)$は有界領域$D$で正則な函数であるとする。
    領域$D$のどの境界点$\zeta$、どの正数$\epsilon$を与えても、
    $z\in D$、$|z-\zeta|<\delta(\zeta,\epsilon)$ならば
    $|f(z)|<M+\epsilon$($M$は定数)であるように正数$\delta(\zeta,\epsilon)$を選びうるときは、
    $D$の各点$z$で$|f(z)|\le M$である。
    なお、このとき、$z\in D$、$|f(z)|=M$なる$z$があれば$f(z)\equiv Me^{\theta i}$である。
\end{mysimplebox}
\paragraph{証明}
ある$D$の点$z_0$において、$|f(z_0)|>M$と仮定する。

$\epsilon_0=|f(z_0)|-M>0$とする。

$D$の任意の境界点$\zeta$に対して、
$z\in D$、$|z-\zeta|<\delta(\zeta,\epsilon_0)$ならば、
$|f(z)|<M+\epsilon_0=|f(z_0)|$となる正数$\delta(\zeta,\epsilon_0)$をとることができる。境界の各点$\zeta$に対して、そのような$\delta(\zeta,\epsilon_0)$を選んでおく。

\begin{align*}
    B_\zeta=\{z\in\C\mid|z-\zeta|<\delta(\zeta,\epsilon_0)\}
\end{align*}
とする。

\begin{figure}[h]
    \centering
    \includegraphics[width=5cm]{chap5_fig/prob6-1.png}
    \caption{領域$D$}
    \label{fig:chap5-6-1}
\end{figure}

\begin{align*}
    \partial D\subset\bigcup_{\zeta\in\partial D}B_\zeta
\end{align*}
である。

$\partial D$は有界閉集合、すなわちコンパクトであるから、
有限個の$B_\zeta$で覆うことができる。
したがって、ある有限個の境界点$\zeta_k$($k=1,\dots,n$)に対して
\begin{align*}
    \partial D\subset\bigcup_{k=1}^{n}B_{\zeta_k}
\end{align*}
とできる。

\begin{figure}[h]
    \centering
    \includegraphics[width=5cm]{chap5_fig/prob6-2.png}
    \caption{$D$の境界を有限個の円板で覆う}
    \label{fig:chap5-6-2}
\end{figure}

\begin{align*}
    D'&:=D\setminus\overline{\bigcup_{k=1}^{n}B_{\zeta_k}}\\
    c&:=\partial D'
\end{align*}
とする。

\begin{figure}[h]
    \centering
    \includegraphics[width=5cm]{chap5_fig/prob6-3.png}
    \caption{$D':=D\setminus\overline{\bigcup_{k=1}^{n}B_{\zeta_k}}$(図中の数式にバーがないのは誤り)}
    \label{fig:chap5-6-3}
\end{figure}

$z\in D\cap \bigcup_{k=1}^{n}B_{\zeta_k}$に対して$|f(z)|<|f(z_0)|$であるから$z_0\not\in D\cap \bigcup_{k=1}^{n}B_{\zeta_k}$である。
よって$z_0\in D'$である($z_0\notin\overline{D'}\setminus D$であることは後で分かる)。$c$は有限個の$B_{\zeta_k}$の境界をつなげた曲線である。個々の$B_{\zeta_k}$の境界は円弧である。

実は$D'$は連結である保証はないが、以降では$z_0$を含む連結成分を考えればよい。

$\overline{D'}\subset D$であるから、
$f(z)$は$D'$で正則であり、$\overline{D'}$で連続である。

$\overline{D'}$における$|f(z)|$の最大値を$M'$とすると、最大値の原理により、$|f(z')|=M'$となる$\partial D'=c$上の点$z'$が存在する(p.99)。

$M'=|f(z_0)|$とすると、$f(z)$は$\overline{D'}$で定数であり、よって一致の定理から$D$でも定数となる。しかし、$D\setminus\overline{D'}$の点$z$では$|f(z)|<|f(z_0)|$であるから不合理である。ゆえに、$M'>|f(z_0)|>M$である。
$d=M'-M>0$とする。

ここで、$z'\in\overline{B_{\zeta_k}}$となる$k$($1\le k\le n$)が存在する。$z'$の任意の近傍に$D\cap B_{\zeta_k}$の点$z$が存在する。その点$z$において、$|f(z)|<M$である。よって$|f(z')|-|f(z)|=M'-|f(z)|>M'-M=d>0$であるから$f(z)$が連続であることに反する。

したがって、$D$の各点$z$で$|f(z)|\le M$である。


%よって、$c\in (\bigcup_{\zeta\in \partial D}B{\zeta})\cap D$である曲線をとれば、$c$で囲まれる領域において$f(z)$は正則である。この領域を$D'$とする。$z_0\in D'$である。$z\in c$において
%$|f(z)|<|f(z_0)|$である。%%しかし、最大値の原理より、$c$上のある点$z$において、$|(z)|$は$\overline{D'}$における最大値をとる。ゆえに、$|f(z_0)|\le|f(z)|$である。これは不合理である。

後半について、$|f(z)|=M$となる$z\in D$があれば$f(z)\equiv Me^{\theta i}$
であることはp.99参照。(証明終)

\paragraph{感想}
最大値の原理は、$f(z)$が有界な領域$D$で正則であり、
かつ$\overline{D}$で連続であるときに適用できる(p.99)。

$f(z)$が有界な閉領域$\overline{D}$で連続ならば、
$f(z)$は$\overline{D}$で有界であることが重要である(p.30)。

ところが、この問6では、$\overline{D}$での連続性が仮定されていない。
そして、境界の近傍での有界性は仮定されているが、
$\overline{D}$での有界性は仮定されていない。

そこで、領域$D$をうまく縮めて、$f(z)$が連続な閉領域$\overline{D'}$をつくり、
最大値の原理を利用した。

その結果、有界領域について境界の近傍での正則函数の有界性から、
もとの有界領域での正則函数の有界性が導けた。

しかし、背理法によるこの証明は、なかなか矛盾に辿り着けず、
難儀している。

%$(\bigcup_{k=1}^{n}B_{\zeta_k})\cap D$に含まれる積分路$c$に対して、
%\begin{align*}
%    |f(z_0)|=|\frac{1}{2\pi i}\int_{c}\frac{f(z)}{z-z_0}dz|
%    \le \frac{1}{2\pi}\int_{c}|\frac{f(z)}{z-z_0}|dz
%    < \frac{|f(z_0)|}{2\pi}\int_{c}\frac{1}{z-z_0}dz
%\end{align*}
%
%$z_0\in D$、$|z_0-z_0|=0<\delta(z_0,\epsilon_0)$であるから、
%$|f(z_0)|<|f(z_0)|$となり、不合理である。よって、$z_0$は$D$の境界点ではありえない。
%
%よって、$z_0$が$D$の境界点でないとする。
%最大値の原理により、$|z|>|z_0|$である$z\in D$
%
%
%
%$z$と$\partial D$の距離を$r$

\begin{mysimplebox}{問7}
    $f(z)$が整函数のとき、$|z|=r$における$|f(z)|$、$|f'(z)|$の最大値をそれぞれ$M(r)$、$M_1(r)$とすれば、次の不等式が成り立つことを示せ。
    \begin{align*}
        \frac{M(r)-|f(0)|}{r}\le M_1(r)\le\frac{M(R)}{R-r}
        \quad(0<r<R)
    \end{align*}
\end{mysimplebox}
\paragraph{証明}
微分積分学の基本定理から
\begin{align*}
    \int_{0}^{z}f'(z)dz=f(z)-f(0)
\end{align*}
よって、$f(z)=f(0)+\int_{0}^{z}f'(z)dz$である。

積分路を円の半径に沿ってとる。
$z=te^{i\theta}$($\theta$は一定)とし、$t=0$から$t=r$まで積分する。

\begin{figure}[h]
    \centering
    \includegraphics[width=5cm]{chap5_fig/prob7-1.png}
    \caption{円の半径に沿った積分路}
    \label{fig:chap5-7-1}
\end{figure}

$dz=e^{i\theta}dt$であるから
\begin{align*}
    f(z)&=f(re^{i\theta})
    =f(0)+\int_{0}^{r}f'(te^{i\theta})e^{i\theta}dt\\
    |f(re^{i\theta})|
    &=\left|f(0)+\int_{0}^{r}f'(te^{i\theta})e^{i\theta}dt\right|\\
    &\le|f(0)|+\left|\int_{0}^{r}f'(te^{i\theta})e^{i\theta}dt\right|\\
    M(r)&\le|f(0)|+M_1(r)r
\end{align*}
よって
\begin{align*}
    \frac{M(r)-|f(0)|}{r}\le M_1(r)
\end{align*}

また、コーシーの積分公式から
\begin{align*}
    f'(z)&=\frac{1}{2\pi i}\int_{|\zeta|=R}\frac{f(\zeta)}{(\zeta-z)^2}d\zeta
    =\frac{1}{2\pi i}\int_{|\zeta-z|=R-r}\frac{f(\zeta)}{(\zeta-z)^2}d\zeta
\end{align*}
である。ここでは$f(\zeta)$が整函数であり、$\C$において正則であることから、積分路を中心$z$、半径$R-r$の円周にとりなおしている。

ここで、$|\zeta-z|=R-r$から、$\zeta=z+(R-r)e^{i\theta}$である。

\begin{figure}[h]
    \centering
    \includegraphics[width=5cm]{chap5_fig/prob7-2.png}
    \caption{積分路の変更}
    \label{fig:chap5-7-2}
\end{figure}

よって、
\begin{align*}
    f'(z)&=\frac{1}{2\pi i}\int_{0}^{2\pi}
    \frac{f(z+(R-r)e^{i\theta})}{(R-r)^2e^{2i\theta}}(R-r)ie^{i\theta}d\theta\\
    &=\frac{1}{2\pi(R-r)}\int_{0}^{2\pi}
    f(z+(R-r)e^{i\theta})e^{-i\theta}d\theta
\end{align*}
ゆえに
\begin{align*}
    |f'(z)|\le\frac{1}{2\pi(R-r)}2\pi M(R)=\frac{M(R)}{R-r}
\end{align*}
したがって、$M_1(r)\le\frac{M(r)}{R-r}$である。(証明終)

\paragraph{感想}
$f(z)$と$f'(z)$に関する式として、微分積分学の基本定理とコーシーの積分定理がある。前者から左側の不等式、後者から右側の不等式が得られる。そのことに気付けず、証明を思いつくことができなかった。最大値をとって$M(r)$や$M_1(r)$などが出てくるときに、$f(z)$などが姿を消すことも、難しいと感じた。

\begin{mysimplebox}{問8}
    Riemannのゼータ函数$\zeta(z)=\sum_{n=1}^{\infty}n^{-z}$は領域$\Re(z)>1$で正則な函数であることを示せ。ただし、$n^{-z}=e^{-z\log n}$。
\end{mysimplebox}
\paragraph{証明}
$D:=\Re(z)>1$とする。
$D$における任意の有界閉領域$\Delta$で、$x_\Delta:=\min_{z\in\Delta}\{\Re(z)\}$とすると、$x_\Delta>1$である。

よって、$z\in\Delta$において、
\begin{align*}
    |\zeta(z)|&\le\sum_{n=1}^{\infty}|n^{-z}|
    =\sum_{n=1}^{\infty}|e^{-z\log n}|
    =\sum_{n=1}^{\infty}|e^{-\Re(z)\log n}|
    =\sum_{n=1}^{\infty}|n^{-\Re(z)}|
    =\sum_{n=1}^{\infty}\frac{1}{n^{\Re(z)}}\\
    &\le\sum_{n=1}^{\infty}\frac{1}{n^{x_\Delta}}
    <\infty
\end{align*}
が成り立つ。

ゆえに、$\zeta(z)$は絶対収束するから、$\zeta(z)$自体も収束する。
この収束は$\Delta$の各点で成り立つから、
%ゆえに、
%\begin{align*}
%    S_m:=\sum_{n=1}^{m}n^{-z}
%\end{align*}
%とすると、正則な函数列$\{S_m(z)\}$は$\Re(z)>1$
$\zeta(z)$は$D$で広義の一様収束をする。

したがって、Weierstrassの二重級数定理(p.104)から、極限函数$\zeta(z)$は$\Re(z)>1$で正則である。
(証明終)

\paragraph{補足}
この後、Weierstrassの二重級数定理が頻繁に使われる。
この定理はコーシーの積分定理から得られるということが重要である。

\begin{mysimplebox}{問9}
    領域$D$で正則な函数ばかりからなる函数族$\mathfrak{F}$が$D$の各点の近傍で正規族ならば、$\mathfrak{F}$は$D$で正規族である。
\end{mysimplebox}
\paragraph{証明}
$\mathfrak{F}$に属する任意の函数列$\{f_n(z)\}$が、$D$で広義の一様収束をすることを示す。そのためには、$D$における任意の有界閉領域$\Delta$において、一様収束する部分列を有することを示せばよい。

$\Delta$の任意の点$z$において、$z$の近傍で$\mathfrak{F}$が正規族となるようなものがある。それを$V_z$とする。

\begin{align*}
    \Delta\subset\bigcup_{z\in\Delta}V_z
\end{align*}
であり、$\Delta$は有界閉集合(コンパクト)であるから、有限個の$\Delta$の点$\{z_1,z_2,\dots,z_l\}$で、
\begin{align*}
    \Delta\subset\bigcup_{k=1,2,\dots,l}V_{z_k}
\end{align*}
となるものがある。

$V_{z_1}$で$\mathfrak{F}$は正規族であるから、$\{f_n(z)\}$は$V_{z_1}$で広義一様収束する部分列をもつ。それを改めて$\{f_n(z)\}$とする。

$V_{z_2}$でも$\mathfrak{F}$は正規族であるから、$\{f_n(z)\}$は$V_{z_2}$で広義一様収束する部分列をもつ。再び、それを改めて$\{f_n(z)\}$とする。

この操作を$l$回繰り返し、$\{f_n(z)\}$は$\Delta$で広義一様収束する部分列をもつことが分かる。

ゆえに、$\{f_n(z)\}$は$\Delta$で一様収束する部分列をもつ。

したがって、$\{f_n(z)\}$は$D$で広義一様収束する部分列をもつから、$\mathfrak{F}$は$D$で正規族である。
(証明終)






\begin{mysimplebox}{問10}
    領域$D$での正規函数族$\mathfrak{F}$に属する函数がすべて$D$で正則ならば、$\mathfrak{F}$に属する函数の導函数全部からなる函数族も$D$で正規族である。
\end{mysimplebox}
\paragraph{証明}
$\mathfrak{F}$に属する函数の導函数全部からなる函数族を$\mathfrak{F'}$とする。

$\mathfrak{F'}$の函数列は、$\mathfrak{F}$の函数列$\{f_n(z)\}$によって、$\{f'_n(z)\}$と表される。

$\mathfrak{F}$は正規族であるから、$\{f_n(z)\}$は$D$において広義一様収束する部分列$\{f_{n_k}(z)\}$を有する。

このとき、Weierstrassの二重級数定理から、$\{f'_{n_k}(z)\}$も$D$において広義一様収束する。

よって、$\mathfrak{F'}$は正規族である。(証明終)

\begin{mysimplebox}{問11}
    §36の最後の定理において、$F^{(k)}(z)=\int_{C}\frac{\partial^k\phi(z,\zeta)}{\partial z^k}d\zeta$であることを示せ。
\end{mysimplebox}
\paragraph{証明}
積分路$C$の方程式を$\zeta=\zeta(t)$($0\le t\le1$)とし、$\zeta_\nu=\zeta(\frac{\nu}{n})$($\nu=0,1,2,\dots,n$)とおく。

\begin{align*}
    S^{(k)}_n(z)=\sum_{\nu=1}^{n}\frac{\partial^k \phi(z,\zeta_\nu)}{\partial z^k}(\zeta_\nu-\zeta_{\nu-1})
\end{align*}
とする。

p.112より、$\{S_n(z)\}$は正則な函数の列であり、広義一様収束する。

よって、Weierstrassの二重級数定理から、$\{S^{(k)}_n(z)\}$も$D$で広義一様収束し、その極限函数は$F^{(k)}(z)$である。

また、$S^{(k)}_n(z)\longrightarrow\int_{C}\frac{\partial^k\phi(z,\zeta)}{\partial z^k}d\zeta$である。

以上から、$F^{(k)}(z)=\int_{C}\frac{\partial^k\phi(z,\zeta)}{\partial z^k}d\zeta$である。(証明終)

\begin{mysimplebox}{問12}
    領域$D$の点$z$を与えれば、$\phi(z,t)$は正則、かつ$t$の函数として$0\le t\le+\infty$で連続、また、$z$が$D$のどの点であっても$|\phi(z,t)|\le M(t)$で、しかも$\int_{0}^{+\infty}M(t)dt<+\infty$ならば、$f(z)=\int_{0}^{+\infty}\phi(z,t)dt$は$D$で正則な函数である。これを示せ。
\end{mysimplebox}
\paragraph{証明}
$D$の任意の点$z$において、$T>0$に対して、
\begin{align*}
    \left|\int_{0}^{T}\phi(z,t)dt\right|
    &\le\int_{0}^{T}|\phi(z,t)|dt
    \le\int_{0}^{T}M(t)dt
    \le\int_{0}^{\infty}M(t)dt<+\infty
\end{align*}
よって、$\lim_{T\to\infty}\int_{0}^{T}\phi(z,t)dt$は収束する。$z$は任意であるから、これは一様収束である。

したがって、Weierstrassの二重級数定理より、その極限函数は正則である。(証明終)

\paragraph{補足1}
書籍では条件として$\phi(z,t)$が正則であることが抜けていた。

\paragraph{補足2}
上記では広義積分の収束を調べるために、M--テスト(優関数の方法)を用いている。
それについて説明する。

\begin{align*}
    \int_{0}^{T}\phi(z,t)dt
\end{align*}
が$T\longrightarrow\infty$の極限で収束する必要十分条件は、コーシー条件を満たすことである。すなわち、任意の$\epsilon>0$に対して、ある$T_0>0$が存在し、任意の$a,b$($b>a>T_0$)に対して
\begin{align*}
    \left|\int_{a}^{b}\phi(z,t)dt\right|<\epsilon
\end{align*}
が満たされることである。

この問12の条件では、適当な$T_0$に対して、$b>a>T_0$であるとき、次が成り立つ。
\begin{align*}
    \left|\int_{a}^{b}\phi(z,t)dt\right|
    &\le\int_{a}^{b}|\phi(z,t)|dt
    \le\int_{a}^{b}M(t)dt
    <\epsilon
\end{align*}

上式の最右辺の不等式は、次の広義積分が収束することから言える。
\begin{align*}
    \int_{0}^{\infty}M(t)dt<+\infty
\end{align*}



\begin{mysimplebox}{問13}
    $\Gamma(z)=\int_{0}^{+\infty}e^{-t}t^{z-1}dt$(ただし、$t$は実変数、$t^{z-1}=e^{(z-1)\log t}$)は、半平面$\Re(z)>0$において正則函数で$\Gamma(z+1)=z\Gamma(z)$なる函数方程式を満たす(Eulerの\textbf{ガンマ函数})。これを示せ。
\end{mysimplebox}
\paragraph{証明}
半平面$\Re(z)>0$を$D$とする。
\begin{align*}
    I(z)&=\int_{0}^{1}e^{-t}t^{z-1}dt\\
    J(z)&=\int_{1}^{\infty}e^{-t}t^{z-1}dt
\end{align*}
とする。$I(z)$と$J(z)$が広義積分として値をもつことを示し、さらにWeierstrassの二重級数定理を使って正則であることを示す。

$D$の中の任意の有界閉集合$\Delta$について、
\begin{align*}
    \min_{z\in\Delta}\{\Re(z)\}=x_\Delta
\end{align*}
とする。$D$の定義から$x_\Delta>0$である。

\begin{align*}
    I_n(z)=\int_{1/n}^{1}e^{-t}t^{z-1}dt
\end{align*}
とすると、
\begin{align*}
    |I_n(z)|&=\left|\int_{1/n}^{1}e^{-t}t^{z-1}dt\right|
    \le\int_{1/n}^{1}|e^{-t}t^{z-1}|dt
    \le\int_{1/n}^{1}t^{x_\Delta-1}dt\\
    &=\left[\frac{t^{x_\Delta}}{x_\Delta}\right]_{1/n}^1
    =\frac{1}{x_\Delta}\left\{1-\left(\frac{1}{n}\right)^{x_\Delta}\right\}
    \longrightarrow\frac{1}{x_\Delta}\quad(n\longrightarrow\infty)
\end{align*}

よって、$I(z)$は広義積分として値をもつ。\footnote{ここで問12の補足2で記したM--テストを利用している。}
p.111の定理から、$I_n(z)$は$\Delta$で正則である。
また、上の計算から$\Delta$において$I(z)$に一様収束する。
よって、$I_n(z)$は$D$で$I(z)$に広義一様収束する。
ゆえに、Weierstrassの二重級数定理から、$I_n(z)$は正則である。

$J(z)$も同様の方法で正則であることを示す。

再び$D$の中の任意の有界閉集合$\Delta$で考える。

さきほどと同じ記号を用いて、
\begin{align*}
    \max_{z\in\Delta}\{\Re(z)\}=x_\Delta
\end{align*}
とする。

自然数$m$で、$m\le x_\Delta<m+1$を満たすものが存在する。

$t\ge1$において、
\begin{align*}
    |e^{-t}t^{z-1}|&\le\frac{t^{x_\Delta-1}}{e^t}\\
    &=\frac{t^{x_\Delta-1}}{1+t+\cdots+t^m/m!+t^{m+1}/(m+1)!+\cdots}\\
    &\le\frac{t^{x_\Delta-1}}{t^{m+1}/(m+1)!}
    =(m+1)!t^{t^{x_\Delta-m-2}}
\end{align*}
ここで、$x_\Delta-m-2<-1$である。

よって、
\begin{align*}
    J_n(z)=\int_{1}^{n}e^{-t}t^{z-1}dt
\end{align*}
とすると
\begin{align*}
    |J_n(z)|&=\left|\int_{1}^{n}e^{-t}t^{z-1}dt\right|
    \le\int_{1}^{n}|e^{-t}t^{z-1}|dt\\
    &\le\int_{1}^{n}(m+1)!t^{x_\Delta-m-2}
    =(m+1)!\left[\frac{t^{x_\Delta-m-1}}{x_\Delta-m-1}\right]_1^n
    =\frac{(m+1)!}{x_\Delta-m-1}(n^{x_\Delta-m-1}-1)\\
    &\longrightarrow-\frac{(m+1)!}{x_\Delta-m-1}
\end{align*}
$I(z)$と同じ理由で$J(z)$も正則である。

以上から、$\Gamma(z)=I(z)+J(z)$は$D$で正則であることがわかった。
函数方程式については、実軸の正の部分で、部分積分によって示し、一致の定理によって$D$でも成り立つことが言える。(証明終)